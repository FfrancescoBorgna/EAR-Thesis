% !TEX encoding = UTF-8 Unicode
% !TEX TS-program = pdflatex

% toptesti document class
\documentclass[%
    a4paper, % not needed, by default it is a4paper, or also b5paper can be used
    corpo=12pt, % dimension of basic font
    % oneside is generally the way to go
    oneside, % two side optimizes for two-face printing, having chapters open on the right (aka odd numbers), if you don't want blank pages put oneside here
    stile=standard,
    %evenboxes, % not needed, to put supervisors and candidate at the same level
    tipotesi=magistrale,
    numerazioneromana, % roman numbering for appendixes and preambles, up to Table of Contents
    openright, % to force opening on the right for double-sided printing
    cucitura=7mm, % for printing, 7mm should be enough
    %dvipsnames, % for compatibility with xcolor, it does not work
]{toptesi}

%%%%%%%%%%%%%%%%%%%%%%%%%%%%%%%%%%%%%%%%%%%%%%%%%%%%
\usepackage[english]{babel}
\usepackage[utf8]{inputenc}
\usepackage[T1]{fontenc}
\usepackage{lmodern}

\usepackage{hyperref} % must be loaded before glossaries-extra

% bibliography
\usepackage[hyperref=true,backref=true,maxbibnames=9,maxcitenames=2,style=numeric,citestyle=numeric,sorting=none]{biblatex} % hyperref uses links, backref goes back to citations, uses biber as backend, with 9 names at most in bibliography and 2 in citations, citing using numbers, and sorting in citation order
% sorting can be also ydnt for year descending, name, title or ynt for ascending year

\usepackage{adjustbox} % to resize boxes by keeping the same aspect ratio
\usepackage{algorithm} % algorithm environment
\usepackage{algpseudocode} % improved pseudo-code
\usepackage{amsfonts}               %  AMS mathematical fonts
\usepackage{amsmath}
\usepackage{amssymb}                %  AMS mathematical symbols
\usepackage{bm}                     %  black/bold mathematical symbols
\usepackage{booktabs}               %  better tables
\usepackage[labelfont=bf]{caption} % font=footnotesize % to have reduced caption font size
\usepackage{csquotes}
\usepackage{enumitem} %left align the bulleted points
\usepackage{geometry}
%\usepackage{glossaries} % to use acronyms and glossary, it has also glossaries-extra as extension, but commands are different
\usepackage[%
    toc, % puts the link in the ToC
    %record, % to use bib2gls
    abbreviations, % to load abbreviations / acronyms
    nonumberlist, % to avoid printing the numbers of the references in the acronyms page
]{glossaries-extra}
\usepackage{graphicx}               %  post-script images
%\usepackage{iwona} % extra fonts, substitute standard ones
\usepackage{listings} % to insert formatted code
\usepackage{lipsum} % for lorem ipsum text, not needed in the real work
\usepackage{makecell} % to change dimensions of cells, for math cases
\usepackage{mathtools} % for additional commands
\usepackage{mfirstuc} % to have capitalization capabilities
\usepackage[final]{microtype}      % microtypography, final lets latex use it also in bibliography
\usepackage{multirow} % to allow for cells covering more than 1 row in tables
\usepackage{nicefrac}       % compact symbols for 1/2, etc.
%\usepackage[lofdepth,lotdepth]{subfig}
\usepackage{ragged2e} % for justifying text
\usepackage{siunitx} % support for SI units of measurement and number typesetting
\usepackage{subfig}
\usepackage{svg} % for svg support, works only if inkscape is installed, default for Overleaf v2
%\usepackage{subfigure}              %  subfigure compatibility, can be removed if subfig
\usepackage{tabularx} % equal-width columns in tables
\usepackage{textcomp} % extra fonts and symbols
\usepackage{url}            % simple URL typesetting
\usepackage{verbatim} % for extended verbatim support
\usepackage{xcolor} % to define colors and use standard CSS names add dvipsnames as option, but it clashes with xcolor loaded in toptesi, pay attention that if it goes in conflict with tikz/beamer, simply use \documentclass[usenames,dvipsnames]{beamer}, along with other custom options when defining the document class


% configuration for glossaries
% convert and load converted glossaries in .tex ,format from .bib
\setabbreviationstyle{long-short-desc} % style before loading resources
% this command sets the style to title for long names of acronyms only in the glossary description, leading to capitalized first-letter for all words
% \glssetcategoryattribute{\glsxtrabbrvtype}{glossname}{capitalisewords} % doesn't work
% resources to load if using a bib file with bib2gls
%\GlsXtrLoadResources[%
% src={glossaries}, % name of the file without extension
% selection=all, % select all the entries
%]
% not needed
%\newglossary*{abbreviation}{Acronyms} % to change the name of this glossary for acronyms

%\renewcommand{\glsclearpage}{\paginavuota} % to allow glossaries to clear pages, done manually is better


% setup for hyperref
\hypersetup{%
    pdfpagemode={UseOutlines},
    bookmarksopen,
    pdfstartview={FitH},
    colorlinks,
    linkcolor={black}, % it is suggested to keep them black, since when printing it it costs per page, and if they have color it's twice the price per page
    citecolor={black},
    urlcolor={black}
  }
%

% setup for svg
\svgsetup{%
    inkscapeformat=pdf, % to force usage of PDF
    inkscapelatex=false, % to disable latex rendering of text, produces errors
}

% setup for siunitx, it does not work in the summary
\sisetup{%
    detect-all, % to use the same font as for writing when using \num
    mode=text, % to allow it to work also in math mode
    group-separator = {,}, % separator for number grouping
    group-minimum-digits = 3, % minimum number of digits a number must have to be grouped in 3-digit groups
}

% listings colours
\definecolor{rulecolor}{rgb}{0,0,0}
\definecolor{commentcolor}{rgb}{0,0.6,0}
\definecolor{linenumbercolor}{rgb}{0.5,0.5,0.5}
\definecolor{keywordcolor}{rgb}{0,0,0.95}
\definecolor{backcolor}{rgb}{1,1,1}%{0.95,0.95,0.92}
\definecolor{stringcolor}{rgb}{0.58,0,0.82}

% setup for lstlisting
\lstset{ %
	backgroundcolor=\color{backcolor},   % choose the background color; you must add \usepackage{color} or \usepackage{xcolor}; should come as last argument
	basicstyle=\footnotesize,        % the size of the fonts that are used for the code
	breakatwhitespace=false,         % sets if automatic breaks should only happen at whitespace
	breaklines=true,                 % sets automatic line breaking
	captionpos=t,                    % sets the caption-position to bottom
	commentstyle=\color{commentcolor},    % comment style
	extendedchars=true,              % lets you use non-ASCII characters; for 8-bits encodings only, does not work with UTF-8
	frame=single,	                   % adds a frame around the code
	keepspaces=true,                 % keeps spaces in text, useful for keeping indentation of code (possibly needs columns=flexible)
	keywordstyle=\color{keywordcolor},       % keyword style
	%language=VHDL,                 % the language of the code
	numbers=left,                    % where to put the line-numbers; possible values are (none, left, right)
	numbersep=5pt,                   % how far the line-numbers are from the code
	numberstyle=\tiny\color{linenumbercolor}, % the style that is used for the line-numbers
	rulecolor=\color{rulecolor},         % if not set, the frame-color may be changed on line-breaks within not-black text (e.g. comments (green here))
	showspaces=false,                % show spaces everywhere adding particular underscores; it overrides 'showstringspaces'
	showstringspaces=false,          % underline spaces within strings only
	showtabs=false,                  % show tabs within strings adding particular underscores
	stepnumber=1,                    % the step between two line-numbers. If it's 1, each line will be numbered
	stringstyle=\color{stringcolor},     % string literal style
	tabsize=4,	                   % sets default tabsize to 2 spaces
	title=\lstname,                   % show the filename of files included with \lstinputlisting; also try caption instead of title
	inputencoding=utf8,
	literate=
	{á}{{\'a}}1 {é}{{\'e}}1 {í}{{\'i}}1 {ó}{{\'o}}1 {ú}{{\'u}}1
	{Á}{{\'A}}1 {É}{{\'E}}1 {Í}{{\'I}}1 {Ó}{{\'O}}1 {Ú}{{\'U}}1
	{à}{{\`a}}1 {è}{{\`e}}1 {ì}{{\`i}}1 {ò}{{\`o}}1 {ù}{{\`u}}1
	{À}{{\`A}}1 {È}{{\'E}}1 {Ì}{{\`I}}1 {Ò}{{\`O}}1 {Ù}{{\`U}}1
	{ä}{{\"a}}1 {ë}{{\"e}}1 {ï}{{\"i}}1 {ö}{{\"o}}1 {ü}{{\"u}}1
	{Ä}{{\"A}}1 {Ë}{{\"E}}1 {Ï}{{\"I}}1 {Ö}{{\"O}}1 {Ü}{{\"U}}1
	{â}{{\^a}}1 {ê}{{\^e}}1 {î}{{\^i}}1 {ô}{{\^o}}1 {û}{{\^u}}1
	{Â}{{\^A}}1 {Ê}{{\^E}}1 {Î}{{\^I}}1 {Ô}{{\^O}}1 {Û}{{\^U}}1
	{œ}{{\oe}}1 {Œ}{{\OE}}1 {æ}{{\ae}}1 {Æ}{{\AE}}1 {ß}{{\ss}}1
	{ű}{{\H{u}}}1 {Ű}{{\H{U}}}1 {ő}{{\H{o}}}1 {Ő}{{\H{O}}}1
	{ç}{{\c c}}1 {Ç}{{\c C}}1 {ø}{{\o}}1 {å}{{\r a}}1 {Å}{{\r A}}1
	{€}{{\euro}}1 {£}{{\pounds}}1 {«}{{\guillemotleft}}1
	{»}{{\guillemotright}}1 {ñ}{{\~n}}1 {Ñ}{{\~N}}1 {¿}{{?`}}1
}
\definecolor{vscode-bkg}{RGB}{254,240,140}
\lstdefinestyle{vscode}{
    backgroundcolor=\color{vscode-bkg!50},
    basicstyle=\ttfamily\color{black},
    keywordstyle=\color{blue},
    commentstyle=\color{green},
    stringstyle=\color{purple},
    numbers=left,
    numberstyle=\tiny\color{gray},
    showstringspaces=false,
    breaklines=true,
    frame=single,
    framesep=5pt,
    rulecolor=\color{black},
    tabsize=4,
    columns=flexible,
    extendedchars=true,
    xleftmargin=2em,
    framexleftmargin=1.5em,
}

% biblatex setup
% generally 9000 is ok, if higher than 10000 it's bad
% If you want to break on URL numbers
\setcounter{biburlnumpenalty}{9000}
% If you want to break on URL lower case letters
\setcounter{biburllcpenalty}{9000}
% If you want to break on URL UPPER CASE letters
\setcounter{biburlucpenalty}{9000}



% how to change Contents to Table of Contents
\addto\captionsenglish{% Replace "english" with the language you use
  \renewcommand{\contentsname}%
    {Table of Contents}%
}

% to change the name of Abbreviations to Acronyms
% not needed if use use entry types and define those
% \renewcommand{\abbreviationsname}{Acronyms}

% to allow line comments in algorithms
\algnewcommand{\LineComment}[1]{\State{} \(\triangleright\) #1}

% to declare abs and norm
\DeclarePairedDelimiter\abs{\lvert}{\rvert}%
\DeclarePairedDelimiter\norm{\lVert}{\rVert}%

% Swap the definition of \abs* and \norm*, so that \abs
% and \norm resizes the size of the brackets, and the 
% starred version does not.
\makeatletter
\let\oldabs\abs{}
\def\abs{\@ifstar{\oldabs}{\oldabs*}}
%
\let\oldnorm\norm{}
\def\norm{\@ifstar{\oldnorm}{\oldnorm*}}
\makeatother


% change this configuration with your info
% if you need fewer or more supervisors you have to change \relatore command by adding or removing lines in the table in toptesi_config
\newcommand{\thesistitle}{Segmenting dynamic points in 3D scenarios}
\newcommand{\thesisuniversitylogo}{images/logo/Logo_PoliTo_new} % choose your logo
\newcommand{\thesiscandidatename}{Francesco}
\newcommand{\thesiscandidatesurname}{Borgna}
\newcommand{\thesissupervisoronetitle}{prof.}
\newcommand{\thesissupervisoronename}{Tatiana}
\newcommand{\thesissupervisoronesurname}{Tommasi}
\newcommand{\thesissupervisortwotitle}{prof.}
\newcommand{\thesissupervisortwoname}{Chiara}
\newcommand{\thesissupervisortwosurname}{Plizzari}
\newcommand{\thesisdate}{March 2024}
\newcommand{\thesiscourse}{Mathematical Engineering}
\newcommand{\thesisuniversity}{Politecnico di Torino}
\newcommand{\thesislevel}{Master} % master or bachelor
\newcommand{\thesiscandidatetext}{Candidate}
\newcommand{\thesissupervisortext}{Supervisors}


% fontsize is {size}{spacing}\family
\newcommand {\institutionfont}{\fontsize {22}{30}\scshape}
\newcommand {\divisionfont}{\fontsize {16}{20}\rmfamily}
\newcommand {\pretitlefont}{\fontsize {16}{16}\rmfamily}
\newcommand {\customtitlefont}{\fontsize {21}{28}\scshape}% {iwona}{bx}{n}}
\newcommand {\fixednamesfont}{\fontsize {14}{20}\mdseries}
\newcommand {\namesfont}{\fontsize {14}{20}\bfseries}
\newcommand {\footfont}{\fontsize {15}{18}\rmfamily}


\addbibresource{bibliography.bib}

% to load the glossaries, not needed if using bib2gls
% for glossary entry
% @entry{bird,
%     name={bird},
%     description = {feathered animal},
%     see={[see also]{duck,goose}}
% }

% if this bib file does not work, try using \input{file.tex}
% where all the \newabbreviation commands have been inserted
% containing all the definitions

% Gls to capitalize first letter
% GLS for full uppercase
% for abbreviations also
% glsxtrshort for abbreviation
% similar for long, full, and capital configurations, add pl at the end for plurals
% glsentryshort, long, plural (referred to shorts) must be used when in section titles
% glslink to allow the link but use a different text (as for href)


% if you want to use also description for the abbreviations/acronyms, you should use bib2gls and define all the entries in a bib file, which is incompatible with Overleaf
\newacronym{SfM}{SfM}{Structure from Motion}
\newacronym{pcd}{pcd}{Pointcloud}

\makeglossaries{}

\begin{document}

\overfullrule=0.00001pt % latex shows a black bar for overfulls over this dimension

%\emergencystretch=1em % to allow some stretching in the lines to avoid overfull boxes, also in bibliography, eventually can be used only before bibliography or in the preamble for the whole document, not needed if using biblatex configuration in most cases


\ateneo{\thesisuniversity} % university name
\logosede[5cm]{\thesisuniversitylogo} % logo, square brackets contain the height

\titolo{\thesistitle} % title
%\sottotitolo{Metodo dei satelliti medicei} % subtitle

% place/remove a slash \\ to put the name on the following line or after Master Degree Course
\corsodilaurea{\thesiscourse} % course name


%~251197 % id number is not needed

\candidato{\thesiscandidatename~\textsc{\thesiscandidatesurname}} % candidate

% using tabular we can have more than 1 supervisor under the same column
\relatore{\tabular{@{}l}%
    \xmakefirstuc{\thesissupervisoronetitle}~\thesissupervisoronename~\textsc{\thesissupervisoronesurname}\\[0.4ex]
    \xmakefirstuc{\thesissupervisortwotitle}~\thesissupervisortwoname~\textsc{\thesissupervisortwosurname}\\[0.4ex]
    \xmakefirstuc{\thesissupervisorthreetitle}~\thesissupervisorthreename~\textsc{\thesissupervisorthreesurname}
    \endtabular}
%\terzorelatore{Ciao}

% in this way we have Academic Year without stile=classica, so without lines
%\sedutadilaurea{\textsc{Academic~Year} 2019-2020}% per la laurea magistrale
% for PoliTo there is only month year
\sedutadilaurea{\thesisdate}% per la laurea magistrale
% PhD
%\esamedidottorato{Novembre 1610}
%\ciclodidottorato{XV}

% offset for binding, the smaller the better
%\setbindingcorrection{3mm}


\english% or \italian (default)

\iflanguage{english}{%
	%\retrofrontespizio{This work is subject to the Creative Commons Licence}

	\CorsoDiLaureaIn{\thesislevel's Degree Course in\space}

	\TesiDiLaurea{\thesislevel's Degree Thesis}

	\InName{in}
	\CandidateName{\xmakefirstuc{\thesiscandidatetext}}% or Candidates
	\AdvisorName{\xmakefirstuc{\thesissupervisortext}}% or Supervisor
	%\TutorName{Tutor}
	%\NomeTutoreAziendale{Internship Tutor}

	%\NomePrimoTomo{First volume}
	%\NomeSecondoTomo{Second Volume}
	%\NomeTerzoTomo{Third Volume}
	%\NomeQuartoTomo{Fourth Volume}
}{}


% front page
% frontespizio can be used for the first page print
% while the custom-made frontpage can be used as hard-cover
% use pdfjoin or pdfseparate to extract or put together the pages if needed
%\frontespizio* % without star the logo is on top
\newgeometry{top=4cm,left=3cm,right=3cm,bottom=4cm,heightrounded}
\begin{titlepage}
\centering
%
{\institutionfont{} \textbf{\MakeUppercase{\thesisuniversity}} \par}
%
\vspace{\stretch{2}} % changing this number and the others changes the proportion
%
{\divisionfont{} \textbf{\thesislevel's Degree in \thesiscourse} \par}
%
\vspace{\stretch{3}}
%
\includegraphics[width=50mm]{\thesisuniversitylogo}\\
%
\vspace{\stretch{4}}
%
{\divisionfont{} \textbf{\thesislevel's Degree Thesis} \par}
%
\vspace{\stretch{3}}
%
{\customtitlefont{} \textbf{\thesistitle} \par}
%
\vspace{\stretch{10}}
%
\makebox[\textwidth]{\null\hfill\def\arraystretch{2}% % to change the spacing change this number
\begin{minipage}[t]{.375\textwidth}\raggedright
    \begin{adjustbox}{width={\textwidth},totalheight={\textheight},keepaspectratio} % with adjustbox it adapts to the lengths of the names, remove it if you want the same font dimension
    \begin{tabular}[t]{@{}l@{}}
        \fixednamesfont{} \textbf{\thesissupervisortext} \\
        \namesfont{} \xmakefirstuc{\thesissupervisoronetitle}~\thesissupervisoronename~\MakeUppercase{\thesissupervisoronesurname}\\
        \namesfont{} \xmakefirstuc{\thesissupervisortwotitle}~\thesissupervisortwoname~\MakeUppercase{\thesissupervisortwosurname}
    \end{tabular}
    \end{adjustbox}
\end{minipage}
%
\hfill
%
\begin{minipage}[t]{.375\textwidth}\raggedleft{}
\begin{adjustbox}{width={\textwidth},totalheight={\textheight},keepaspectratio} % with adjustbox it adapts to the lengths of the names, remove it if you want the same font dimension
\begin{tabular}[t]{@{}l@{}}
    \fixednamesfont{} \textbf{\thesiscandidatetext} \\
    \namesfont{} \thesiscandidatename~\MakeUppercase{\thesiscandidatesurname}
\end{tabular}
\end{adjustbox}
\end{minipage}\hfill\null}\\
%
\vspace{\stretch{5}}
%
{\footfont{} \textbf{\thesisdate} \par}
%
\end{titlepage}

\restoregeometry{}
 % custom frontpage
%\retrofrontespizio
% insert text for the back of the front page
% if you insert any remove the following \paginavuota
% either a blank page or a back is needed to have double-sided printing
% pay attention to leave the space for the page

%\paginavuota % clears a page

\frontmatter

% abstract if needed
% \begin{abstract}
%     % abstract, choose between abstract and summary
With the increasing availability of egocentric wearable devices, 
there has been a surge in first-person videos, leading to numerous studies
aiming to leverage this data. Among these efforts, 3D scene
reconstruction stands out as a key area of interest. This process allows for
the recreation of the scene where the video was captured, providing
 invaluable support for the growing field of augmented reality applications.
Some egocentric datasets include static 3D scans of recording locations,
usually requiring costly hardware or dedicated scans.
An alternative approach involves reconstructing the scene
directly from video frames using Structure from Motion (SfM)
techniques. This method not only captures the motion of the actor
and the objects they interact with, including transformations 
(e.g., slicing a carrot) but also enables the use of any egocentric 
footage for scene reconstruction, even without physical access to the 
environment in real life. However, the task of decomposing dynamic 
scenes into objects has received limited attention. For example, SfM 
finds it challenging to distinguish between moving and static parts, 
resulting in cluttered point cloud reconstructions where the same
object may appear superimposed or in multiple places within the scene.

In this thesis, we combine SfM with egocentric methods to segment moving
objects in 3D. This is achieved by creating a scene with COLMAP,
a SfM algorithm, and then modifying a recent algorithm called 
NeuralDiff, originally designed for producing 2D segmentations of
static objects, foreground, and actors, to extract 3D geometry. 
Additionally, we explored ways to reduce the overall computational 
demands, such as by simplifying the NeuralDiff architecture to better
meet our goals by merging the foreground and actor streams,
and by developing an intelligent video frame sampling technique that
captures the essence of the scene using fewer frames.

% \end{abstract}

% to create blank pages for openright in frontmatter
% use one of the following two methods
% 1) use the following three lines
%\phantom{0} % needed otherwise cleardoublepage does not clean the page because it sees it empty
%\cleardoublepage
%\thispagestyle{empty} % to have empty page, without numbers
% 2) or
\paginavuota{} % to manually create a blank page

\abstract{}
% abstract, choose between abstract and summary
With the increasing availability of egocentric wearable devices, 
there has been a surge in first-person videos, leading to numerous studies
aiming to leverage this data. Among these efforts, 3D scene
reconstruction stands out as a key area of interest. This process allows for
the recreation of the scene where the video was captured, providing
 invaluable support for the growing field of augmented reality applications.
Some egocentric datasets include static 3D scans of recording locations,
usually requiring costly hardware or dedicated scans.
An alternative approach involves reconstructing the scene
directly from video frames using Structure from Motion (SfM)
techniques. This method not only captures the motion of the actor
and the objects they interact with, including transformations 
(e.g., slicing a carrot) but also enables the use of any egocentric 
footage for scene reconstruction, even without physical access to the 
environment in real life. However, the task of decomposing dynamic 
scenes into objects has received limited attention. For example, SfM 
finds it challenging to distinguish between moving and static parts, 
resulting in cluttered point cloud reconstructions where the same
object may appear superimposed or in multiple places within the scene.

In this thesis, we combine SfM with egocentric methods to segment moving
objects in 3D. This is achieved by creating a scene with COLMAP,
a SfM algorithm, and then modifying a recent algorithm called 
NeuralDiff, originally designed for producing 2D segmentations of
static objects, foreground, and actors, to extract 3D geometry. 
Additionally, we explored ways to reduce the overall computational 
demands, such as by simplifying the NeuralDiff architecture to better
meet our goals by merging the foreground and actor streams,
and by developing an intelligent video frame sampling technique that
captures the essence of the scene using fewer frames.


\phantom{0}
\cleardoublepage{}
\thispagestyle{empty}

\ringraziamenti% acknowledgements
% acknowledgements

ACKNOWLEDGMENTS

\vspace*{5\baselineskip}

\begin{flushright}
    \textit{``HI''\\
    Goofy, \href{https://google.com}{Google} by Google}
\end{flushright}


\paginavuota{}
\tableofcontents

\listoftables % ToC for tables

\listoffigures % ToC for figures

% actually abbreviation is the name used for acronym in glossaries-extra
% title sets the name
% type tells the type of glossary to print
% style overrides the global style
% here we are printing only abbreviations
% printunsrtglossary if using record, otherwise printglossary is ok
\paginavuota{}
\printunsrtglossary[style=altlist,title=Acronyms,type=\glsxtrabbrvtype]

% also list of symbols here if needed

% to remove all first use occurrences given the presence of the summary
\glsresetall
% to skip all the first use occurrences, using only short forms
% \glsunsetall


\mainmatter{}

%\part{Prima Parte} % parts division, not needed
% Chapters always open on a right-side page, i.e. odd numbers, so a blank page is inserted if needed
%\cleardoublepage[empty] % to have a fully blank page
% a blank page appears before the first chapter in some configurations, on the last version it doesn't

% list here all the chapters
\part{Related Works}
\label{sec:Related}

\chapter{Datasets}
\textcolor{red}{Motivazioni per cui abbiamo usato questi datasets?
The choice of the following datasets was dictated by the fact that up to our knowledge these are 
currently the largest datasets available in the egocentric scenarios. Let us see these in details.}
\section{EPIC-Kitchens}\label{sec:EK}
EPIC-Kitchens~\cite{EPICKITCHENS} is the largest and most varied dataset in egocentric vision up to our knowledge.
It contains 55 hours of annotated video data recorded by a head-mounted camera of non scripted actions, 
meaning that the actors were not following any \textit{scripted} actions(we will see this in more detail
later).
\subsection{Introduction/motivation}
EPIC-Kitchens was born to fill the gap in the scarcity of annotated video datasets.
As a leading comparison, at the time of writing significant progress have been seen in many domains
such as image classification~\cite{residualImage},object detection~\cite{fasterRCNN},
captioning~\cite{captioning} and visual question answering~\cite{vqa}; due to the advances in deep
learning but mainly due to the availability of large-scale image benchmarks 
such as PASCAL VOC~\cite{pascalImage},ImageNet~\cite{imagenet},Microsoft COCO~\cite{COCO},
ADE20K~\cite{ADE20K}. In the same way the authors thought that by introducing
a large scale video dataset could contribute to the development 
of video domains.

Some video datasets were already available for action classification~\cite{somethingSomething,yt,movieBench,movieQA,vlogs}
but, a part from~\cite{movieQA}, these all contain very short videos, focusing on just
a single action. A solution to this problem was given by Charades~\cite{charades} where 10k
videos have been collected of humans performing daily tasks at home.
The problem with this dataset is that the action recorded were scripted,
meaning that the actor had a text in which he was asked to perform some steps.
In this way the actions lose their naturalness, their inbred evolving 
and multi-tasking properties.

To solve these problems they decided to focus on first-person vision, such that
the recording would not interfere with the actor actions, increasing the 
possibilities of a succesful recording. Also, the viewpoint given by 
first-person vision allows us to record multi-task actions and the many different
ways to perform a variety of important everyday tasks. In Table~\ref{tab:epic_comparison}
we report a summary of the datasets compared by the authors.

\input{content/RelatedWorks/Datasets/epic_table.tex}

\subsection{Data Collection}
To the data collection were involved 32 people in 4 cities in different
countries(in North America and Europe): 15 in Bristol/UK, 8 in Toronto/Canada,
8 in Catania/Italy and 1 in Seattle/USA between May and Nov 2017. Participants
were asked to record each time they visit the kitchen for three consecutive days,
starting filming just before entering the kitchen and stopping before leaving it.
They participated to the process of their own free will without being paid in any way.


Few requests were asked to them. The first was to be in the kitchen alone during
the recording, such that no inter-person interaction could interfere.  The
second one instead was to remove all items that could disclose their identity, for
example portraits or mirrors. In this way they could remain anonymous.

Each participant was equipped with a head-mounted camera with adjustable mounting
such that it could be adapted to the participant's height and possibly different 
environment. They had to check, before each recording, the battery life and the viewpoint,
such that their stretched hand were approximately located at the middle of the 
camera frame.
The camera settings was set for most of videos to linear field of view, using 
59.94fps as frame rate and Full HD resolution of 1920x1080, however some 
subjects made minor changes like wide or ultra-wide FOV or resolution.
In particular 1\% of the videos were recorded at 1280x720 and 0.5\% at 1920x1440.
Also 1\% at 30fps, 1\% at 48fps and 0.2\% at 90fps.

On average, each participant recorded 13.6 sequences, each of those lasted on average 1.7 h
while the maximum duration recorded was of 4.6h. The duuration of the recording was obviously 
linked to the person's kitchen engagement. In Figure~\ref{fig:epic_stat} we can see some statistics
of the data acquired.

\begin{figure}
    \centering
    \includegraphics[width=1\linewidth]{images/relatedWorks/epic_stat.png} % Replace "example-image" with your image file name
    \caption{\textbf{Top} (left to right): time of day of the recording, pie chart of high-level goals,
        histogram of sequence durations and dataset logo; \textbf{Bottom}:Wordles
        of narrations in native languages(English, Italian, Spanish, Greek and Chinese).}\label{fig:epic_stat}
\end{figure}

\subsection{Data Annotation pipeline}
After the end of a sequence each participant was asked to watch the recording and narrate verbally the actions
carried out to a microphone. The sound narration was chosen beacuse it was faster than a written one, and 
participants were thus more willing to provide these annotations.
The guide lines for narrations are reported in Figure~\ref{fig:epic_guidelines}.

\begin{figure}
    \centering
    \includegraphics[width=1\linewidth]{images/relatedWorks/epic_guidelines.png} % Replace "example-image" with your image file name
    \caption{Narration Guidelines given to each participant to be followed after the completion of a recording.}\label{fig:epic_guidelines}
\end{figure}

The most used language was English, but other languages was used, if the participant was not so fluent in English. In particular
a total of 5 languages were used: 17 people narrated in English, 7 in Italian, 6 in Spanish, 1 in Greek and 1 in Chinese.

The motivation to obtain the narrations directly from the actors was due to the fact that they surely knew what they were doing,
avoiding misinterpreting some possible actions. The posthumous narration was instead motivated by the fact that actors could
perform their actions in the most natural way, without being concerned about labelling.

The second step of annotations consists in the transcription of the speech narrations. After testing some automatic audio-to-text 
algorithms, which led to inaccurate transcriptions, they opted for manual transcriptions and translation via Amazon Mechanical Turk(AMT), a crowdsourcing
marketplace that allows to do task that computers  are still unable to complete. More in detail requests to AMT are called HIT(Human Intelligence Tasks).
To ensure consistency, the authors divided speechs in chunks of around 30 seconds by also removing silent parts and sent each chunk 3 times as HIT.
In this way they selected just HIT which had a correspondance. An example of transcription is shown in Figure~\ref{fig:epic_trans}
\begin{figure}
    \centering
    \includegraphics[width=1\linewidth]{images/relatedWorks/epic_trans.png} % Replace "example-image" with your image file name
    \caption{Extracts from 6 transcription files in .sbv format}\label{fig:epic_trans}
\end{figure}
%%%%%%%%%%%%%%%%%%%%%%%%%%%%%%%%%%%%%%%%%%%%%%%%%%%%%%%%%%%%%%%%%%%%%%%%%%%%%%%%%%%%%%%%%%%%%%%%
\textcolor{red}{Qua potrei aggiungere ancora qualcosa a pag 7 del paper in cui 
descrivono come ogni HIT  è composta da 10 consecutive narrated phrases... in più
4 annotators per ogni HIT -> Overlap regions come in Figure \ref{fig:epic_anno} o
magari modificare immagin senza fare veedere $\alpha(\dot)$}
\begin{figure}
    \centering
    \includegraphics[width=1\linewidth]{images/relatedWorks/epic_anno.png} % Replace "example-image" with your image file name
    \caption{Example of annotated action segments for 2 consecutive actions}\label{fig:epic_anno}
\end{figure}
%%%%%%%%%%%%%%%%%%%%%%%%%%%%%%%%%%%%%%%%%%%%%%%%%%%%%%%%%%%%%%%%%%%%%%%%%%%%%%%%%%%%%%%%%%%%%%%%%%%%%%%%%%%%%%%%%%%%%%%%%%%%%%%%%%%%%%%%%%%%%%%%%%%%%%%%%%%%%%%%

In the end they collected 39,596 action narrations, corresponding to a narration
every 4.9s in the video. These narrations gave them a good starting point for labelling
all actions with a rough temporal alignment, obtained from the timestamp of the audio narration
with respect to the video, but still were not perfect. Infact:
\begin{itemize}
    \item The narrations can be incomplete. So only narrated action will be considered in evaluation.
    \item The narration can be belated, after the action takes place.
    \item The narration consists of participants' vocabulary and free language. Similar terms
    have been grouped in minimally overlapping classes.
\end{itemize}
It is worth adding few words about verb and noun annotations. Due to the freedom of 
terms and language, a variety of verbs and nouns have been collected. To reduce the number
of them they grouped these into classes with minimal semantic overlapping. More in detail,
as regards verbs they tried using automatic tools to cluster them but ended up manually
clustering, due to the inefficient results; on the other hand for nouns they semi-automatically
cluster them, preprocessing the compound nouns e.g. "pizza cutter" as a subset  of the second 
noun e.g. "cutter" and also manually adjusting the clustering, merging the variety of 
names used for the same object, e.g. "cup" and "mug". In total they obtained 125 verb classes and
331 noun classes. In Figure~\ref{fig:epic_table} we can see some examples of grouped verbs and
nouns into classes, while in Figure~\ref{fig:epic_freq} the authors show the verb classes
oredered by frequency of occurrence in action segments, as well as the noun classes
ordered by number of annotated bounding boxes.

\begin{figure}
    \centering
    \includegraphics[width=1\linewidth]{images/relatedWorks/epic_cluster.png} % Replace "example-image" with your image file name
    \caption{Sample Verb and Noun Classes}\label{fig:epic_table}
\end{figure}
\begin{figure}
    \centering
    \includegraphics[width=1\linewidth]{images/relatedWorks/Epic_words.png} % Replace "example-image" with your image file name
    \caption{\textbf{From Top:} Frequency of verb classes in action segments;
        Frequency of noun clusters in action segments, by category; Frequency of noun
        clusters in bounding box annotations, by category; Mean and standard deviation of
        bounding box, by category}\label{fig:epic_freq}
\end{figure}

In addition to verb and nouns annotations they also provide active object bounding box annotations.
Similarly to verbs and nouns, they use AMT also for this task. Where each HIT aims to 
get an annotation for one object, for the maximum duration of 25s, which corresponds
to 50 consecutive frames at 2fps. The annotator can also state that the object is inexistent
in at frame \textit{f}. In total they collected 454,255 bounding boxes, some examples
are provided in Figure~\ref{fig:epic_bb}.
\begin{figure}
    \centering
    \includegraphics[width=1\linewidth]{images/relatedWorks/epic_bb.png} 
    \caption{Sample consecutive action segments with keyframe object annotations}\label{fig:epic_bb}
\end{figure}


\subsection{Benchmarks and Baseline Results}
The introduction of a new video datasets implies a variety of potential challenges
that were not available before. Some of these are routine understanding, activity
recognition and object detection. To spur the beginning the authors define 
the previous stated three challenges, providing baseline results.
Let us see the challenges in more detail.

\subsubsection{Action Recognition Challenge} \label{sec:ep_AR_chall}
Provided a trimmed action segment, the challenge requires to recognize
what action class is performed, detecting the pair of verb and noun classes that 
compose the action. To participate to the challenge is asked to test the model
on both splits\footnote{ To test the generalizability to novel environments they 
structured the test set to have a collection of \textit{seen} and \textit{unseen}
kitchens.
\begin{itemize}
    \item \textbf{Seen Kitchens (S1):} in this split each kitchen is seen in both training and testing.
    \item \textbf{Unseen Kitchens (S2):} This divides the participants/kitchens so all sequences
     of tge same kitchen are either in training or testing.
\end{itemize}
} and for each test segment report the econfidence 
scores for each verb and noun class. In Figure~\ref{fig:epic_ar_chall} is reported
a qualitative example of the task.
\begin{figure}
    \centering
    \includegraphics[width=1\linewidth]{images/relatedWorks/AR_chall.png} 
    \caption{Sample qualitative results from the challenge's baseline of the Action Recognition Task}\label{fig:epic_ar_chall}
\end{figure}


\subsubsection{Action Anticipation Challenge}\label{sec:ep_AA_chall}
Provided an anticipation time, which is 1s before the action starts,
the challenge consists in classifying the future action into its action class
composed of the pair of verb and noun classes. To participate to the challenge is asked to test the model
on both splits and for each test segment report the econfidence 
scores for each verb and noun class. In Figure~\ref{fig:epic_aa_chall} is reported
a qualitative example of the task.
\begin{figure}
    \centering
    \includegraphics[width=1\linewidth]{images/relatedWorks/AA_ch.png} 
    \caption{Sample qualitative results from the challenge's baseline of the Action Anticipation Task}\label{fig:epic_aa_chall}
\end{figure}

 \subsubsection{Object Detection Challenge}
In this challenge is required to perform object detectino and localisation.
It must be noted that the annotations captured only \textit{active} objects, namely
objects involved in the action. To partecipate is required to provide predicted
bounding boxes and their confidence scores on both dataset splits. In Figure~\ref{fig:epic_od_chall}
a qualitative example is reported.

\begin{figure}
    \centering
    \includegraphics[width=1\linewidth]{images/relatedWorks/obj_ch.png} 
    \caption{Sample qualitative results from the challenge's baseline of the Object Detection Task}\label{fig:epic_od_chall}
\end{figure}

\subsection{Dataset Release}
\begin{itemize}
    \item Dataset sequences, extracted frames and optical flow are available at:
    
    \url{http://dx.doi.org/10.5523/bris.3h91syskeag572hl6tvuovwv4d}
    \item Annotations, challenge leader-board results and updates and news are available
    at \url{http://epic-kitchens.github.io}
\end{itemize}

\section{EPIC-Kitchens 100}\label{sec:EK100}
In 2021 with~\cite{EK100} a new pipeline is introduced to extend EPIC-Kitchens dataset.
EPIC-KITCHENS-100 collects 100 hours, 20M frames, 90k actions in 700 variable-length
videos, capturing longterm unscripted actions in 45 different environments using headmounted
cameras. Due to its novel annotation pipeline, which will be described more in detail later,
more complete annotations of fine-grained actions are available, allowing the creation of new challenges
such as: action detection\footnote{Action detection involves both recognizing the action and localizing the temporal intervals and spatial regions where the actions occur in a video.}
,cross-modal retrieval(e.g.Audio-Based Interaction Recognition) and domain adaptation\footnote{Training on a domain, e.g a specific kitchen, and test on another domain, e.g. a kitchen of a different person.}.
\subsection{Motivation}
The introduction of EPIC-KITCHENS has transformed egocentric vision, showcasing the unique potential of
first-person views for action recognition and in particular hand-object interactions. To continue
on this previously marked path they decided to enlarge EPIC-KITCHENS, mantaining the \textit{unscripted}
and \textit{unedited} object interactions nature. In fact, the unscripted characteristics make 
the dataset results in a unbalance of data, with novel compositions of actions in new environments,
making it a challenging dataset for domain adaptation.

The most important novelty is the new annotation pipeline which allows to obtain denser and more complete actions'
annotations in the recorded videos, enabling different task on the same dataset.

\subsection{Data Collection}
The additional videos were obtained by half of the previous participants, 16 persons, half of the 
32 previously involved, and 5 new additional subjects. In the end the total participants reached 37
and the different kitchens were 45. 

The new request for the subjects was to reecord 2-4 days of their kitchen routine.

\subsection{Annotation}
An overview of the pipeline taken from the paper is reported in Figure~\ref{fig:e100_ann}.
\begin{figure}
    \centering
    \includegraphics[width=1\linewidth]{images/relatedWorks/AnnotationPipeline.png} 
    \caption{Annotation pipeline: \textbf{a} narrator, \textbf{b} transcriber
    \textbf{c} temporal segment annotator and \textbf{d} dependency parser.
    Red arrows show AMT crowdsourcing of annotations.}\label{fig:e100_ann}
\end{figure}
\subsubsection{Narrator}
The non-stop audio narration has been replaced with a \textit{pause-and-talk} approach.
By pausing the narrator can propose an initial temporal "pointing" but mostly avoids
to miss or misspoke some actions due to lack of time. He does not have to narrate
past actions while watching future actions, so short and overlapping actions are easier
to be annotated.

For this an interface was built for the participants, it can be seen in Figure~\ref{fig:e100_ann}(a).
An important new feature is the possibility to re-record and to delete a narration.

\subsubsection{Transcriber}
Each narration is first transcribed and then translated in English by a hired translator for correctness
and consistency. The transcription process have been facilitated by providing a new transcriber
interface showing three images sampled around the time stamp. As a matter of fact, in the 
old EPIC-Kitchens transcriptor struggled to understand some of the narration wwithout
any video context.

Each narration was analyzed by 3 AMT workers using a consensus of 2 or more workers.
A transcription was rejected if its Word2Vec~\ref{Word2Vec} embeddings was lower than a threshold of 0.9.
In case of consensus failure, the transcription was selected manually.

\subsubsection{Parser}
They used spaCy (\url{https://spacy.io}) to parse the transcriptions into verbs and nouns.
Then they manually grouped those into minimally overlapping classes.
\subsubsection{Temporal Annotator}
They built a AMT interface for the start/end times of action segments(see Figure~\ref{fig:e100_ann}.d).
To improve the quality this time the number of workers were increased from 4 to 5.

\subsection{Quality Improvements}
The attentions cared during the annotation process led to denser and more accurate annotations.
We can see the results by comparing the same action and their respective annotation from the
two different pipelines in Figure~\ref{fig:ep100_comp}.

\begin{figure}
    \centering
    \includegraphics[width=1\linewidth]{images/relatedWorks/Epic_comp.png} 
    \caption{Comparing non-stop narrations (blue) to 'pause-and-talk' narrations (red).
    Right: timestsamps (dots) and segments (bars) for two sample sequences. "pause-and-talk"
    captures all actions including short ones. Black frames depict missed actions.}\label{fig:ep100_comp}
\end{figure}

\subsection{Challenges and Baselines}
\textcolor{red}{With respect to EPIC-Kitchens, 4 new challenges have 
been added.Magari non lo metto? }

\section{EPIC-Fields:\textcolor{red}{Lo metto?Anche se non l'abbiamo usato?}}
The necessity of suitable datasets and benchmarks in the unified problem of 3D geometry and video understanding, which has been
pushed by Neural Rendering (See Section~\ref{sec:Neural}), led to the rise of EPIC Fields. EPIC Fields is a expanded version of 
EPIC-KITCHENS comprehending 3D camera information. 96\% of EPIC-KITCHENS videos were reconstructed, registering 19M frames in 
99 hours recorded in 45 kitchens.

EPIC-KITCHENS is suited for studying the unified problem of geometric reconstruction and semantic understanding. As a matter
of fact egocentric videos  are relevant to mixed and augmented reality applications which are spreading in the last years, and 
the videos probe dynamic neural reconstruction due to their length (up to one hour) and to their dynamic nature.

Anyway obtaining camera information from EPIC-KITCHENS is difficult due to the complexity of its videos. Removing this step
the authors try to ease the research in marrying 3D geometry to video understanding.

In conclusion they made two contributions:
\begin{itemize}
    \item Intelligent Subsampling of frames for SfM algorithms 
    \item Introduce a set of benchmark tasks: \begin{itemize}
        \item dynamic novel view synthesis: reconstruct the same scene from a different point of view.
        \item identifying independently from the camera moving objects
        \item segmenting independently from the camera moving objects
        \item video object segmentation.
    \end{itemize}
\end{itemize}

\subsection{Data}
Some of past egocentric datasets~\cite{visor35,visor5} contain static 3D scans of the environment, separately reconstructed from 
the actions. This additional step is an additional expense both in time and money, since the reconstructions are done with some dedicated costly hardware.
In this work they provide a pipeline to extract the geometric reconstruction of the scene by just processing the egocentric video.
EPIC Fields extends EPIC-KITCHENS(See Section~\ref{sec:EK100}) to include camera pose information. For each frame camera extrinsics and intrinsics parameterrs are provided, 
which enable tasks like 3D reconstruction. In total the succesfully processed 671 videos resulting in 18,790,333 registered video frames with estimated camera poses.

\subsubsection{Motivation.} 
The 3D reconstruction could help recognizing different actions. Some actions could be located in the same 3D spot, e.g washing the dishes at the sink.
Also the construction of tihs dataset could enable studying the relevance of 3D egocentric trajectories to actions(for anticipation),objects(for understanding object state changes)
and hand-object understanding.

\subsubsection{Collection}
Since EPIC-KITCHENS did not collect videos with 3D reconstruction in mind, its videos are difficult to reconstruct. In fact Structure from Motion algorithms take as assumption
that the recorded scene is \textit{static}, meaning that each object will always have the same posiiton in 3D. However kitchen's activities involve the movement of objects like 
ingredients or utensils, and above all the presence of the operating hands.

Some other difficulties are introduced by:
\begin{itemize}
    \item the \textbf{length of videos}, which on average last 9 mins
    \item the \textbf{skewed distribution of viewpoints}: the time spent in different part of the scene is different. In particular we have alternating phases of small motion
    around hot-spots,e.g. washing dishes, and of fast motions, like taking something to finish some task.
\end{itemize} 
The solutions to these problems were given by:
\begin{itemize}
    \item\textbf{ Intelligent subsampling} of video frames.
    \item Using \textbf{SfM} for reconstructing the filtered frames.
    \item \textbf{Registering remaining frames} to the reconstruction.
\end{itemize}

\subsubsection{Filtering}
The aim of this step is to reduce the number of frames while keeping enough overlapping viewpoints for accurate reconstruction while diminishing the viewpoint skew.
Overlap is measured \textcolor{red}{by estimating homographies  on matched SIFT features. Given a homography H, we define visual overlap r as the fraction of image 
area covered by the quadrilateral formed by warping the image corners by H. Windows are formed greedily, finding runs of frames 
(i+1,...,i+k) with overlap $r\geq0.9$ to the first frame i. Filtering discards about 81.8 \% of frames}

\subsubsection{Sparse reconstruction}
Once filtered, frames are fed to COLMAP(Its functioning is reported in Section~\ref{sec:col}).
\subsubsection{Dense reconstruction, automated verification, and restart}
The remaining frames are fed to COLMAP with initial reconstruction. The final reconstruciton is accepted if over 70\% of frames are registered succesfully.
In the end 631 videos were obtained.

In case of failure the threshold r is increase,e.g. r$\geq$0.95. This usually results in doubling the frames, but increasing success rate to 96\%.


\subsection{Benchmarks, Experiments and Results}
The authors defined three new benchmarks  to explore the combination of 3D and video understanding.

\subsubsection{New-View Synthesis(NVS)}
Given a reconstruction based on a subsample of frames, the goal is to predict new video frames based on their timestamps and camera parameters.
The quality of the reconstruction is evaluated as proposed in~\cite{nerf},measuring Peak Signal-to-Noise Ratio (PSNR) of the reconstructed frames
compared to the real ones, making the lack of a 3D ground-truth irrelevant.

\textbf{Video and Frame Selection.} Due to the computational expensive cost of Neural Reconstruction, they provided a benchmark of limited selection of 
videos, namely 50 for a duration of 14.7 hours and 2.86M registered frames.

Frame selection instead is needed to divided the data in train and evaluation splits. The evaluation frames were divided into three tiers of difficulty:
\begin{itemize}
    \item \textbf{In-Action (Hard):} frames belonging to an annotated action segment. During an action it is likely that object in the scene are moved making it difficult
        to reconstruct.
    \item  \textbf{Out-of-Action:} frames NOT belonging to an annotated action segment. These frames can be further divided in:
                \begin{itemize}
                    \item \textbf{Easy}: Frames for which exists a neighbouring frame in the training set. The temporal proximity should ease the process of reconstruction.
                    \item \textbf{Medium}: Frames for which do not exists a neighbouring frame in the training set. 
                \end{itemize}
\end{itemize}

\textbf{Benchmark methods.} Three different neural rendering techniques were used to illustrate their possibilities and limits in such challenging scenarios like
EPIC Fields. The methods used were:
\begin{itemize}
    \item \textbf{NeuralDiff~\cite{neuraldiff}:} consists of three different NeRFs, each one tailored to a part of the scene: static background, moving foreground and the actor. See 
        Section~\ref{sec:Neural} for more details.
    \item \textbf{NeRF-W~\cite{nerfw}:} extend NeRF abilities by learning a low latent space that can modulate scene appearance and geometry. As a results it can separates static and transient
            components.
    \item \textbf{T-NeRF+~\cite{Tnerf}:} time conditioned NeRF at which was added another NeRF to model the static background.
\end{itemize}

In Table~\ref{tab:NVS_comp} I report the authors' resulting experiments, while in Figure~\ref{NVS} we can see the comparison of the output of the three methods. 
\begin{table}
\centering
\begin{tabular}{lccccc}
    \hline \multirow{2}{*}{ \textbf{Method} } & \multirow{2}{*}{ \textbf{Easy} } & \multirow{2}{*}{ \textbf{Medium} } & \multicolumn{3}{c}{ \textbf{Hard} } \\
    \cline { 4 - 6 } & & & All & BG & FG \\
    \hline \textbf{NeRF-W}~\cite{nerfw} & 21.13 & 19.3 & 17.93 & 18.99 & 13.54 \\
    \textbf{T-NeRF+} ~\cite{Tnerf} & 21.58 & 19.81 & 18.44 & 19.73 & 13.74 \\
    \textbf{NeuralDiff}~\cite{neuraldiff} & 22.14 & 19.88 & 18.36 & 19.54 & 13.37 \\
    \hline
\end{tabular}
\caption{\textbf{Dynamic New View Synthesis.}Comparison of different neural rendering methods on varying difficult frames. The values
reported corresponds to PSNR considering all pixels in each test frame.} \label{tab:NVS_comp}
\end{table}


\begin{figure}
    \centering
    \includegraphics[width=1\linewidth]{images/relatedWorks/NVS.png} 
    \caption{\textbf{Dynamic New View Synthesis}. I report an example of the output for the three different methods used: NeRF-W, T-NeRF+ and 
    NeuralDiff. We can see how the initial labelling of difficulty for frames was actually accurate as the reconstructions
    struggle with Hard frames.}\label{fig:NVS}
\end{figure}


\subsubsection{Unsupervised Dynamic Object Segmentation(UDOS)}
In Unsupervised Dynamic Object Segmentation(UDOS) the objective is to find those regions in each frames that correspond to dynamic objects. The lack
of a 3D ground-truth make 2D segmentation accuracy the only way to assess the model. In particular mean average precision (mAP) was used, as proposed in~\cite{neuraldiff}.

\textcolor{red}{DEVO CONTROLLARE CHE EFFETTIVAMENTE VISOR NON ABBIA QUELLO CHE CI SERVE. forse qui in epic fields li hanno messi a posto in qualche modo...    
\textbf{Video and Frame selection.} The used videos were the same of NVS, with the difference that only In-Action frames where considered,
 with VISOR annotations as ground-truth. Actually VISOR annotations were processed in the following way: the original masks were converted into 
 foreground-background masks in three different ways, depending on the type of objects present.
 \begin{itemize}
    \item \textbf{Dynamic objects only} setting:a dynamic object is an object that is currently being moved by visible hands.
    \item \textbf{Dynamic and semi-static objects} setting: objects that moved, not necessarily in the current frame, are semi-static objects.
    \item \textbf{Dynamic and semi-static excluding body parts} setting: active hands are excluded, as some methods overfit to predicting hands solely
    as dynamic objects, ignoring other moving objects.
 \end{itemize}
 As Baseline methods the authors used 4 methods: three based on 3D neural rendering techniques(NeRF-W,T-NeRF,NeuralDiff) and one based on 2D optical flow(Motion
 }
 Grouping(MG)~\cite{MG}. The results are shown in Figure~\ref{fig:UDOS} and Table~\ref{tab:UDOS}.It is worth noting how 3D methods are better discovering semi-static objects.
 However none of the 3D methods explicitly consider motion. and this can be seen as MG performs better on purely dynamic motion, due to the input being the optical flow.
\begin{figure}
    \centering
    \includegraphics[width=1\linewidth]{images/relatedWorks/UDOS.png} 
    \caption{\textbf{UDOS.} Here is reported the comparison of the different methods' output.
    The 2D based perform very good on dynamic objects, while 3D methods struggle a bit but can detect even semi-static objects.}\label{fig:UDOS}
\end{figure}
\begin{table}
    \begin{tabular}{lcccc}
        \hline \textbf{Method} & \textbf{3D} & \textbf{SS+Dyn} & \textbf{SS+Dyn (w/o body)} &\textbf{ Dynamic} \\
        \hline \textbf{MG} ~\cite{MG} & - & 60.19 & 21.65 & 69.26 \\
        \textbf{NeRF-W} ~\cite{nerfw} & $\checkmark$ & 37.26 & 22.96 & 27.41 \\
        \textbf{T-NeRF+}~\cite{Tnerf} & $\checkmark$ & 54.23 & 31.23 & 42.68 \\
        & & & & \\
        \textbf{NeuralDiff}~\cite{neuraldiff}& $\checkmark$ & 62.30 & 31.11 & 55.10 \\
        \hline
        \end{tabular}
        \caption{\textbf{UDOS.} Here I reported the author results for UDOS. The values visible corresponds to the mAP on segmenting semi-static(SS) 
        and dynamic(Dyn) components of the scene.}\label{tab:UDOS}
\end{table}

\subsubsection{Semi-Supervised Video Object Segmentation(VOS)}
Semi-Supervised Video Object Segmentation is a standard video understanding task which consists in propagating some given masks for one or more
objects in a reference frame to the subsequent ones. Usually this task is performed by 2D models but here the authors show how integrating the third dimension
could be beneficial. The idea is to project the 2D mask in 3D, fixing its position in the 3D scene and reproject it depending on the new camera position.

Two baselines were provided, a 2D and a 3D one. In Figure~\ref{fig:VOS} we can see the results and how the intuition previously described led to a good improvement.
\begin{figure}
    \centering
    \includegraphics[width=1\linewidth]{images/relatedWorks/VOS.png} 
    \caption{\textbf{VOS.} Here is reported the comparison of the two different methods.
    The 3D method is clearly beter having as output something really close to the groundtruth. The 2D method instead is performing poorly.}\label{fig:VOS}
\end{figure}

\section{VISOR \textcolor{red}{Lo metto?non usato}}
Non va bene perchè gli oggeti che vengono segmentati sono solo quelli rilevanti all azione descritta, quindi se sposto un bicchiere mentre sto pelando 
le carote non viene segmentato il bicchiere -> non va bene per noi. Oppure oggetti attivi possono essere il lavandino, il tosta pane che sono
però statici.
\section{((Ego4D))}
\textcolor{red}{Lo metto? Non lo ho usato per niente...}?

\chapter{Neural Rendering}\label{sec:Neural}
From CVPR 2020 tutorial~\cite{CVPRtutorial} Neural Rendering is defined as \textit{'
 a new class of deep image and video generation approaches that 
 enable explicit or implicit control of scene properties such as 
 illumination, camera parameters, pose, geometry, appearance, and 
 semantic structure. It combines generative machine learning techniques 
 with physical knowledge from computer graphics to obtain controllable 
 and photo-realistic outputs'}.

\section{Related works}
The actual idea of neural rendering to encode objects and scenes in 
the weights of a MLP was already present before NeRF.
 Hower these methods still struggled compared to other strategies based on 
 discrete representations like triangle meshes or voxel grids. 
 In the next paragraph I will give a quick review of the work preeceding NeRF.
 
 \paragraph{Neural 3D shape representations} Implicit continous 3D shapes 
 representation as level sets had been investigated by~\cite{nerf15,nerf32}.
 These methods were limited by the necessity of a 3D ground truth, which is 
 rarely available. Subsequent works reformulated the problem using differentiable
 rendering functions that allowed to use just 2D images as ground truth. 
 One of the main methods was~\cite{nerf42}, which uses a 
 representation that outputs a feature vector and RGB color at each continous 
 coordinate, proposing a differentiable rendering function consisting of a 
 RNN(Recurrent neural network).
 
 Anyway these methods have been limited to simple shapes with low complexity,
 resulting in oversmoothed renderings.
 
 \paragraph{View synthesis and image-based rendering.} Photorealistic novel views 
 can be reconstructed if a dense sampling of images have been provided by 
 simple light field sample interpolation techniques~\cite{nerf21,nerf5}.
 If sparser images are available, some methods have been introduced that 
 rely on extracting traditional geometry and appearance representations
 from observed images. A popular class of approached uses mesh-based representations
 of scene with diffuse~\cite{nerf48} or view-dependent~\cite{nerf2} appearance. Also there are differentiable
 rasterizers~\cite{nerf4,nerf10} that can directly optimize mesh recontruction to reproduce
 a set of input images using gradient descent. However due to local minima or 
 poor conditioning of the loss landscape, the optimization is usually difficult.
 
 High-quality photorealistic view synthesis is also performed by volumetric representations,
 from a set of given RGB images. Volumetric methods can represent complex shapes and materials
 ,well suited for gradient-based optimization. First works~\cite{nerf19} used observed images to 
 directly predict color voxel grids.
 Later~\cite{nerf9,nerf13,nerf17} used deep networks to color a sampled volumetric region from some given images.
 
 Anyway these methods, being based on discrete sampling of the voxel grid, are restricted to low resolution due to time 
 and computational power. With neural rendering instead a \textit{continous} representation was proposed, reducing 
 the memory requirements and producing higher quality renderings.
 
\paragraph{Background Subtraction}
Background subtraction techniques have been used to detect moving objects 
in videos(see~\cite{ndiff_2}). The simplest way to obtain the background would be to capture
a background image that does not contain any foreground object. Unfortunately in some scenes
it is not possible and it could be changed under critical situations like illumination
changes, objects appeearing and disappearing from the scene. Some expedients have been
introduced that try to predict the background via a background initialization step, which base
its guess on the first few frames of the video. Anyway egocentric videos still contains too much
challenging problems(light change,multiple different objects,moving camera,etc.) that do not allow background subtraction to be a viable solution.

\paragraph{Motion segmentation}
Motion segmentation consists in decomposing a video into individually moving 
objects~\cite{ndiff_18}. Amongst others, it has applications in 
robotics, traffic monitoring, sports analysis, inspection, video surveillance and compression.
Anyway these techniques usually relies on optical flow, which can be subject to some ambiguities
like, in our case,motion parallax. Even occlusion plays an important role. Motion segmentation
struggles for example when an object moves in front of or behind other objects in the 
scene, leading to ambiguity in the flow field. Also, many methods fails when a dynamic 
object temporarilly remain static.

All these problems prevent to apply traditional motion segmentation algorithms to egocentric videos.

\paragraph{Discovering and segmenting objects in videos}
Discovering and segmenting objects in videos is related to background subtraction and 
motion segmentation.
As instance moving objects can be segmented from the background by using a probabilistic model
that acts on optical flow~\cite{ndiff_1}. In~\cite{ndiff_3}, pixel trajectories and
spectral clustering are combined to produce motion segments. Some recent works revisits classical motion segmentation
techniques from a data-driven perspective~\cite{ndiff_38}, \textit{e.g.} using 
physical motion cues to learn 3D representations or learning a scene representation
using neural rendering~\ref{sec:nerf}. 



\section{NeRF:Representing Scenes as Neural Radiance Fields for View Synthesis}\label{sec:nerf}
With NeRF~\cite{nerf} the authors introduced a new state-of-the-art model
for synthesizing novel views of complex scenes by just using a set of sparse
images and their relative positions.

\paragraph*{Introduction}With this work the novel view synthesis problem of complex optimization
 is dealt  
by modeling the scene representation with a fully-connected neural network(MLP) without any
convolutional layer, this is called \textit{neural radiance field}(NeRF). The network
take as input the position of a point $(x,y,z)$ and the direction $(\theta,\phi)$ from which we are looking it
and gives as results the color$(r,g,b)$ and density $(\sigma)$ of that point. 

To render a NeRF from a viewpoint one can:$1)$ march camera rays in the 
scene and sample some points from them $2)$ use those points as input to the neural
network to produce an output set of colors and densities, and $3)$ use volume 
rendering techniques to obtain a final 2D image. All the previous steps
are differentiable such tat we can use gradient descent to optimize the model
by minimizing the error between each image and the corresponding prediction.
Repeating this process from multiple viewpoints encourages the network to 
grasp the 3D scene representation. In Figure~\ref{fig:batteria} is reported an example
,presented in the paper,of the main steps.
\begin{figure}[t]
    \centering
    \includegraphics[width=1\linewidth]{images/relatedWorks/batteria.png} 
    \caption{\textbf{NeRF.}Optimization of a continous 5D neural radiance field
    representation(volume density and view-dependent color at any continuous location)
    of a scene from a seet of input images. The 2D novel views are obtained thanks
    to classic volume rendering techniques. Here in this example, given 100
    images acquired from different viewpoints, they sample two novel views.}\label{fig:batteria}
\end{figure}

Anyway this approch is not enough for a complex scene, in fact the optimization 
does not converge to a sufficiently high-resolution representation. This problem
is solved by trasforming the input coordinates with a positional encoding
that enables the network to represent higher frequency functions.
Summing up, their contribution consists in:
\begin{itemize}
    \item An approach for representing continous scenes as a 5D neural radiance fields.
    \item A differentiable rendering pipeline based on classical volume rendering
    techniques,used to optimize the scene representation via input images.
    \item A positional encoding to map each input 5D coordinate into a higher dimensional
    space which allows to represent high-frequency scene content.
\end{itemize}
\subsection{Method}
A continous 3D scene is represented as a 5D vector-valued function whose input is a 3D coordinate 
$\textbf{x}=(x,y,z)$ and 2D viewing direction $\textbf{d}=(\theta,\phi)$, and whose output is an emitted color 
$\textbf{c} = (r,g,b)$ and volume density $\sigma$. In practice the scene is represented by an MLP network
$F_{\Theta}:(\textbf{x},\textbf{d}) \rightarrow (\textbf{c},\sigma)$, where $\Theta$ are the weights of the network.

The representation is encouraged to be multiview consistent by restricting the network to predict the 
volume density $\sigma$ as a function of just the location \textbf{x}, while the color \textbf{c} is a
function of both the input, location and direction. To do this the MLP $F_{\Theta}$ first processes the 
input \textbf{x} with 8 fully-connected layers (using ReLU activation functions and 256 channels per layer),
and outputs $\sigma$ and a 256-dimensional feature vector. The scheme is summed up in Figure~\ref{fig:mlp}.
\begin{figure}[t]
    \centering
    \includegraphics[width=0.8\linewidth]{images/relatedWorks/Neural/NeRF_mlp.png} 
    \caption{\textbf{$F_{\Theta}$ Scheme.} The input position \textbf{x} pass through 8 Fully connected
    (FC) layers of 256-channels. Each FC layer is followed by a ReLU activation function. This intermediate
    result is then concatenated with the input direction (\textbf{d}) and fed to
    one last FC with 128 channels that feeds its output to a ReLU function. The output of the ReLU
    are the color \textbf{c} and the volume density ($\sigma$).}\label{fig:mlp}
\end{figure}

The effects of the direction in input can be seen in Figure~\ref{fig:lego}.
\begin{figure}[t]
    \centering
    \includegraphics[width=1\linewidth]{images/relatedWorks/lego.png} 
    \caption{Here are reported the results obtained with different strategies, as written underneath each image.
     In particular removing view dependence prevents the model from recreating the specular 
     reflection on the bulldozer tread. Removing the Positional encoding instead we 
     obtain a blurred image, meaning that high frequencies are not captured nor represented.}\label{fig:lego}
\end{figure}

\subsubsection{Volume Rendering with Radiance Fields}
The implicit representation of the scene relies on the volume densities and on the color
of every point in that scene. The color of any ray passing through the scene is rendered using
principles from classical volume rendering~\cite{nerf16}. The volume density $\sigma(\textbf{x})$
can be interpreted as the differential probability of a ray terminating at an infinitesimal
particle at location \textbf{x}. While the expected color C(\textbf{r}) of camera ray 
\textbf{r}(t) = \textbf{o}+ t\textbf{d} with near and far bounds $t_n$ and $t_f$ is:
\begin{equation}
    C(\textbf{r}) = \int_{t_n}^{t_f} T(t)\sigma(\textbf{r}(t))\textbf{c}(\textbf{\textbf{r}(t),\textbf{d}}) \,dt 
\end{equation} 
where $T(t)$ denotes the accumulated transmittance along the ray from $t_n$ to $t$,i.e:
the probability that the ray travels from $t_n$ to $t$ without hitting any other particle.Namely:
\begin{equation}
    T(t) = exp(-\int_{t_n}^{t}\sigma(\textbf{r}(s))\,ds)
\end{equation}
To obtain a new view in our neural radiance field, we should estimate the $C(\textbf{r})$
function for each ray passing through each pixel of the focal plane(see Figure~\ref{fig:rt}).
\begin{figure}[t]
    \centering
    \includegraphics[width=0.5\linewidth]{images/relatedWorks/Neural/ray_tracing.png} 
    \caption{Example of rays passing through an image plane of size 3x3 pixels.}\label{fig:rt}
\end{figure}
This integral is approximated using a numerical method known as \textit{quadrature}.Typically deterministic 
quadrature is used for rendering discretized voxel grids, but in our case it would limit our model's resolution
since the network would only be queried at a fixed discrete set of locations. To solve this problem a
\textit{stratified} sampling approach has been used, where each interval $[t_n,t_f]$ has been partitioned
into N evenly-spaced bins, from which a random sample is then uniformly extracted. Namely:
\begin{equation}
    t_i  \thicksim \mathcal{U} [t_n + \frac{i-1}{N}(t_f-t_n), t_n + \frac{i}{N}(t_f-t_n)]
\end{equation}
In this way, even if we are using a discrete set of samples, using stratified sampling allows us to represent a continuous scene 
representation because the network is evaluated at continuous positions during the training phase.
Following the quadrature rule discussed in~\cite{nerf26} they used the samples to estimate $C(\textbf{r})$:
\begin{equation}\label{eq:neural_C}
    \widehat{C}(\textbf{r}) = \sum_{i=1}^N T_i (1-exp(-\sigma_i \delta_i))\textbf{c}_i, \quad where \quad T_i = exp(-\sum_{j=1}^{i-1}\sigma_j\delta_j)
\end{equation}
where $\delta_i =t_(i+1)$ is the distance between adjacent samples. It can be noticed that the function that approximate $C(\textbf{r})$ is differentiable and can be 
reduced to traditional \textit{alpha compositing}\footnote{\textit{Alpha compositing} is a digital imaging technique used to combine multiple layers of images or graphics with transparency, known as alpha channels.}
with alpha values being $\alpha_i = 1-exp(-\sigma_i \delta_i)$.

\subsubsection{Optimizing a Neural Radiance Field}
In the previous section the core components of a Neural Radiance Field were presented, anyway these parts alone are not able to 
achieve state-of-the-art results. To improve the quality of the representation two improvements were found succesful:
\begin{itemize}
    \item \textbf{Positional Encoding} of the input coordinates, to encourage the representation of high-frequencies.
    \item \textbf{Hierarchical Sampling} procedure that allows to efficiently sample high-frequency representation.
\end{itemize} 

\paragraph{Positional Encoding} After having found that the model $F_\Theta$ performed poorly operating solely on $xyz\theta\phi$ input, in accordance
to the work by Rahaman \textit{et al.}~\cite{nerf35} that states that deep networks are biased towards
learning low-frequency functions; they encode the inputs to a higher dimensional space using high frequency functions. 

The new model $F_\Theta$ can thus be represented as a combinatino of functions $F_\Theta = F_\Theta'\circ \gamma$, where $\gamma$ is a function from $\mathbb{R}$ 
to a higher dimensional space $\mathbb{R}^{2L}$, and  $F_\Theta'$ is still the basic block introduced in the previous sections.
Formally the function used for the encoding part is:
\begin{equation}
    \gamma(p) = (sin(2^0 \pi p), cos(2^0 \pi p), ...,sin(2^L-1 \pi p), cos(2^L-1 \pi p))
\end{equation}
In particular $\gamma(\dot)$ is applied  separately to each component of \textbf{x} and \textbf{d}, after these values are normalized to lie in $[-1,1]$.
Reporting the paper results, they found out that good values of L were: 10 for $\gamma(\textbf{x})$ and 4 for $\gamma(\textbf{d})$.

\paragraph{Hierarchical Sampling} Stratified sampling allows to cover continuous
regions but anyway is still inefficient: free space and occluded regions that do not 
contribute to the rendered image are still sampled repeatedly. To solve this 
problem the authors proposed a \textit{hierarchical} representation which increased
rendering efficiency by distributing samples proportionally to their expected 
effect on the final rendering.

This solution consists in simultaneously optimize two networks: one "coarse" and
one "fine". The first step expect to sample a set of $N_c$ points using stratified 
sampling and feed those to the coarse model. Once this first partial result has 
been obtained a more informed sampling of points is prooduced. The coarse sampling 
infact allows to get a rough idea of which parts of the volume are the most relevant.
To do this they rewrite the alpha composited color from the coarse model $\hat{C}_c(\textbf{r})$
in Eq.~\ref{eq:neural_C} as a weighted sum of all sampled colors $c_i$ along the ray:
\begin{equation}
    \hat{C}_c(\textbf{r}) = \sum_{i=1}^{N_c} w_i c_i, \quad \quad w_i = T_i(1-exp(-\sigma_i \delta_i))
\end{equation}
By normalizing the weights as $\hat{w}_i = \frac{w_i}{\sum_{j=1}^{N_c}w_j}$ we can produce
a piecewise-constant probability density function along the ray as seen in Figure~\ref{fig:pdf_ray}.
\begin{figure}[t]
    \centering
    \includegraphics[width=0.8\linewidth]{images/relatedWorks/Neural/pdf_rays.png} 
    \caption{Probability Density Function of normalized coarse weights $\hat{w}_i$ along a ray with $N_c$ samples.}\label{fig:pdf_ray}
\end{figure}
The next $N_f$ samples are extraced from this distribution and then, combined to the previous $N_c$ samples,
fed to the "fine" network. The final rendered color $\hat{C}_f(\textbf{r})$ is obtained using
Eq.~\ref{eq:neural_C} with $N_c+N_f$. This procedure revealed succesful in obtaining more
samples from regions we expect to contain visible content.

\subsubsection{Implementation Details}
To optimize a scene RGB frames are required with the corresponding camera poses and intrinsic
parameters(for synthetic data these are easily retrievable from the scene model; while for
real images COLMAP was used to extract them).
At each iteration a batch of rays from the set of all pixels of all images of the dataset
is extracted and following the sampling procedure previosuly described the actual color 
is predicted for both the "coarse" and the "fine" model. The loss is computed as the 
total squared error betwwen the rendered and true pixel colors for the two models:
\begin{equation}
    \mathcal{L} = \sum_{r\in\mathcal{R} }[\left\lVert {\hat{C}_c(\textbf{r})-C(\textbf{r})} \right\rVert_2^2+\left\lVert {\hat{C}_f(\textbf{r})-C(\textbf{r})} \right\rVert_2^2]
\end{equation}

where $\mathcal{R}$ is the set of rays of each batch, $C(\textbf{r})$ is the RGB color
ground truth and $\hat{C}_c(\textbf{r})$,$\hat{C}_f(\textbf{r})$ are the predicted color
for the "coarse" and the "fine" model for ray \textbf{r}.

% TODO 
As for some more specific details, they used a batch size of 4086 rays, each
sampled at $N_c=64$ points for the "coarse" model and $N_f=128$ for the "fine"
model. They used the Adam optimizer with a learning rate beginning at $5x10^{-4}$
and decays exponentially to $5x10^-5$ over the course of optimization. Others parameters
of the optimizer where left at default values: $\beta_1 = 0.9,\beta_2=0.999$ and $\epsilon=10^{-7}$.
Each scene took them around 100-300l iterations to converge on NVIDIA V100 GPU(about 1-2 days).
Their tensorflow implementation is provided on \url{https://github.com/bmild/nerf}.

\subsection{Results}
\begin{comment}
To validate their results they generated and aggregated some datasets.
The main distinction is given by synthetic scenes, denoted as\textit{ Diffuse 
Synthetic 360$\degree$}~\cite{deepvoxels} and \textit{Realistic Synthetic 360$\degree$}, and 
realistic scenes,\textit{Real Forward Facing}. More in detail:
\begin{itemize}
    \item \textit{ Diffuse Synthetic 360$\degree$}: contains four Lambertian objects 
    with simple geometry. Each object is rendered at 512x512 pixels from viewpoints taken
    from the upper hemisphere.
    \item \textit{ Realistic Synthetic 360$\degree$}: new dataset that the authors introduce
    containing images of eight objects that exhibit complicated geometry and realistic non-Lambertian\footnote{Lambertian reflectance is the property that defines an ideal "matte" or diffusely reflecting surface. The apparent brightness of a Lambertian surface to an observer is the same regardless of the observer's angle of view~\cite{lambertian}.}
    materials. Six scenes were sampled from the upper hemisphere while the remaining two
    are rendered from a full sphere. For each scene 100 views are captured for training and 
    200 for testing, all at 800x800 pixels.
    \item \textit{Real Forward Facing}: consists of 8 scenes captured with a handheld cellphone
    (5 taken from the LLFF~\cite{LLFF} paper and 3 captured by them), captured with 20 to 62 images,
    holding out $1/8$ of these for the test set. All images have a resolution of 1008x756 pixels.
\end{itemize}

The algorithm which have been compared were the top-performing techniques for view synthesis and 
are the following:
\begin{itemize}
    \item \textbf{Neural Volumes (NV)}~\cite{neuralvol} synthesizes novel views of objects
    lying in a confined space with a distinct background(which must be captured alone).
    The model is based on a 3D convolutional network to predict a discretized RGB$\alpha$ voxel grid.
    \item \textbf{Scene Representation Networks (SRN)}~\cite{srn} represent a continuous scene
    as an opaque surface, implicitly defined by an MLP that maps spatial coordinates to a
    feature vector. The color is then obtained by training a RNN along the ray.
    \item \textbf{Local Light Field Fusion (LLFF)}~\cite{LLFF} is designed for producing photorealistic
    novel views for well-sampled forward facing scenes.
\end{itemize}

%\input{content/RelatedWorks/NeuralRendering/nerf_quantitative_res.tex}
\end{comment}

A qualitative idea of the potentiality of NeRF is reported in Figure~\ref{fig:blender2} and~\ref{fig:blender1}.
\begin{figure}[H]
    \centering
    \includegraphics[width=1\linewidth]{images/relatedWorks/Neural/blender2.png} 
    \caption{Comparison on test images from the newly introduced
    synthetic dataset. The algorithms compared are NeRF,Local Light Field Fusion LLFF~\cite{} and Scene
    Representation Network SRN. NeRF method is able to recover fine details in both geometry
    and appearance. LLFF exhibits some artifacts on the microphone and some ghosting
    artifact in the other scenes. SRN produces distorted and blurry rendering for every
    scene. Neural Volumesstruggle capturing details we can see from
    the ship reeconstruction.}\label{fig:blender2}
\end{figure}

\begin{figure}[H]
    \centering
    \includegraphics[width=1\linewidth]{images/relatedWorks/Neural/blender1.png} 
    \caption{Comparison on the test set of the real images.
    As expected LLFF is performing pretty well being projected
    for this specific use case(forward-facing captures of real scenes).
    Anyway NeRF is able to represent fine geometry more consistently across
    rendered views than LLFF as we can see in Fern's and in T-rex. NeRF
    is also able to reproduce partially occluded scene as in the second row.
    SRN instead completely fail to represent any high-frequency content.}\label{fig:blender1}
\end{figure}
%An additional validation of their design choices is given by an ablation
%study on the various parts that have been discussed in the implementation part.
%In particular the result of the study is reported in Table~\ref{tab:nerf_ablation}.
%\input{content/RelatedWorks/NeuralRendering/nerf_ablation.tex}
%
\input{content/RelatedWorks/NeuralRendering/NeuralDiff.tex}
\section{Photogrammetry}
\textit{Photogrammetry is the science and technology of 
obtaining reliable information about physical objects and
the environment through the process of recording,
measuring and interpreting photographic images and patterns 
of electromagnetic radiant imagery and other 
phenomena}\cite{examplewebsite}.

It comprises all techniques concerned with making measurements of
real-world objects features from images.
Its utility range from the measuring of coordinates, quantification
of distances, heights, areas and volumes, preparation
of topographic maps, to generation of digital elevation 
models and orthophotographs. The functioning rely mostly on optics and projective geometry rules. 

\vspace{12pt}

As first assumption we have the modellization of the camera as a simplified
version of itself: the \textit{Pinhole Camera}. As in the first designed
cameras(\textit{camera obscura}), in the Pinhole Camera world's light is 
captured through a pinhole and then projected into the \textit{focal plane}.

\begin{figure}
    \centering
    \includegraphics[width=0.3\linewidth]{images/relatedWorks/Pinhole.png} % Replace "example-image" with your image file name
    \caption{Pinhole Camera}
    \label{fig:pinhole}
  \end{figure}
  
A Pinhole camera is characterized by two collection of parameters:
\begin{itemize}
    \item  \textbf{Extrinsic} parameters: gives us information on location
                                        and rotation in the world.
    \item  \textbf{Intrinscic} parameters: gives us internal property such as:
                                    focal length, field of view, resolution...
\end{itemize}  
These parameters can be rewritten in their corresponding matrices:
\[
  Intrinsic=K= \begin{bmatrix}
    f_{x} & s & c_{x} \\
    0 & f_{y} & c_{y} \\
    0 & 0     & 1     \\
  \end{bmatrix}
\]
where:
\begin{itemize}
    \item $f_{x},f_{y}$ are are the \textit{focal lengths} of the camera in the x and y directions, 
    they are needed to keep the image aspect ratio.
    \item $c_{x},c_{y}$ are the coordinates of the \textit{principal point}
    (the point where the optical axis intersects the image plane).
\end{itemize}

\[
  Extrinsic= \begin{bmatrix}
    \textbf{R}_{3x3} & \textbf{t}_{3x1}  \\
    0_{1x3} & \textbf{1}_{1x1}  \\
  \end{bmatrix}
\]
where:
\begin{itemize}
    \item $\textbf{R}_{3x3}$ is a rotation matrix
    \item $\textbf{t}_{3x1}$ is a translation vector
\end{itemize}
Extrinsic matrix is also known as the 4x4 transformation matrix 
that converts points from the world coordinate system to the camera
coordinate system.

Exploting homogeneous coordinates we can rewrite the image capturing process as the 
following combination of matrices:
\begin{align*}
    \begin{bmatrix}
        u \\
        v \\
        z
      \end{bmatrix}
    &= \begin{bmatrix}
        f_{x} & s & c_{x} & 0 \\
        0 & f_{y} & c_{y} & 0\\
        0 & 0     & 1     & 0\\
      \end{bmatrix} \begin{bmatrix}
        \textbf{R}_{3x3} & \textbf{t}_{3x1}  \\
        0_{1x3} & \textbf{1}_{1x1}  \\
      \end{bmatrix}
      \begin{bmatrix}
        X_w \\
        Y_w \\
        Z_w \\
        1    
      \end{bmatrix}
  \end{align*}






%\part{Tabelle}\label{sec:Tabelle}
\section{Tabelle}
Sezione fatta solo per renderci conto di cosa abbiamo
\begin{figure}
    \centering
    \includegraphics[width=1\linewidth]{images/tabelle/ColmapP01.png} 
    \caption{Tabella che fa vedere come cambiano i tempi di ricostruzione ma
    anche il numero di frame ricostruiti e quanti punti ottengo alla fine. Questo poi
    viene fatto vedere visualizzando le pointcloud}\label{fig:blender1}
\end{figure}
\begin{table}[h]
    \centering
    \begin{tabular}{|p{2cm}|p{2cm}|p{2cm}|p{2cm}|p{2cm}|p{2cm}|p{2cm}|p{2cm}|}
        \hline
        \multirow{2}{*}{\textbf{Feature Extractor}} & \multirow{2}{*}{\textbf{Exhaustive Matcher}} & \multirow{2}{*}{\textbf{Mapper}} & \multirow{2}{*}{\textbf{Image Undistorter}} & \multirow{2}{*}{\textbf{Patch Match Stereo}} & \multirow{2}{*}{\textbf{Stereo Fusion}} & \multirow{2}{*}{\textbf{Total Time}} \\
        & & & & & & \\
        \hline
        A & & & & & & \\
        \hline
        B & & & & & & \\
        \hline
        C & & & & & & \\
        \hline
        D & & & & & & \\
        \hline
        E & & & & & & \\
        \hline
        F & & & & & & \\
        \hline
        G & & & & & & \\
        \hline
    \end{tabular}
    \caption{Your Table Caption Here}
    \label{tab:mytable}
\end{table}

\begin{table}[h]
\begin{center}
    \begin{tabular}{ | c | c | c |}
      \hline
      \thead{A Head} & \thead{A Second \\ Head} & \thead{A Third \\ Head} \\
      \hline
      Some text &  \makecell{Some really \\ longer text}  & Text text text  \\
      \hline
    \end{tabular}
  \end{center}
\end{table}

Poi abbiamo un po di NeuralDiff solo su P01-01 a 228x128 cambiando abbassando soglia di sampling
\begin{figure}[H]
    \centering
    \includegraphics[width=1\linewidth]{images/tabelle/NeuralDiff228.png} 
    \caption{NeuralDiff solo su P01-01 a 228x128 cambiando abbassando soglia di sampling}\label{fig:blender1}
\end{figure}
\begin{figure}[H]
    \centering
    \includegraphics[width=1\linewidth]{images/tabelle/Ndiff114.png} 
    \caption{NeuralDiff confronto dei 3 sampling a 114x64. Se rifacessi un po di calcoli potrei valutare anche NeuralCleaner(Actor+foreground uniti)}\label{fig:blender1}
\end{figure}
\begin{figure}[H]
    \centering
    \includegraphics[width=1\linewidth]{images/tabelle/NeuralCleaner.png} 
    \caption{Qui avevo fatto NDiff con sampling ma prima del Bug quindi risultati non esatti, però c'era confronto con NeuralCleaner(Actor+foreground uniti) per vedere almeno tempi su stessi split}\label{fig:blender1}
\end{figure}

\begin{figure}[H]
    \centering
    \includegraphics[width=1\linewidth]{images/tabelle/sampling.png} 
    \caption{Questa potrebbe esseere tabella in cui faccio vedere i frame iniziali, e i vari
    sampling che faccio durante la pipeline per arrivare ad avere 1000 frames o 700 per NeuralDiff}\label{fig:blender1}
\end{figure}
Poi altre tabelle che mancano che devo fare:
\begin{itemize}
    \item Tabella in cui faccio i risultati per tutte le Scene(P01-01 P03-04 P04-01 etc...)
    \item Tabella in cui mostro i risultati di metriche che troverò per confrontare le distribuzioni dei sampling
    \item ?
\end{itemize}

%\part{Da Sistemare}\label{sec:Sistemare}

\section{Metrics}\label{sec:Metrics}
\textcolor{red}{CHIEDERE COME MOTIVARE LA SCELTA DELLE NOSTRE METRICHE}
As regards metrics we looked in literature for a way to evaluate
our results but unfortunately each method involved a ground truth
which for our dataset is not available. Possible ways to obtain
a groundtruth could be manual annotations or simulating the environments.
Both these two methods would take a considerable large amount
of time and are also beyond the scope of this thesis.

For this reason we ended up by using the metrics proposed in~\cite{neuraldiff}.
Namely these are:
\begin{itemize}
    \item \textbf{PSNR}
    \item \textbf{AP}
\end{itemize}
\subsection{PSNR:Peak signal-to-noise ratio}
The Peak Signal-to-Noise Ratio (PSNR) is a metric commonly used in image and
video processing to quantify the quality of a reconstructed or processed signal,
like an image or video. It gives a measures of the ratio between the 
maximum possible power of a signal (MAX) and the power of the distortion or noise 
that affects the signal (MSE).

The formula for PSNR is usually expressed in decibels (dB) and is given by:

\[ \text{PSNR} = 20 \cdot \log_{10}\left(\frac{{\text{MAX}}}{{\text{MSE}}}\right) \]

where:
\begin{itemize}
    \item MAX is the maximum possible pixel value of the image (1 in our case).
    \item MSE is the Mean Squared Error, which represents the average squared 
    difference between the original signal and the reconstructed or distorted signal.
\end{itemize}

It is worth noting that a high PSNR does not guarantee that the processed signal 
will be perceived as visually pleasing or high-quality by humans, especially in the case of perceptually sensitive applications like image and video compression.

\subsection{AP:Average Precision}
Average Precision (AP) is a metric commonly used in object detection and information retrieval
to evaluate the performance of machine learning models. It measures the \textit{precision-recall} trade-off of a model.

It can be useful to remind what \textit{Precision} and \textit{Recall} are. Namely:

\begin{equation}
    Precision=\frac{TP}{TP+FP}
\end{equation}
\begin{equation}
    Recall=\frac{TP}{TP+FN}
\end{equation}

where:
\begin{itemize}
    \item TP=True positive
    \item FP=False positive
    \item TN=True negative
\end{itemize}
Average precision is then computed as the area below the precision-recall curve, specifically
the curve obtained by varying the confidence threshold of the inference model as shown
in Figure \ref{fig:auc}. That is why 
it can also be found in literature as AUC(Area Under Curve).
Its scalar value summarize the precision-recall performance of the model.
A higher AP is desirable, indicating a model that effectively retrieves 
relevant instances while minimizing false positives.

\begin{figure}
    \centering
    \includegraphics[width=0.5\linewidth]{images/metrics/auc.png} % Replace "example-image" with your image file name
    \caption{Example of Precision-recall curve.We can see how the bottom line model 
        represents the worst a model can perform, e.g. predict every sample as it is 
        coming from the same class,if the dataset is balanced. A better model would
        \textit{tend} to the upper-right corner, which instead represents the best
        possible model, a model that have maximum precision and recall.}\label{fig:auc}
\end{figure}

\begin{table}[h]
    \centering
    \begin{tabular}{|p{2cm}|p{2cm}|p{2cm}|p{2cm}|p{2cm}|p{2cm}|p{2cm}|p{2cm}|}
        \hline
        \multirow{2}{*}{\textbf{Feature Extractor}} & \multirow{2}{*}{\textbf{Exhaustive Matcher}} & \multirow{2}{*}{\textbf{Mapper}} & \multirow{2}{*}{\textbf{Image Undistorter}} & \multirow{2}{*}{\textbf{Patch Match Stereo}} & \multirow{2}{*}{\textbf{Stereo Fusion}} & \multirow{2}{*}{\textbf{Total Time}} \\
        & & & & & & \\
        \hline
        A & & & & & & \\
        \hline
        B & & & & & & \\
        \hline
        C & & & & & & \\
        \hline
        D & & & & & & \\
        \hline
        E & & & & & & \\
        \hline
        F & & & & & & \\
        \hline
        G & & & & & & \\
        \hline
    \end{tabular}
    \caption{Your Table Caption Here}
    \label{tab:mytable}
\end{table}

\begin{table}[h]
\begin{center}
    \begin{tabular}{ | c | c | c |}
      \hline
      \thead{A Head} & \thead{A Second \\ Head} & \thead{A Third \\ Head} \\
      \hline
      Some text &  \makecell{Some really \\ longer text}  & Text text text  \\
      \hline
    \end{tabular}
  \end{center}
\end{table}


% \paginavuota % it works even without stile=classica

\appendix
% appendix
\chapter{Galileo}
\label{sec:appendix_galileo}

%\lstinputlisting[]{} % for source code files directly
% lstlisting environment for direct inclusion
\begin{lstlisting}[language=Python]
    import os
    os.system("echo 1")
\end{lstlisting}

% for computational complexity
$\mathcal{O}\left(n\log{n}\right)$

% verbatim
\verb+numpy+



% endnotes here if needed

\phantom{0}
\cleardoublepage{}
\printbibliography[heading=bibintoc] % heading required to show it in ToC

\end{document}
