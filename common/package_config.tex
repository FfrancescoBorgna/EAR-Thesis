% configuration for glossaries
% convert and load converted glossaries in .tex ,format from .bib
\setabbreviationstyle{long-short-desc} % style before loading resources
% this command sets the style to title for long names of acronyms only in the glossary description, leading to capitalized first-letter for all words
% \glssetcategoryattribute{\glsxtrabbrvtype}{glossname}{capitalisewords} % doesn't work
% resources to load if using a bib file with bib2gls
%\GlsXtrLoadResources[%
% src={glossaries}, % name of the file without extension
% selection=all, % select all the entries
%]
% not needed
%\newglossary*{abbreviation}{Acronyms} % to change the name of this glossary for acronyms

%\renewcommand{\glsclearpage}{\paginavuota} % to allow glossaries to clear pages, done manually is better


% setup for hyperref
\hypersetup{%
    pdfpagemode={UseOutlines},
    bookmarksopen,
    pdfstartview={FitH},
    colorlinks,
    linkcolor={black}, % it is suggested to keep them black, since when printing it it costs per page, and if they have color it's twice the price per page
    citecolor={black},
    urlcolor={black}
  }
%

% setup for svg
\svgsetup{%
    inkscapeformat=pdf, % to force usage of PDF
    inkscapelatex=false, % to disable latex rendering of text, produces errors
}

% setup for siunitx, it does not work in the summary
\sisetup{%
    detect-all, % to use the same font as for writing when using \num
    mode=text, % to allow it to work also in math mode
    group-separator = {,}, % separator for number grouping
    group-minimum-digits = 3, % minimum number of digits a number must have to be grouped in 3-digit groups
}

% listings colours
\definecolor{rulecolor}{rgb}{0,0,0}
\definecolor{commentcolor}{rgb}{0,0.6,0}
\definecolor{linenumbercolor}{rgb}{0.5,0.5,0.5}
\definecolor{keywordcolor}{rgb}{0,0,0.95}
\definecolor{backcolor}{rgb}{1,1,1}%{0.95,0.95,0.92}
\definecolor{stringcolor}{rgb}{0.58,0,0.82}

% setup for lstlisting
\lstset{ %
	backgroundcolor=\color{backcolor},   % choose the background color; you must add \usepackage{color} or \usepackage{xcolor}; should come as last argument
	basicstyle=\footnotesize,        % the size of the fonts that are used for the code
	breakatwhitespace=false,         % sets if automatic breaks should only happen at whitespace
	breaklines=true,                 % sets automatic line breaking
	captionpos=t,                    % sets the caption-position to bottom
	commentstyle=\color{commentcolor},    % comment style
	extendedchars=true,              % lets you use non-ASCII characters; for 8-bits encodings only, does not work with UTF-8
	frame=single,	                   % adds a frame around the code
	keepspaces=true,                 % keeps spaces in text, useful for keeping indentation of code (possibly needs columns=flexible)
	keywordstyle=\color{keywordcolor},       % keyword style
	%language=VHDL,                 % the language of the code
	numbers=left,                    % where to put the line-numbers; possible values are (none, left, right)
	numbersep=5pt,                   % how far the line-numbers are from the code
	numberstyle=\tiny\color{linenumbercolor}, % the style that is used for the line-numbers
	rulecolor=\color{rulecolor},         % if not set, the frame-color may be changed on line-breaks within not-black text (e.g. comments (green here))
	showspaces=false,                % show spaces everywhere adding particular underscores; it overrides 'showstringspaces'
	showstringspaces=false,          % underline spaces within strings only
	showtabs=false,                  % show tabs within strings adding particular underscores
	stepnumber=1,                    % the step between two line-numbers. If it's 1, each line will be numbered
	stringstyle=\color{stringcolor},     % string literal style
	tabsize=4,	                   % sets default tabsize to 2 spaces
	title=\lstname,                   % show the filename of files included with \lstinputlisting; also try caption instead of title
	inputencoding=utf8,
	literate=
	{á}{{\'a}}1 {é}{{\'e}}1 {í}{{\'i}}1 {ó}{{\'o}}1 {ú}{{\'u}}1
	{Á}{{\'A}}1 {É}{{\'E}}1 {Í}{{\'I}}1 {Ó}{{\'O}}1 {Ú}{{\'U}}1
	{à}{{\`a}}1 {è}{{\`e}}1 {ì}{{\`i}}1 {ò}{{\`o}}1 {ù}{{\`u}}1
	{À}{{\`A}}1 {È}{{\'E}}1 {Ì}{{\`I}}1 {Ò}{{\`O}}1 {Ù}{{\`U}}1
	{ä}{{\"a}}1 {ë}{{\"e}}1 {ï}{{\"i}}1 {ö}{{\"o}}1 {ü}{{\"u}}1
	{Ä}{{\"A}}1 {Ë}{{\"E}}1 {Ï}{{\"I}}1 {Ö}{{\"O}}1 {Ü}{{\"U}}1
	{â}{{\^a}}1 {ê}{{\^e}}1 {î}{{\^i}}1 {ô}{{\^o}}1 {û}{{\^u}}1
	{Â}{{\^A}}1 {Ê}{{\^E}}1 {Î}{{\^I}}1 {Ô}{{\^O}}1 {Û}{{\^U}}1
	{œ}{{\oe}}1 {Œ}{{\OE}}1 {æ}{{\ae}}1 {Æ}{{\AE}}1 {ß}{{\ss}}1
	{ű}{{\H{u}}}1 {Ű}{{\H{U}}}1 {ő}{{\H{o}}}1 {Ő}{{\H{O}}}1
	{ç}{{\c c}}1 {Ç}{{\c C}}1 {ø}{{\o}}1 {å}{{\r a}}1 {Å}{{\r A}}1
	{€}{{\euro}}1 {£}{{\pounds}}1 {«}{{\guillemotleft}}1
	{»}{{\guillemotright}}1 {ñ}{{\~n}}1 {Ñ}{{\~N}}1 {¿}{{?`}}1
}


% biblatex setup
% generally 9000 is ok, if higher than 10000 it's bad
% If you want to break on URL numbers
\setcounter{biburlnumpenalty}{9000}
% If you want to break on URL lower case letters
\setcounter{biburllcpenalty}{9000}
% If you want to break on URL UPPER CASE letters
\setcounter{biburlucpenalty}{9000}
