\chapter{Datasets}
\textcolor{red}{Motivazioni per cui abbiamo usato questi datasets?
The choice of the following datasets was dictated by the fact that up to our knowledge these are 
currently the largest datasets available in the egocentric scenarios. Let us see these in details.}
\section{EPIC-Kitchens}
EPIC-Kitchens~\cite{EPICKITCHENS} is the largest and most varied dataset in egocentric vision up to our knowledge.
It contains 55 hours of annotated video data recorded by a head-mounted camera of non scripted actions, 
meaning that the actors were not following any \textit{scripted} actions(we will see this in more detail
later).
\subsection{Introduction/motivation}
EPIC-Kitchens was born to fill the gap in the scarcity of annotated video datasets.
As a leading comparison, at the time of writing significant progress have been seen in many domains
such as image classification~\cite{residualImage},object detection~\cite{fasterRCNN},
captioning~\cite{captioning} and visual question answering~\cite{vqa}; due to the advances in deep
learning but mainly due to the availability of large-scale image benchmarks 
such as PASCAL VOC~\cite{pascalImage},ImageNet~\cite{imagenet},Microsoft COCO~\cite{COCO},
ADE20K~\cite{ADE20K}. In the same way the authors thought that by introducing
a large scale video dataset could contribute to the development 
of video domains.

Some video datasets were already available for action classification~\cite{somethingSomething,yt,movieBench,movieQA,vlogs}
but, a part from~\cite{movieQA}, these all contain very short videos, focusing on just
a single action. A solution to this problem was given by Charades~\cite{charades} where 10k
videos have been collected of humans performing daily tasks at home.
The problem with this dataset is that the action recorded were scripted,
meaning that the actor had a text in which he was asked to perform some steps.
In this way the actions lose their naturalness, their inbred evolving 
and multi-tasking properties.

To solve these problems they decided to focus on first-person vision, such that
the recording would not interfere with the actor actions, increasing the 
possibilities of a succesful recording. Also, the viewpoint given by 
first-person vision allows us to record multi-task actions and the many different
ways to perform a variety of important everyday tasks. In Table \ref{tab:epic_comparison}
we report a summary of the datasets compared by the authors.

\begin{table}
\resizebox{\textwidth}{!}{
\begin{tabular}{|l|c|c|c|c|r|r|r|r|r|r|c|c|}
    \hline Dataset & Ego? & \begin{tabular}{c} 
    Non- \\
    Scripted?
    \end{tabular} & \begin{tabular}{c} 
    Native \\
    Env?
    \end{tabular} & Year & Frames & \begin{tabular}{r} 
    Sequ- \\
    ences
    \end{tabular} & \begin{tabular}{r} 
    Action \\
    Segments
    \end{tabular} & \begin{tabular}{c} 
    Action \\
    Classes
    \end{tabular} & \begin{tabular}{r} 
    Object \\
    BBs
    \end{tabular} & \begin{tabular}{c} 
    Object \\
    Classes
    \end{tabular} & \begin{tabular}{c} 
    Partici- \\
    pants
    \end{tabular} & \begin{tabular}{c} 
    No. \\
    Env.s
    \end{tabular} \\
    \hline EPIC-KITCHENS & $\checkmark$ & $\checkmark$ & $\checkmark$ & 2018 & $11.5 \mathrm{M}$ & 432 & 39,596 & $149^*$ & 454,255 & 323 & 32 & 32 \\
    \hline EGTEA Gaze+ [16] & $\checkmark$ & $\times$ & $\times$ & 2018 & $2.4 \mathrm{M}$ & 86 & 10,325 & 106 & 0 & 0 & 32 & 1 \\
    Charades-ego [41] & $70 \% \checkmark$ & $\times$ & $\checkmark$ & 2018 & $2.3 \mathrm{M}$ & 2,751 & 30,516 & 157 & 0 & 38 & 71 & N/A \\
    BEOID [6] & $\checkmark$ & $\times$ & $\times$ & 2014 & $0.1 \mathrm{M}$ & 58 & 742 & 34 & 0 & 0 & 5 & 1 \\
    GTEA Gaze+ [13] & $\checkmark$ & $\times$ & $\times$ & 2012 & $0.4 \mathrm{M}$ & 35 & 3,371 & 42 & 0 & 0 & 13 & 1 \\
    ADL [36] & $\checkmark$ & $\times$ & $\checkmark$ & 2012 & $1.0 \mathrm{M}$ & 20 & 436 & 32 & 137,780 & 42 & 20 & 20 \\
    CMU [8] & $\checkmark$ & $\times$ & $\times$ & 2009 & $0.2 \mathrm{M}$ & 16 & 516 & 31 & 0 & 0 & 16 & 1 \\
    \hline YouCook2 [56] & $\times$ & $\checkmark$ & $\checkmark$ & 2018 & $@ 30 \mathrm{fps} 15.8 \mathrm{M}$ & 2,000 & 13,829 & 89 & 0 & 0 & $2 \mathrm{~K}$ & N/A \\
    VLOG [14] & $\times$ & $\checkmark$ & $\checkmark$ & 2017 & $37.2 \mathrm{M}$ & $114 \mathrm{~K}$ & 0 & 0 & 0 & 0 & $10.7 \mathrm{~K}$ & N/A \\
    Charades [42] & $\times$ & $\times$ & $\checkmark$ & 2016 & $7.4 \mathrm{M}$ & 9,848 & 67,000 & 157 & 0 & 0 & N/A & 267 \\
    Breakfast [28] & $\times$ & $\checkmark$ & $\checkmark$ & 2014 & $3.0 \mathrm{M}$ & 433 & 3078 & 50 & 0 & 0 & 52 & 18 \\
    50 Salads [44] & $\times$ & $\times$ & $\times$ & 2013 & $0.6 \mathrm{M}$ & 50 & 2967 & 52 & 0 & 0 & 25 & 1 \\
    MPII Cooking 2 [39] & $\times$ & $\times$ & $\times$ & 2012 & $2.9 \mathrm{M}$ & 273 & 14,105 & 88 & 0 & 0 & 30 & 1 \\
    \hline
    \end{tabular}}
    \caption{Comparative overview of relevant datasets(action classes with > 50 samples)}
    \label{tab:epic_comparison}
\end{table}


\subsection{Data Collection}
To the data collection were involved 32 people in 4 cities in different
countries(in North America and Europe): 15 in Bristol/UK, 8 in Toronto/Canada,
8 in Catania/Italy and 1 in Seattle/USA between May and Nov 2017. Participants
were asked to record each time they visit the kitchen for three consecutive days,
starting filming just before entering the kitchen and stopping before leaving it.
They participated to the process of their own free will without being paid in any way.


Few requests were asked to them. The first was to be in the kitchen alone during
the recording, such that no inter-person interaction could interfere.  The
second one instead was to remove all items that could disclose their identity, for
example portraits or mirrors. In this way they could remain anonymous.

Each participant was equipped with a head-mounted camera with adjustable mounting
such that it could be adapted to the participant's height and possibly different 
environment. They had to check, before each recording, the battery life and the viewpoint,
such that their stretched hand were approximately located at the middle of the 
camera frame.
The camera settings was set for most of videos to linear field of view, using 
59.94fps as frame rate and Full HD resolution of 1920x1080, however some 
subjects made minor changes like wide or ultra-wide FOV or resolution.
In particular 1\% of the videos were recorded at 1280x720 and 0.5\% at 1920x1440.
Also 1\% at 30fps, 1\% at 48fps and 0.2\% at 90fps.
\subsection{Data Annotation pipeline}

\section{(EPIC-Fields)}
?
\section{((Ego4D))}
?
