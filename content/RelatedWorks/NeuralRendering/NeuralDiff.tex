\section{NeuralDiff: Segmenting 3D objects that move in egocentric videos}
NeuralDiff is a neural radiance field adapted to distinguish three different parts of
egocentric videos:$1)$ \textit{static}, that is everything that does not move,$2)$
\textit{foreground}, which is everything that at some point of the video moves, 
and $3)$ \textit{actor},which comprehends the body parts of the person who 
is wearing the camera. 

Anyway this is not an easy task. Infact motion in egocentric vision is a 
complex attribute to identify. We need to distinguish what is 
independent of the camera motion while dealing with the camera large 
viewpoint change and parallax that generate a large apparent motion.

In a static camera scenario the problem of separating the foreground from
the static background would be easily solved by recurring to 
background subtraction techniques, but the parallax effect of a moving camera makes 
this technique useless. As an example provided in the paper we can think of a egocentric
video of a person cooking: the actor behave in the scene by moving (and trasforming) objects.
However, egomotion is the dominant effect since objects move only sporadically,
and in a way that is hardly distinguishable from the much larger apparent motion induced 
by the viewpoint change, making it very difficult to segment dynamic objects automatically.

Even motion segmentation techniques struggles to separate  a scene in different motion components,
since they require correspondences, they reason locally, across a handful frames and usually
avoid explicit 3D reasoning, making it difficult to treat egocentric videos with many small
objects that move only occasionally throughout a long sequence.

\paragraph{Neural rendering for dynamic videos.}
The spread of neural rendering with NeRF (Section~\ref{sec:nerf}) paved the way 
for new research also in dynamic scenarios(see NeRF-W~\cite{ndiff_17}). 
Another direction tried focusing on modeling dynamic scenes mostly with monocular 
videos as input~\cite{ndiff_15,ndiff_34}. Most of this approaches use a 
canonical model in conjunction with a deformation network, or warp space~\cite{ndiff_34},
starting from a canonical volume. Closer to the work presented is~\cite{ndiff_15},
where a static NeRF model is combined with a dynamic one.

Anyway none of the previous managed to segment 3D objects in such long and challenging videos.

\paragraph{Contribution.}With this work the authors take inspiration from neural rendering techniques~\cite{nerf}
to create a motion analysis tool to obtain their goal. In particular they leverage the
ability of reconstructing accurately 3D scenes for recovering the background and then build
on top of it the dynamic parts. In fact 3D objects that are manipulated in the video also
present a significant structure. They usally move in "bursts", changing their state
when they are involved in an interaction, otherwise staying rigidly attached to the 
background. Exploiting this property  they extend the neural render to reconstruct
the moving object appearance using a slowly-varying time encoding. The last part that 
show unique properties in the scene is the actor due to its continuous movement
while occluding the scene and with a motion linked to the camera.

Summing up they ended up with a three-stream neural rendering architecture, where each stream
models respectively: \textit{static background},\textit{dynamic foreground} and
the \textit{actor}. These are then combined to explain the video as a whole. In Figure~\ref{fig:ndiff_1}
is reported a general overview of NeuralDiff capabilities.
\begin{figure}[t]
    \centering
    \includegraphics[width=1\linewidth]{images/relatedWorks/Neural/ndiff_1.png} 
    \caption{Given an egocentric video with camera reconstruction, NeuralDiff, a neural
    architecture, learns how to decompose each frame into a static background
    and a a dynamic foreground, which includes every object that sooner or later will move
    and the actor's body parts. Each of these streams is learned exploiting the characteristics
    of the scene that is going to be captured. Being a neural radiance field, NeuralDiff is 
    also capable to render images from novel viewpoints as can be seen in the bottom right part
    of the scene.}\label{fig:ndiff_1}
\end{figure}

Each stream is designed differently, in order to include inductive biases that match the statistics
of each layer(background,foreground,actor). The resulting analysis-via-synthesis method
shows that neural rendering techniques are also useful for analysis, and not just synthesis.
In particular NeuralDiff was the first to demonstrate the effectiveness of these techniques
in interpreting challenging egocentric videos, providing indications for the extraction of moving
objects in scenes with a complex 3D structure and dynamics.

For evaluating their results, they augmented the EPIC-KITCHENS dataset~\cite{EPICKITCHENS} by 
manually segmenting all objects that move at some point. In this way they are able to 
assess the quality of the decomposition of the different components. In addition they defined
a new benchmark for measuring progress in the challenging task of dynamic object segmentation
in complex videos, aiming to promote research in the area.


\subsection{Method}
The goal of this method is to extract a mask from an input video sequence that discern foreground
objects from background objects. It is worth noting that in this paper \textit{background} is 
the part of the scene that remain static throughout the entire video; while \textit{foreground}
is defined as any object that moves independently of the camera in at least one frame.

The basic block upon which NeuralDiff is built upon is NeRF(see Section~\ref{sec:nerf}) that can predict
the appearance of a static object,\textit{i.e.} the background, from different viewpoints. 
The authors also suggest a modification of the basic NeRF which is capable of 
distinguish foreground objects and it will be described in the following sections.

\subsubsection{Static Components}
As reported in Section~\ref{sec:nerf}, a video x is a collection $(x_t)_{t\in[0,...,T-1]}$
of T RGB frames $x_t \in \mathbb{R}^{3xHxW}$. Each of these frames can be seen as 
a function $x_t = h(B,F_t,g_t)$, where: B is the static background, $F_t$ is the variable
Foreground and $g_t$ is the moving camera. The motion and parameters of the camera are assumed
to be known, usually being extracted via a SfM algorithm such as COLMAP~\cite{colmap} which is 
explained in detail in Section~\ref{sec:col}. The background and foreground layers include the shape 
and reflectance of the 3D surfaces in the scene as well as the illumination.

Instead of trying to invert the function h to get B and $F_t$, which is known as \textit{inverse rendering},
neural rendering directly learns h using a neural network f, $h(B,F_t,g_t)=f(g_t,t)$, provided
the time and the camera viewpoint for the frame $x_t$. Viewpoint g and time t can be 
factorized by a careful desing of the function f, thus f can be used to generalize new unobserved
viewpoints. In the basic NeRF implementation the main assumption is that the scene is static,
meaning that $F_t$ is empty and f can be rewritten as $f(g_t)$. The color of each frame is then 
obtained by a volumetric sampling process that simulates ray casting. More in detail
the pixel color $x_{ut}\in \mathbb{R}^3$ of pixel $u \in \Omega = \{0,...,H-1\}\times\{0,...,W-1\}$ is obtained
by averaging the color of the 3D points along the ray $g_t r_k$ weighted by the probability
that a photon emanates from the point and reaches the camera.

A neural network, $(\sigma_k^b,c_k^b) = MLP^b(g_t r_k,d_t)$ retrieve the density 
$\sigma_k^b \in\mathbb{R}_{+}$ and the color $c_k^b \in \mathbb{R}^{3}$ of each point $g_t r_k$,
where the superscript $b$ refers to the fact that we are dealing with the static background and
$d_t$ is the unit-norm viewing direction.

A photon while travelling along the segment $(r_k,r_{k+1})$ is transmitted with probability
$T_k^b = e^{-\delta_k \sigma_k^b}$ where the quantity $\delta_k = |r_{k+1} - r_k|$ is the length of 
the segment. This definition allows to calculate the probability of transmission across several segments
as the product of the individual transmission probabilities. The color of pixel u can thus be written as:
\begin{equation}
    x_{ut} = f_u(g_t) = \sum_{k=0}^{M} v_k (1-T_k^b)c_k^b, \quad v_k = \prod_{q=0}^{k-1} T_q^b
\end{equation}
The model is trained by minimizing the reconstruction error $\Vert x - f(g_t)\Vert $.

\subsubsection{Dynamic Components}
To econstruct egocentric videos anyway we will deal with moving objects, so we can
not neglect the foreground $F_t$. To capture this layer they propose to build on top 
of the background $MLP^b$ a foreground-specific 
$MLP^f$,$(\sigma_k^f,c_k^f,\beta_k^f) = MLP^f(g_t r_k, z_t^f)$. Its outputs are a 'foreground'
occupancy $\sigma^f$ and color $c^f$. It also predict an uncertainty score $\beta_k^f$
whose role is discussed in the next paragraphs. The variable $z_t^f$ is introduced 
to capture the properties of the foreground that change over time.

The color $x_{ut}$ of a pixel u is obtained by the composition of both background and
foreground, so $\mathcal{S} = {b,f}$:
\begin{equation}\label{eq:2}
    x_{ut} = f_{u}(g_t,z_t) = \sum_{k=0}^{M} v_k(\sum_{p \in \mathcal{S}} w^p(T_k)c_k^p)
    \quad where \quad v_k = \prod_{q=0}^{k-1}\prod_{p \in \mathcal{S}} T_q^p
\end{equation}
The factor $v_k$ reqquires a photon to be transmitted from the camera to point $r_k$
through different materials. The weights $w^p(T_k)$ mix the colors based on points densities.
As done in NeRF-W~\cite{nerfw}:
\begin{equation}\label{eq:weights}
    w^p(T_k) = 1 - T_k^p \in [0,1]
\end{equation}

\paragraph{Smooth Dynamics} The model can now be optimized by minimizing the loss across
all input frames:
\begin{equation}\label{eq:timeNdiff}
    \min_{f,z_1,...z_T} \frac{1}{T |\Omega|} \sum_{t=1}^{T} \sum_{u\in\Omega}\Vert x_t-f(g_t,z_t)\Vert^2 
\end{equation}

However there is a problem with the characteristics of foreground, because, even if 
dynamic, it does not change at every frame in fact most of objects spend most of 
their time rigidly attached to the background. This make the dependecy on independent
frame-specific codes $z_t$ almost useless. They came up with the idea of 
replacing it with a low-rank expansion of the trajectory of sates, taking 
$z_t = B(t) \Gamma$ where $B(t) \in \mathbb{R}^p$ is a fixed basis and the motion
$\Gamma \in \mathbb{R}^{P \times D}$ are coefficients such that $P << T$.
In particular $B(t)=[1,t,sin 2 \pi t, cos 2 1pi t, sin 4 \pi t, cos 4 \pi t,...]$

\textbf{Improved Geometry: capturing the actor}
The foreground layer can be divided again in objects manipulated by the actor,
 that can move sporadically, and the actor, which instead moves continually.
 To detect the actor a third MLP is integrated to the previous model.
 The actor's MLP is similar to the foreground MLP: $(\sigma_k^a,c_k^a.\beta_k^a) = MLP^a(r_k,z_t^a)$.
 The main difference is that this time the 3D point $r_k$ is expressed with respect 
 to the camera, and not to the world ($g_t r_k$). This follows the physical properties
 of the recording stage, the camera is in fact attached to the actor head, making his
 body parts almost always present in front of the camera therefore it shows a reduced
 variability in the reference frame of the camera. On the contrary, the background and 
 foreground is invariant with respect to the world reference system.

 \textbf{Improved Color Mixing}
A principled mixing model is proposed to replace Equation~\ref{eq:weights}. It is obtained
by dividing the segment $\delta_k$ in Pn sub-segments, alternating between the P different
materials($P = \vert \mathcal{S}\vert$, $e.g. P = 3$ if all three layers are considered).
They show that the probability that a photon is absorbed in a subsegment of material p
is given by:
\begin{equation}
    w^p(T_k) = \frac{\sigma_k^p}{\sum_{q=1}^P \sigma_k^q} (1-\prod_{q=1}^{P} T_k^q)
\end{equation}
where the first term represent the probability that a given material p is responsible
for the absorption. The latter one instead is the probability that the photon is absorbed by all 
the materials involved.

\subsubsection{Uncertainty and regularization}
\textbf{Uncertainty.} Here we clarify the role of the previously announced $\beta_k^p$ variable.
It is predicted from each MLPs ($\beta_k^b = 0$ for the background) and represent the uncertainty
of the color associated to each 3D point along the ray $r_k$ for each layer p as pseudo-standard
deviations(StDs). As reported in~\cite{ndiff_17}, the StD of a rendered color $x_{ut}$ is given
as the sum of all the StDs for each material: $\beta_{ut} = \sum_{p} \beta_{u t}^p$ where 
$\beta_{u t}^p$ is obtained from Equation~\ref{eq:2} by replacing $c_k^p$ with $\beta_k^p$.
Now we can introduce a probabilistic loss:
\begin{equation}
    \mathcal{L}_{prob}(f,z_t \vert x_t, g_t,u) = \frac{\Vert x_{ut} - f_u(g_t,z_t)\Vert^2}{2 \beta_{ut}^2}
\end{equation}
\textbf{Sparsity.} The occupancy of the foreground and actor components is further penalized
by using:

\begin{equation}
    \mathcal{L}_{sparse}(f,z_t \vert x_t, g_t,u) = \sum_{p=1}^{P}\sum_{k=0}^{M} \sigma_k^p
\end{equation}
\textbf{Training Loss.} Finally, the loss used for training the model is:
\begin{equation}
    \mathcal{L} = \mathcal{L}_{prob}+\lambda \mathcal{L}_{sparse}
\end{equation}
where $\lambda>0$ is a weight set to 0.01.

\subsection{EPIC-Diff benchmark}
The goal of this paper was to identify any 'detachable' object, namely a onject that moves independently
from the camera. An extension of EPIC-KITCHENS~\cite{EPICKITCHENS} was introduced to give an evaluation to their method.

\paragraph{Data Selection.}10 videos sequences,or also called \textit{scene}, were selected from EPIC-Kitchen, each lasting 14 minutes
on average. Then 1000 frames where sampled from each and fed to COLMAP~\cite{colmap} to obtain
camera reconstructions. The scenes followed these constraints. $1)$The videos
should contain different viewpoints and multiple manipulated objects.$2)$ COLMAP should reconstruct
the sequence with at least 600 frames. In the end the obtain 10 sequence with an average of 900 frames.

\paragraph{Data annotation.}
Being an unsupervised algorithm, the only data annotations that collected were for testing and 
validation. They uniformly hold out 56 frames on average for validation(for setting parameters)
and for testing. The latter were manually annotated with segmentation pixelwise binary masks to assess static/dynamic components.
The test frames are not used for training such that they can be used also to also evaluate 
novel-view synthesis. In total they obtained 560 manual image-level segmentation masks. An example can
be see in Figure~\ref{fig:exMasks}
\begin{figure}[t]
    \centering
    \includegraphics[width=1\linewidth]{images/relatedWorks/Neural/ExampleMasks.png} 
    \caption{Examples of frames with their corresponding manually segmented binary pixelwise masks.}\label{fig:exMasks}
\end{figure}

\paragraph{Evaluation.}The task evaluated is background subtraction. To accomplish this they
used standard segmentation metrics: each frames is demposed in its pixels and a mask is 
extracted from the predicted frame. Then it is compared with its ground-truth calculating 
average precision(AP). Each AP is then averaged with every frame and scene to obtain mean 
average precision (mAP).

For novel view synthesis instead PSNR is used. Specifically they provided,using the 
ground-truth masks, the PSNR of the static and of the moving parts.\vspace{0.4cm}

\paragraph{Results.} The experiments are based on a PyTorch implementation of the model
that has been published on \url{https://github.com/dichotomies/NeuralDiff}, more details 
can be found in the paper. 

The Baselines compared are:

$(1)$\textbf{NeRF}~\cite{nerf} which uses a single stream to predict static scenes.

$(2)$\textbf{NeRF-BF} trains two NeRF models in parallel, one for the Background (B) and the
other for the foreground (F). The latter stream is conditioned on time by passing a
positional-encoded version of the time variable, that in this case corresponds with 
the frame number. This differ from NeuralDiff time encoding for Equation~\ref{eq:timeNdiff}.

$(3)$\textbf{NeRF-W}~\cite{ndiff_17} also contains two interlinked background and foreground
streams. Anyway it was designed to deal with image collection, so it struggles to 
adapt to videos. Some adjustments were done to adapt it to this task, more details
are reported in the paper.

%In Table~\ref{tab:ndiff_res} are reported the results for the various model previously
%presented for the evaluation of their capacity to discover and segment 3D objects
%but also reconstruct dynamic scenes. It is worth noting that NeuralDiff largely improve
%the results with respect to other methods. Also the temporal information confirms to 
%be fundamental for improving the performance over NeRF and NeRF-W. As a third observation
%each update proposed upon the vanilla NeuralDiff proveed to be succesful. Hoewever
%we have to remind that these metrics do not reflect the ability of the full model to 
%separate foreground into objects and actor, since they are merged together in 
%the annotated masks. This ability can be seen in Figure~\ref{fig:ndiff_1}.
%%\begin{table}
    
\centering
\begin{tabular}{lcccc}
    \hline Method & mAP & PSNR & PSNR $_b$ & PSNR $_f$ \\
    \hline NeRF ~\cite{nerf} & 47.8 & 20.9 & 22.8 & 17.6 \\
    NeRF-W ~\cite{nerfw} & 59.2 & 23.2 & 26.4 & 18.9 \\
    NeRF-BF & 64.4 & 23.8 & 26.8 & 19.6 \\
    NeuralDiff  & 66.7 & 24.0 & 27.2 & 19.8 \\
    NeuralDiff+A  & $\mathbf{6 9 . 1}$ & 24.1 & $\mathbf{2 7 . 3}$ & 19.9 \\
    NeuralDiff+C  & 67.4 & 24.1 & 27.2 & 19.9 \\
    NeuralDiff+C+A  & 67.8 & $\mathbf{2 4 . 2}$ & $\mathbf{2 7 . 3}$ & $\mathbf{2 0 . 0}$ \\
    \hline
\end{tabular}
\caption{Results} \label{tab:ndiff_res}
\end{table}

\textbf{Qualitative Results.} We report just some qualitative results to grasp the effective 
results obtained from their method. In Figure~\ref{fig:ndiff_qualitative} the output of the compared
methods is reported for three different scenes.It can be seen that NeuralDiff produce less ghosting artifacts,
captures most moving objects and shows more details, especially the \textit{upgraded} version.
In particular: NeRF struggles with the dynamic components and create blurry
reconstructions; NeRF-W obtains sharper static images but still struggle with
dynamic regions; NeuralDiff produces sharper results but the \textit{upgraded}
version capture more details, such as the arms in the efirst scene or the spoon in the
second.
\begin{figure}[t]
    \centering
    \includegraphics[width=1\linewidth]{images/relatedWorks/Neural/qualRes.png} 
    \caption{Three Scenes reconstruction from NeRF, NeRF-W and NeuralDiff, NeuralDiff+C+A.
     It can be seen that NeuralDiff produce less ghosting artifacts, captures most moving
     objects and shows more details, especially the \textit{upgraded} version.}\label{fig:ndiff_qualitative}
\end{figure}
Another qualitative result is shown in Figure~\ref{fig:qual2}.
\begin{figure}[t]
    \centering
    \includegraphics[width=1\linewidth]{images/relatedWorks/Neural/qual2.png} 
    \caption{\textbf{Segmentation Masks.}Here the masks for which NeuralDiff scored best(top 4 rows)
    and worst (last 2 rows) are reported.}\label{fig:qual2}
\end{figure}
In the best segmentation case for NeuralDiff+A, NeRF-WW create noisy results, where 
the plate, the pasta colander and the actor's body are barely captured. NeuralDiff 
instead manage to capture all moving objects and more body parts but misclassifies 
floor as foreground. NeuralDiff-A better estimates the shape of the actor's body(second and third row)
and correctly recognize the floor as static.

For the failure case instead, NeuralDiff still beats NeRF-W. In detail, NeRF-W fail 
in recognizing any foreground object. NeuralDiff performs a little better by discriminating
some foreground objects but still including some of the static background as foreground
(e.g. Last row, part of the sink is reported as foreground).