\chapter{Neural Rendering}\label{sec:Neural}
From CVPR 2020 tutorial~\cite{CVPRtutorial} Neural Rendering is defined as \textit{'
 a new class of deep image and video generation approaches that 
 enable explicit or implicit control of scene properties such as 
 illumination, camera parameters, pose, geometry, appearance, and 
 semantic structure. It combines generative machine learning techniques 
 with physical knowledge from computer graphics to obtain controllable 
 and photo-realistic outputs'}.

\section{Related works}
Before NeRF a promising direction in computer vision
 consisted in encoding objects and scenes in the weights of an MLP. Hower this
 method still struggled compared to other methods based on discrete representations
 like triangle meshes or voxel grids. The main methods 
 that preeceded neural rendering are reported in the following paragraphs.
 
 \paragraph{Neural 3D shape representations.} Implicit continous 3D shapes 
 representation as level sets had been investigated by~\cite{nerf15,nerf32}, optimizing
 signed distance functions or occupancy fields in~\cite{nerf11,nerf27}.
 These methods were limited by the necessity of a 3D ground truth, which is 
 rarely available. Subsequent works reformulated the problem using differentiable
 rendering functions that allowed to use just 2D images as ground truth. Two were the main methods:
 ~\cite{nerf29} represents surfaces as 3D occupancy fields;~\cite{nerf42}use a 
 representation that outputs a feature vector and RGB color at each continous 
 coordinate, proposing a differentiable rendering function consisting of a 
 RNN(Recurrent neural network).
 
 Anyway these methods have been limited to simple shapes with low complexity,
 resulting in oversmoothed renderings.
 
 \paragraph{View synthesis and image-based rendering.} Photorealistic novel views 
 can be reconstructed if a dense sampling of images have been provided by 
 simple light field sample interpolation techniques~\cite{nerf21,nerf5}.
 If sparser images are available, some methods have been introduced that 
 rely on extracting traditional geometry and appearance representations
 from observed images. A popular class of approached uses mesh-based representations
 of scene with diffuse~\cite{nerf48} or view-dependent~\cite{nerf2} appearance. Also there are differentiable
 rasterizers~\cite{nerf4,nerf10,nerf23} that can directly optimize mesh recontruction to reproduce
 a set of input images using gradient descent. However due to local minima or 
 poor conditioning of the loss landscape, the optimization is usually difficult.
 
 High-quality photorealistic view synthesis is also performed by volumetric representations,
 from a set of given RGB images. Volumetric methods can represent complex shapes and materials
 ,well suited for gradient-based optimization. First works~\cite{nerf19} used observed images to directly predict color voxel grids.
 Later~\cite{nerf9,nerf13,nerf17} used deep networks to color a sampled volumetric region from some given images.
 Other works tried using a mix of convolutional networks and sampled voxel grids.
 
 Anyway these methods, being based on discrete sampling of the voxel grid, are restricted to low resolution due to time 
 and computational power. With neural rendering instead a \textit{continous} representation was proposed, reducing 
 the memory requirements and producing higher quality renderings.
 
\paragraph{Background Subtraction}
Background subtraction techniques have been used to detect moving objects 
in videos(see~\cite{ndiff_2}). The simplest way to obtain the background would be to capture
a background image that does not contain any foreground object. Unfortunately in some scenes
it is not possible and it could be changed under critical situations like illumination
changes, objects appeearing and disappearing from the scene. Some expedients have been
introduced that try to predict the background via a background initialization step, which base
its guess on the first few frames of the video. Anyway egocentric videos still contains too much
challenging problems(light change,multiple different objects,moving camera,etc.) that do not allow background subtraction to be a viable solution.

\paragraph{Motion segmentation}
Motion segmentation consists in decomposing a video into individually moving 
objects~\cite{ndiff_18}. Amongst others, it has applications in 
robotics, traffic monitoring, sports analysis, inspection, video surveillance and compression.
Anyway these techniques usually relies on optical flow, which can be subject to some ambiguities
like, in our case,motion parallax. Even occlusion plays an important role. Motion segmentation
struggles for example when an object moves in front of or behind other objects in the 
scene, leading to ambiguity in the flow field. Also, many methods fails when a dynamic 
object temporarilly remain static.

All these problems prevent to apply motion segmentation algorithms to egocentric videos.

\paragraph{Discovering and segmenting objects in videos}
Discovering and segmenting objects in videos is related to background subtraction and 
motion segmentation.
As instance moving objects can be segmented from the background by using a probabilistic model
that acts on optical flow~\cite{ndiff_1}. In~\cite{ndiff_3}, pixel trajectories and
spectral clustering are combined to produce motion segments. Some recent works revisits classical motion segmentation
techniques from a data-driven perspective~\cite{ndiff_38}, \textit{e.g.} using 
physical motion cues to learn 3D representations or learning a scene representation
using neural rendering~\ref{sec:nerf}. 

With NeuralDiff they obtain a holistic representation capable of handling occlusions
and detachable objects.


\section{NeRF:Representing Scenes as Neural Radiance Fields for View Synthesis}\label{sec:nerf}
With NeRF~\cite{nerf} the authors introduced a new state-of-the-art model
for synthesizing novel views of complex scenes by just using a set of sparse
images and their relative positions.
\subsection{Intro?}
With this work they deal with the novel view synthesis problem of complex optimization
by modeling the scene representation with a fully-connected neural network(MLP) without any
convolutional layer, this is called \textit{neural radiance field}(NeRF). The network
take as input the position of a point $(x,y,z)$ and the direction $(\theta,\phi)$ from which we are looking it
and gives as results the color$(r,g,b)$ and density $(\sigma)$ of that point. 

To render a NeRF from a viewpoint one can:$1)$ march camera rays in the 
scene and sample some points from them $2)$ use those points as input to the neural
network to produce an output set of colors and densities, and $3)$ use volume 
rendering techniques to obtain a final 2D image. All the previous steps
are differentiable such tat we can use gradient descent to optimize the model
by minimizing the error between each image and the corresponding prediction.
Repeating this process from multiple viewpoints encourages the network to 
grasp the 3D scene representation. In Figure~\ref{fig:batteria} is reported an example
,presented in the paper,of the main steps.
\begin{figure}
    \centering
    \includegraphics[width=1\linewidth]{images/relatedWorks/batteria.png} 
    \caption{\textbf{NeRF.}Optimization of a continous 5D neural radiance field
    representation(volume density and view-dependent color at any continuous location)
    of a scene from a seet of input images. The 2D novel views are obtained thanks
    to classic volume rendering techniques. Here in this example, given 100
    images acquired from different viewpoints, they sample two novel views.}\label{fig:batteria}
\end{figure}

Anyway this approch is not enough for a complex scene, in fact the optimization 
does not converge to a sufficiently high-resolution representation. This problem
is solved by trasforming the input coordinates with a positional encoding
that enables the network to represent higher frequency functions.
Summing up, their contribution consists in:
\begin{itemize}
    \item An approach for representinc continous scenes as a 5D neural radiance fields.
    \item A differentiable rendering pipeline based on classical volume rendering
    techniques,used to optimize the scene representation via input images.
    \item A positional encoding to map each input 5D coordinate into a higher dimensional
    space which allows to represent high-frequency scene content.
\end{itemize}
\subsection{Method}
A continous 3D scene is represented as a 5D vector-valued function whose input is a 3D coordinate 
$\textbf{x}=(x,y,z)$ and 2D viewing direction $\textbf{d}=(\theta,\phi)$, and whose output is an emited color 
$\textbf{c} = (r,g,b)$ and volume density $\sigma$. In practice the scene is represented by an MLP network
$F_{\Theta}:(\textbf{x},\textbf{d}) \rightarrow (\textbf{c},\sigma)$, where $\Theta$ are the weights of the network.

The representation is encouraged to be multiview consistent by restricting the network to predict the 
volume density $\sigma$ as a function of just the location \textbf{x}, while the color \textbf{c} is a
function of both the input, location and direction. To do this the MLP $F_{\Theta}$ first processes the 
input \textbf{x} with 8 fully-connected layers (using ReLU activation functions and 256 channels per layer),
and outputs $\sigma$ and a 256-dimensional feature vector. The scheme is summed up in Figure~\ref{fig:mlp}.
\begin{figure}
    \centering
    \includegraphics[width=0.8\linewidth]{images/relatedWorks/Neural/NeRF_mlp.png} 
    \caption{\textbf{$F_{\Theta}$ Scheme.} The input position \textbf{x} pass through 8 Fully connected
    (FC) layers of 256-channels. Each FC layer is followed by a ReLU activation function. This intermediate
    result is then concatenated with the input direction (\textbf{d}) and fed to
    one last FC with 128 channels that feeds its output to a ReLU function. The output of the ReLU
    are the color \textbf{c} and the volume density ($\sigma$).}\label{fig:mlp}
\end{figure}

The effects of the direction in input can be seen in Figure~\ref{fig:lego}.
\begin{figure}
    \centering
    \includegraphics[width=1\linewidth]{images/relatedWorks/lego.png} 
    \caption{Here are reported the results obtained with different strategies, as written underneath each image.
     In particular removing view dependence prevents the model from recreating the specular 
     reflection on the bulldozer tread. Removing the Positional encoding instead we 
     obtain a blurred image, meaning that high frequencies are not captured nor represented.}\label{fig:lego}
\end{figure}

\subsubsection{Volume Rendering with Radiance Fields}
The implicit representation of the scene relies on the volume densities and on the color
of every point in that scene. The color of any ray passing through the scene is rendered using
principles from classical volume rendering~\cite{nerf16}. \textbf{The volume density $\sigma(\textbf{x})$
can be interpreted as the differential probability of a ray terminating at an infinitesimal
particle at location \textbf{x}.} While the expected color C(\textbf{r}) of camera ray 
\textbf{r}(t) = \textbf{o}+ t\textbf{d} with near and far bounds $t_n$ and $t_f$ is:
\begin{equation}
    C(\textbf{r}) = \int_{t_n}^{t_f} T(t)\sigma(\textbf{r}(t))\textbf{c}(\textbf{\textbf{r}(t),\textbf{d}}) \,dt 
\end{equation} 
where $T(t)$ denotes the accumulated transmittance along the ray from $t_n$ to $t$,i.e:
the probability that the ray travels from $t_n$ to $t$ without hitting any other particle.Namely:
\begin{equation}
    T(t) = exp(-\int_{t_n}^{t}\sigma(\textbf{r}(s))\,ds)
\end{equation}
To obtain a new view in our neural radiance field, we should estimate the $C(\textbf{r})$
function for each ray passing through each pixel of the focal plane(see Figure~\ref{fig:rt}).
\begin{figure}
    \centering
    \includegraphics[width=0.5\linewidth]{images/relatedWorks/Neural/ray_tracing.png} 
    \caption{Example of rays passing through an image plane of size 3x3 pixels.}\label{fig:rt}
\end{figure}
This integral is approximated using a numerical method known as \textit{quadrature}.Typically deterministic 
quadrature is used for rendering discretized voxel grids, but in our case it would limit our model's resolution
since the network would only be queried at a fixed discrete set of locations. To solve this problem a
\textit{stratified} sampling approach has been used, where each interval $[t_n,t_f]$ has been partitioned
into N evenly-spaced bins, from which a random sample is then uniformly extracted. Namely:
\begin{equation}
    t_i  \thicksim \mathcal{U} [t_n + \frac{i-1}{N}(t_f-t_n), t_n + \frac{i}{N}(t_f-t_n)]
\end{equation}
In this way, even if we are using a discrete set of samples, using stratified sampling allows us to represent a continuous scene 
representation because the network is evaluated at continuous positions during the training phase.
Following the quadrature rule discussed in~\cite{nerf26} they used the samples to estimate $C(\textbf{r})$:
\begin{equation}\label{eq:neural_C}
    \widehat{C}(\textbf{r}) = \sum_{i=1}^N T_i (1-exp(-\sigma_i \delta_i))\textbf{c}_i, \quad where \quad T_i = exp(-\sum_{j=1}^{i-1}\sigma_j\delta_j)
\end{equation}
where $\delta_i =t_(i+1)$ is the distance between adjacent samples. It can be noticed that the function that approximate $C(\textbf{r})$ is differentiable and can be 
reduced to traditional \textit{alpha compositing}\footnote{\textit{Alpha compositing} is a digital imaging technique used to combine multiple layers of images or graphics with transparency, known as alpha channels.}
with alpha values being $\alpha_i = 1-exp(-\sigma_i \delta_i)$.

\subsubsection{Optimizing a Neural Radiance Field}
In the previous section the core components of a Neural Radiance Field were presented, anyway these partss alone are not able to 
achieve state-of-the-art results. To improve the quality of the representation two improvements were found succesful:
\begin{itemize}
    \item \textbf{Positional Encoding} of the input coordinates, to encourage the representation of high-frequencies.
    \item \textbf{Hierarchical Sampling} procedure that allows to efficiently sample high-frequency representation.
\end{itemize} 

\textbf{Positional Encoding.} After having found that the model $F_\Theta$ performed poorly operating solely on $xyz\theta\phi$ input, in accordance
to the work by Rahaman \textit{et al.}~\cite{nerf35} that states that deep networks are biased towards
learning low-frequency functions; they encode the inputs to a higher dimensional space using high frequency functions. 

The new model $F_\Theta$ can thus be represented as a combinatino of functions $F_\Theta = F_\Theta'\circ \gamma$, where $\gamma$ is a function from $\mathbb{R}$ 
to a higher dimensional space $\mathbb{R}^{2L}$, and  $F_\Theta'$ is still the basic block introduced in the previous sections.
Formally the function used for the encoding part is:
\begin{equation}
    \gamma(p) = (sin(2^0 \pi p), cos(2^0 \pi p), ...,sin(2^L-1 \pi p), cos(2^L-1 \pi p))
\end{equation}
In particular $\gamma(\dot)$ is applied  separately to each component of \textbf{x} and \textbf{d}, after these values are normalized to lie in $[-1,1]$.
Reporting the paper results, they found out that good values of L were: 10 for $\gamma(\textbf{x})$ and 4 for $\gamma(\textbf{d})$.

\textbf{Hierarchical Sampling.} Stratified sampling allows to cover continuous
regions but anyway is still inefficient: free space and occluded regions that do not 
contribute to the rendered image are still sampled repeatedly. To solve this 
problem the authors proposed a \textit{hierarchical} representation which increased
rendering efficiency by distributing samples proportionally to their expected 
effect on the final rendering.

This solution consists in simultaneously optimize two networks: one "coarse" and
one "fine". The first step expect to sample a set of $N_c$ points using stratified 
sampling and feed those to the coarse model. Once this first partial result has 
been obtained a more informed sampling of points is prooduced. The coarse sampling 
infact allows to get a rough idea of which parts of the volume are the most relevant.
To do this they rewrite the alpha composited color from the coarse model $\hat{C}_c(\textbf{r})$
in Eq.~\ref{eq:neural_C} as a weighted sum of all sampled colors $c_i$ along the ray:
\begin{equation}
    \hat{C}_c(\textbf{r}) = \sum_{i=1}^{N_c} w_i c_i, \quad \quad w_i = T_i(1-exp(-\sigma_i \delta_i))
\end{equation}
By normalizing the weights as $\hat{w}_i = \frac{w_i}{\sum_{j=1}^{N_c}w_j}$ we can produce
a piecewise-constant probability density function along the ray as seen in Figure~\ref{fig:pdf_ray}.
\begin{figure}
    \centering
    \includegraphics[width=0.8\linewidth]{images/relatedWorks/Neural/pdf_rays.png} 
    \caption{PDF of normalized coarse weights $\hat{w}_i$ along a ray with $N_c$ samples.}\label{fig:pdf_ray}
\end{figure}
The next $N_f$ samples are extraced from this distribution and then, combined to the previous $N_c$ samples,
fed to the "fine" network. The final rendered color $\hat{C}_f(\textbf{r})$ is obtained using
Eq.~\ref{eq:neural_C} with $N_c+N_f$. This procedure revealed succesful in obtaining more
samples from regions we expect to contain visible content.

\subsubsection{Actual Implementation}
To optimize a scene RGB frames are required with the corresponding camera poses and intrinsic
parameters(for synthetic data these are easily retrievable from the scene model; while for
real images COLMAP was used to extract them).
At each iteration a batch of rays from the set of all pixels of all images of the dataset
is extracted and following the sampling procedure previosuly described the actual color 
is predicted for both the "coarse" and the "fine" model. The loss is computed as the 
total squared error betwwen the rendered and true pixel colors for the two models:
\begin{equation}
    \mathcal{L} = \sum_{r\in\mathcal{R} }[\left\lVert {\hat{C}_c(\textbf{r})-C(\textbf{r})} \right\rVert_2^2+\left\lVert {\hat{C}_f(\textbf{r})-C(\textbf{r})} \right\rVert_2^2]
\end{equation}

where $\mathcal{R}$ is the set of rays of each batch, $C(\textbf{r})$ is the RGB color
ground truth and $\hat{C}_c(\textbf{r})$,$\hat{C}_f(\textbf{r})$ are the predicted color
for the "coarse" and the "fine" model for ray \textbf{r}.

As for some more specific details, they used a batch size of 4086 rays, each
sampled at $N_c=64$ points for the "coarse" model and $N_f=128$ for the "fine"
model. They used the Adam optimizer with a learning rate beginning at $5x10^{-4}$
and decays exponentially to $5x10^-5$ over the course of optimization. Others parameters
of the optimizer where left at default values: $\beta_1 = 0.9,\beta_2=0.999$ and $\epsilon=10^{-7}$.
Each scene took them around 100-300l iterations to converge on NVIDIA V100 GPU(about 1-2 days).
Their tensorflow implementation is provided on \url{https://github.com/bmild/nerf}.

\subsection{Results}
To validate their results they generated and aggregated some datasets.
The main distinction is given by synthetic scenes, denoted as\textit{ Diffuse 
Synthetic 360$\degree$}~\cite{deepvoxels} and \textit{Realistic Synthetic 360$\degree$}, and 
realistic scenes,\textit{Real Forward Facing}. More in detail:
\begin{itemize}
    \item \textit{ Diffuse Synthetic 360$\degree$}: contains four Lambertian objects 
    with simple geometry. Each object is rendered at 512x512 pixels from viewpoints taken
    from the upper hemisphere.
    \item \textit{ Realistic Synthetic 360$\degree$}: new dataset that the authors introduce
    containing images of eight objects that exhibit complicated geometry and realistic non-Lambertian\footnote{Lambertian reflectance is the property that defines an ideal "matte" or diffusely reflecting surface. The apparent brightness of a Lambertian surface to an observer is the same regardless of the observer's angle of view~\cite{lambertian}.}
    materials. Six scenes were sampled from the upper hemisphere while the remaining two
    are rendered from a full sphere. For each scene 100 views are captured for training and 
    200 for testing, all at 800x800 pixels.
    \item \textit{Real Forward Facing}: consists of 8 scenes captured with a handheld cellphone
    (5 taken from the LLFF~\cite{LLFF} paper and 3 captured by them), captured with 20 to 62 images,
    holding out $1/8$ of these for the test set. All images have a resolution of 1008x756 pixels.
\end{itemize}

The algorithm which have been compared were the top-performing techniques for view synthesis and 
are the following:
\begin{itemize}
    \item \textbf{Neural Volumes (NV)}~\cite{neuralvol} synthesizes novel views of objects
    lying in a confined space with a distinct background(which must be captured alone).
    The model is based on a 3D convolutional network to predict a discretized RGB$\alpha$ voxel grid.
    \item \textbf{Scene Representation Networks (SRN)}~\cite{srn} represent a continuous scene
    as an opaque surface, implicitly defined by an MLP that maps spatial coordinates to a
    feature vector. The color is then obtained by training a RNN along the ray.
    \item \textbf{Local Light Field Fusion (LLFF)}~\cite{LLFF} is designed for producing photorealistic
    novel views for well-sampled forward facing scenes.
\end{itemize}
Quantitative results are reported from the paper in Table~\ref{tab:nerf_res}.
The metrics used were: PSNR,SSIM and LPIPS (see Section~\ref{sec:Metrics} for more details).
We can see that the method introduced outperformed past works in both realistic and 
synthetic scenarios.

\begin{table}

\resizebox{\textwidth}{!}{%
\begin{tabular}{l|ccc|ccc|ccc} 
    & \multicolumn{3}{|c|}{ Diffuse Synthetic $360^{\circ}$} & \multicolumn{3}{c|}{ Realistic Synthetic 360 } & \multicolumn{3}{c}{ Real Forward-Facing  } \\
    Method & PSNR $\uparrow$ & SSIM $\uparrow$ & LPIPS $\downarrow$ & PSNR $\uparrow$ & SSIM $\uparrow$ & LPIPS $\downarrow$ & PSNR $\uparrow$ & SSIM $\uparrow$ & LPIPS $\downarrow$ \\
    \hline SRN & 33.20 & 0.963 & 0.073 & 22.26 & 0.846 & 0.170 & 22.84 & 0.668 & 0.378 \\
    NV  & 29.62 & 0.929 & 0.099 & 26.05 & 0.893 & 0.160 & - & - & - \\
    LLFF  & 34.38 & 0.985 & 0.048 & 24.88 & 0.911 & 0.114 & 24.13 & 0.798 & $\mathbf{0 . 2 1 2}$ \\
    NeRF & $\mathbf{4 0 . 1 5}$ & $\mathbf{0 . 9 9 1}$ & $\mathbf{0 . 0 2 3}$ & $\mathbf{3 1 . 0 1}$ & $\mathbf{0 . 9 4 7}$ & $\mathbf{0 . 0 8 1}$ & $\mathbf{2 6 . 5 0}$ & $\mathbf{0 . 8 1 1}$ & 0.250
\end{tabular}}
\caption{\textbf{Quantitative results} In all datasets, for all metrics,except for the LPIPS,
the NeRF method outperforms old methods. }\label{tab:nerf_res}
\end{table}

A qualitative idea instead is given by Figure~\ref{fig:blender2} and ~\ref{fig:blender1}.
\begin{figure}[H]
    \centering
    \includegraphics[width=1\linewidth]{images/relatedWorks/Neural/blender2.png} 
    \caption{Comparison on test images from the newly introduced
    synthetic dataset. NeRF method is able to recover fine details in both geometry
    and appearance. LLFF exhibits some artifacts on the microphone and some ghosting
    artifact in the other scenes. SRN produces distorted and blurry rendering for every
    scene. Neural Volumesstruggle capturing details we can see from
    the ship reeconstruction.}\label{fig:blender2}
\end{figure}

\begin{figure}[H]
    \centering
    \includegraphics[width=1\linewidth]{images/relatedWorks/Neural/blender1.png} 
    \caption{Comparison on the test set of the real images.
    As expected LLFF is performing pretty well being projected
    for this specific use case(forward-facing captures of real scenes).
    Anyway NeRF is able to represent fine geometry more consistently across
    rendered views than LLFF as we can see in Fern's and in T-rex. NeRF
    is also able to reproduce partially occluded scene as in the second row.
    SRN instead completely fail to represent any high-frequency content.}\label{fig:blender1}
\end{figure}
An additional validation of their design choices is given by an ablation
study on the various parts that have been discussed in the implementation part.
In particular the result of the study is reported in Table~\ref{tab:nerf_ablation}.
\begin{table}

    \resizebox{\textwidth}{!}{%
\begin{tabular}{l|cccc|ccc} 
    & Input & \#Im. & $L$ & $\left(N_c, N_f\right)$ & PSNR $\uparrow$ & SSIM $\uparrow$ & LPIPS $\downarrow$ \\
    \hline 1) No PE, VD, H & $x y z$ & 100 & - & $(256,-)$ & 26.67 & 0.906 & 0.136 \\
    2) No Pos. Encoding & $x y z \theta \phi$ & 100 & - & $(64,128)$ & 28.77 & 0.924 & 0.108 \\
    3) No View Dependence & $x y z$ & 100 & 10 & $(64,128)$ & 27.66 & 0.925 & 0.117 \\
    4) No Hierarchical & $x y z \theta \phi$ & 100 & 10 & $(256,-)$ & 30.06 & 0.938 & 0.109 \\
    \hline 5) Far Fewer Images & $x y z \theta \phi$ & 25 & 10 & $(64,128)$ & 27.78 & 0.925 & 0.107 \\
    6) Fewer Images & $x y z \theta \phi$ & 50 & 10 & $(64,128)$ & 29.79 & 0.940 & 0.096 \\
    \hline 7) Fewer Frequencies & $x y z \theta \phi$ & 100 & 5 & $(64,128)$ & 30.59 & 0.944 & 0.088 \\
    8) More Frequencies & $x y z \theta \phi$ & 100 & 15 & $(64,128)$ & 30.81 & 0.946 & 0.096 \\
    \hline 9) Complete Model & $x y z \theta \phi$ & 100 & 10 & $(64,128)$ & $\mathbf{3 1 . 0 1}$ & $\mathbf{0 . 9 4 7}$ & $\mathbf{0 . 0 8 1}$
    \end{tabular}}
    \caption{\textbf{Quantitative results} In all datasets, for all metrics,except for the LPIPS,
the NeRF method outperforms old methods. }\label{tab:nerf_ablation}
\end{table}

\section{NeuralDiff: Segmenting 3D objects that move in egocentric videos}
NeuralDiff is a neural radiance field adapted to distinguish three different parts of
egocentric videos:$1)$ \textit{static}, that is everything that does not move,$2)$
\textit{foreground}, which is everything that at some point of the video moves, 
and $3)$ \textit{actor},which comprehends the body parts of the person who 
is wearing the camera. 

Anyway this is not an easy task. Infact motion in egocentric vision is a 
complex attribute to identify. We need to distinguish what is 
independent of the camera motion while dealing with the camera large 
viewpoint change and parallax that generate a large apparent motion.

In a static camera scenario the problem of separating the foreground from
the static background would be easily solved by recurring to 
background subtraction techniques, but the parallax effect of a moving camera makes 
this technique useless. As an example provided in the paper we can think of a egocentric
video of a person cooking: the actor behave in the scene by moving (and trasforming) objects.
However, egomotion is the dominant effect since objects move only sporadically,
and in a way that is hardly distinguishable from the much larger apparent motion induced 
by the viewpoint change, making it very difficult to segment dynamic objects automatically.

Even motion segmentation techniques struggles to separate  a scene in different motion components,
since they require correspondences, they reason locally, across a handful frames and usually
avoid explicit 3D reasoning, making it difficult to treat egocentric videos with many small
objects that move only occasionally throughout a long sequence.

\paragraph{Neural rendering for dynamic videos.}
The spread of neural rendering with NeRF (Section~\ref{sec:nerf}) paved the way 
for new research also in dynamic scenarios(see NeRF-W~\cite{ndiff_17}). 
Another direction tried focusing on modeling dynamic scenes mostly with monocular 
videos as input~\cite{ndiff_15,ndiff_34}. Most of this approaches use a 
canonical model in conjunction with a deformation network, or warp space~\cite{ndiff_34},
starting from a canonical volume. Closer to the work presented is~\cite{ndiff_15},
where a static NeRF model is combined with a dynamic one.

Anyway none of the previous managed to segment 3D objects in such long and challenging videos.

\paragraph{Contribution.}With this work the authors take inspiration from neural rendering techniques~\cite{nerf}
to create a motion analysis tool to obtain their goal. In particular they leverage the
ability of reconstructing accurately 3D scenes for recovering the background and then build
on top of it the dynamic parts. In fact 3D objects that are manipulated in the video also
present a significant structure. They usally move in "bursts", changing their state
when they are involved in an interaction, otherwise staying rigidly attached to the 
background. Exploiting this property  they extend the neural render to reconstruct
the moving object appearance using a slowly-varying time encoding. The last part that 
show unique properties in the scene is the actor due to its continuous movement
while occluding the scene and with a motion linked to the camera.

Summing up they ended up with a three-stream neural rendering architecture, where each stream
models respectively: \textit{static background},\textit{dynamic foreground} and
the \textit{actor}. These are then combined to explain the video as a whole. In Figure~\ref{fig:ndiff_1}
is reported a general overview of NeuralDiff capabilities.
\begin{figure}[t]
    \centering
    \includegraphics[width=1\linewidth]{images/relatedWorks/Neural/ndiff_1.png} 
    \caption{Given an egocentric video with camera reconstruction, NeuralDiff, a neural
    architecture, learns how to decompose each frame into a static background
    and a a dynamic foreground, which includes every object that sooner or later will move
    and the actor's body parts. Each of these streams is learned exploiting the characteristics
    of the scene that is going to be captured. Being a neural radiance field, NeuralDiff is 
    also capable to render images from novel viewpoints as can be seen in the bottom right part
    of the scene.}\label{fig:ndiff_1}
\end{figure}

Each stream is designed differently, in order to include inductive biases that match the statistics
of each layer(background,foreground,actor). The resulting analysis-via-synthesis method
shows that neural rendering techniques are also useful for analysis, and not just synthesis.
In particular NeuralDiff was the first to demonstrate the effectiveness of these techniques
in interpreting challenging egocentric videos, providing indications for the extraction of moving
objects in scenes with a complex 3D structure and dynamics.

For evaluating their results, they augmented the EPIC-KITCHENS dataset~\cite{EPICKITCHENS} by 
manually segmenting all objects that move at some point. In this way they are able to 
assess the quality of the decomposition of the different components. In addition they defined
a new benchmark for measuring progress in the challenging task of dynamic object segmentation
in complex videos, aiming to promote research in the area.


\subsection{Method}
The goal of this method is to extract a mask from an input video sequence that discern foreground
objects from background objects. It is worth noting that in this paper \textit{background} is 
the part of the scene that remain static throughout the entire video; while \textit{foreground}
is defined as any object that moves independently of the camera in at least one frame.

The basic block upon which NeuralDiff is built upon is NeRF(see Section~\ref{sec:nerf}) that can predict
the appearance of a static object,\textit{i.e.} the background, from different viewpoints. 
The authors also suggest a modification of the basic NeRF which is capable of 
distinguish foreground objects and it will be described in the following sections.

\subsubsection{Static Components}
As reported in Section~\ref{sec:nerf}, a video x is a collection $(x_t)_{t\in[0,...,T-1]}$
of T RGB frames $x_t \in \mathbb{R}^{3xHxW}$. Each of these frames can be seen as 
a function $x_t = h(B,F_t,g_t)$, where: B is the static background, $F_t$ is the variable
Foreground and $g_t$ is the moving camera. The motion and parameters of the camera are assumed
to be known, usually being extracted via a SfM algorithm such as COLMAP~\cite{colmap} which is 
explained in detail in Section~\ref{sec:col}. The background and foreground layers include the shape 
and reflectance of the 3D surfaces in the scene as well as the illumination.

Instead of trying to invert the function h to get B and $F_t$, which is known as \textit{inverse rendering},
neural rendering directly learns h using a neural network f, $h(B,F_t,g_t)=f(g_t,t)$, provided
the time and the camera viewpoint for the frame $x_t$. Viewpoint g and time t can be 
factorized by a careful desing of the function f, thus f can be used to generalize new unobserved
viewpoints. In the basic NeRF implementation the main assumption is that the scene is static,
meaning that $F_t$ is empty and f can be rewritten as $f(g_t)$. The color of each frame is then 
obtained by a volumetric sampling process that simulates ray casting. More in detail
the pixel color $x_{ut}\in \mathbb{R}^3$ of pixel $u \in \Omega = \{0,...,H-1\}\times\{0,...,W-1\}$ is obtained
by averaging the color of the 3D points along the ray $g_t r_k$ weighted by the probability
that a photon emanates from the point and reaches the camera.

A neural network, $(\sigma_k^b,c_k^b) = MLP^b(g_t r_k,d_t)$ retrieve the density 
$\sigma_k^b \in\mathbb{R}_{+}$ and the color $c_k^b \in \mathbb{R}^{3}$ of each point $g_t r_k$,
where the superscript $b$ refers to the fact that we are dealing with the static background and
$d_t$ is the unit-norm viewing direction.

A photon while travelling along the segment $(r_k,r_{k+1})$ is transmitted with probability
$T_k^b = e^{-\delta_k \sigma_k^b}$ where the quantity $\delta_k = |r_{k+1} - r_k|$ is the length of 
the segment. This definition allows to calculate the probability of transmission across several segments
as the product of the individual transmission probabilities. The color of pixel u can thus be written as:
\begin{equation}
    x_{ut} = f_u(g_t) = \sum_{k=0}^{M} v_k (1-T_k^b)c_k^b, \quad v_k = \prod_{q=0}^{k-1} T_q^b
\end{equation}
The model is trained by minimizing the reconstruction error $\Vert x - f(g_t)\Vert $.

\subsubsection{Dynamic Components}
To econstruct egocentric videos anyway we will deal with moving objects, so we can
not neglect the foreground $F_t$. To capture this layer they propose to build on top 
of the background $MLP^b$ a foreground-specific 
$MLP^f$,$(\sigma_k^f,c_k^f,\beta_k^f) = MLP^f(g_t r_k, z_t^f)$. Its outputs are a 'foreground'
occupancy $\sigma^f$ and color $c^f$. It also predict an uncertainty score $\beta_k^f$
whose role is discussed in the next paragraphs. The variable $z_t^f$ is introduced 
to capture the properties of the foreground that change over time.

The color $x_{ut}$ of a pixel u is obtained by the composition of both background and
foreground, so $\mathcal{S} = {b,f}$:
\begin{equation}\label{eq:2}
    x_{ut} = f_{u}(g_t,z_t) = \sum_{k=0}^{M} v_k(\sum_{p \in \mathcal{S}} w^p(T_k)c_k^p)
    \quad where \quad v_k = \prod_{q=0}^{k-1}\prod_{p \in \mathcal{S}} T_q^p
\end{equation}
The factor $v_k$ reqquires a photon to be transmitted from the camera to point $r_k$
through different materials. The weights $w^p(T_k)$ mix the colors based on points densities.
As done in NeRF-W~\cite{nerfw}:
\begin{equation}\label{eq:weights}
    w^p(T_k) = 1 - T_k^p \in [0,1]
\end{equation}

\paragraph{Smooth Dynamics} The model can now be optimized by minimizing the loss across
all input frames:
\begin{equation}\label{eq:timeNdiff}
    \min_{f,z_1,...z_T} \frac{1}{T |\Omega|} \sum_{t=1}^{T} \sum_{u\in\Omega}\Vert x_t-f(g_t,z_t)\Vert^2 
\end{equation}

However there is a problem with the characteristics of foreground, because, even if 
dynamic, it does not change at every frame in fact most of objects spend most of 
their time rigidly attached to the background. This make the dependecy on independent
frame-specific codes $z_t$ almost useless. They came up with the idea of 
replacing it with a low-rank expansion of the trajectory of sates, taking 
$z_t = B(t) \Gamma$ where $B(t) \in \mathbb{R}^p$ is a fixed basis and the motion
$\Gamma \in \mathbb{R}^{P \times D}$ are coefficients such that $P << T$.
In particular $B(t)=[1,t,sin 2 \pi t, cos 2 1pi t, sin 4 \pi t, cos 4 \pi t,...]$

\textbf{Improved Geometry: capturing the actor}
The foreground layer can be divided again in objects manipulated by the actor,
 that can move sporadically, and the actor, which instead moves continually.
 To detect the actor a third MLP is integrated to the previous model.
 The actor's MLP is similar to the foreground MLP: $(\sigma_k^a,c_k^a.\beta_k^a) = MLP^a(r_k,z_t^a)$.
 The main difference is that this time the 3D point $r_k$ is expressed with respect 
 to the camera, and not to the world ($g_t r_k$). This follows the physical properties
 of the recording stage, the camera is in fact attached to the actor head, making his
 body parts almost always present in front of the camera therefore it shows a reduced
 variability in the reference frame of the camera. On the contrary, the background and 
 foreground is invariant with respect to the world reference system.

 \textbf{Improved Color Mixing}
A principled mixing model is proposed to replace Equation~\ref{eq:weights}. It is obtained
by dividing the segment $\delta_k$ in Pn sub-segments, alternating between the P different
materials($P = \vert \mathcal{S}\vert$, $e.g. P = 3$ if all three layers are considered).
They show that the probability that a photon is absorbed in a subsegment of material p
is given by:
\begin{equation}
    w^p(T_k) = \frac{\sigma_k^p}{\sum_{q=1}^P \sigma_k^q} (1-\prod_{q=1}^{P} T_k^q)
\end{equation}
where the first term represent the probability that a given material p is responsible
for the absorption. The latter one instead is the probability that the photon is absorbed by all 
the materials involved.

\subsubsection{Uncertainty and regularization}
\textbf{Uncertainty.} Here we clarify the role of the previously announced $\beta_k^p$ variable.
It is predicted from each MLPs ($\beta_k^b = 0$ for the background) and represent the uncertainty
of the color associated to each 3D point along the ray $r_k$ for each layer p as pseudo-standard
deviations(StDs). As reported in~\cite{ndiff_17}, the StD of a rendered color $x_{ut}$ is given
as the sum of all the StDs for each material: $\beta_{ut} = \sum_{p} \beta_{u t}^p$ where 
$\beta_{u t}^p$ is obtained from Equation~\ref{eq:2} by replacing $c_k^p$ with $\beta_k^p$.
Now we can introduce a probabilistic loss:
\begin{equation}
    \mathcal{L}_{prob}(f,z_t \vert x_t, g_t,u) = \frac{\Vert x_{ut} - f_u(g_t,z_t)\Vert^2}{2 \beta_{ut}^2}
\end{equation}
\textbf{Sparsity.} The occupancy of the foreground and actor components is further penalized
by using:

\begin{equation}
    \mathcal{L}_{sparse}(f,z_t \vert x_t, g_t,u) = \sum_{p=1}^{P}\sum_{k=0}^{M} \sigma_k^p
\end{equation}
\textbf{Training Loss.} Finally, the loss used for training the model is:
\begin{equation}
    \mathcal{L} = \mathcal{L}_{prob}+\lambda \mathcal{L}_{sparse}
\end{equation}
where $\lambda>0$ is a weight set to 0.01.

\subsection{EPIC-Diff benchmark}
The goal of this paper was to identify any 'detachable' object, namely a onject that moves independently
from the camera. An extension of EPIC-KITCHENS~\cite{EPICKITCHENS} was introduced to give an evaluation to their method.

\paragraph{Data Selection.}10 videos sequences,or also called \textit{scene}, were selected from EPIC-Kitchen, each lasting 14 minutes
on average. Then 1000 frames where sampled from each and fed to COLMAP~\cite{colmap} to obtain
camera reconstructions. The scenes followed these constraints. $1)$The videos
should contain different viewpoints and multiple manipulated objects.$2)$ COLMAP should reconstruct
the sequence with at least 600 frames. In the end the obtain 10 sequence with an average of 900 frames.

\paragraph{Data annotation.}
Being an unsupervised algorithm, the only data annotations that collected were for testing and 
validation. They uniformly hold out 56 frames on average for validation(for setting parameters)
and for testing. The latter were manually annotated with segmentation pixelwise binary masks to assess static/dynamic components.
The test frames are not used for training such that they can be used also to also evaluate 
novel-view synthesis. In total they obtained 560 manual image-level segmentation masks. An example can
be see in Figure~\ref{fig:exMasks}
\begin{figure}[t]
    \centering
    \includegraphics[width=1\linewidth]{images/relatedWorks/Neural/ExampleMasks.png} 
    \caption{Examples of frames with their corresponding manually segmented binary pixelwise masks.}\label{fig:exMasks}
\end{figure}

\paragraph{Evaluation.}The task evaluated is background subtraction. To accomplish this they
used standard segmentation metrics: each frames is demposed in its pixels and a mask is 
extracted from the predicted frame. Then it is compared with its ground-truth calculating 
average precision(AP). Each AP is then averaged with every frame and scene to obtain mean 
average precision (mAP).

For novel view synthesis instead PSNR is used. Specifically they provided,using the 
ground-truth masks, the PSNR of the static and of the moving parts.\vspace{0.4cm}

\paragraph{Results.} The experiments are based on a PyTorch implementation of the model
that has been published on \url{https://github.com/dichotomies/NeuralDiff}, more details 
can be found in the paper. 

The Baselines compared are:

$(1)$\textbf{NeRF}~\cite{nerf} which uses a single stream to predict static scenes.

$(2)$\textbf{NeRF-BF} trains two NeRF models in parallel, one for the Background (B) and the
other for the foreground (F). The latter stream is conditioned on time by passing a
positional-encoded version of the time variable, that in this case corresponds with 
the frame number. This differ from NeuralDiff time encoding for Equation~\ref{eq:timeNdiff}.

$(3)$\textbf{NeRF-W}~\cite{ndiff_17} also contains two interlinked background and foreground
streams. Anyway it was designed to deal with image collection, so it struggles to 
adapt to videos. Some adjustments were done to adapt it to this task, more details
are reported in the paper.

%In Table~\ref{tab:ndiff_res} are reported the results for the various model previously
%presented for the evaluation of their capacity to discover and segment 3D objects
%but also reconstruct dynamic scenes. It is worth noting that NeuralDiff largely improve
%the results with respect to other methods. Also the temporal information confirms to 
%be fundamental for improving the performance over NeRF and NeRF-W. As a third observation
%each update proposed upon the vanilla NeuralDiff proveed to be succesful. Hoewever
%we have to remind that these metrics do not reflect the ability of the full model to 
%separate foreground into objects and actor, since they are merged together in 
%the annotated masks. This ability can be seen in Figure~\ref{fig:ndiff_1}.
%%\begin{table}
    
\centering
\begin{tabular}{lcccc}
    \hline Method & mAP & PSNR & PSNR $_b$ & PSNR $_f$ \\
    \hline NeRF ~\cite{nerf} & 47.8 & 20.9 & 22.8 & 17.6 \\
    NeRF-W ~\cite{nerfw} & 59.2 & 23.2 & 26.4 & 18.9 \\
    NeRF-BF & 64.4 & 23.8 & 26.8 & 19.6 \\
    NeuralDiff  & 66.7 & 24.0 & 27.2 & 19.8 \\
    NeuralDiff+A  & $\mathbf{6 9 . 1}$ & 24.1 & $\mathbf{2 7 . 3}$ & 19.9 \\
    NeuralDiff+C  & 67.4 & 24.1 & 27.2 & 19.9 \\
    NeuralDiff+C+A  & 67.8 & $\mathbf{2 4 . 2}$ & $\mathbf{2 7 . 3}$ & $\mathbf{2 0 . 0}$ \\
    \hline
\end{tabular}
\caption{Results} \label{tab:ndiff_res}
\end{table}

\textbf{Qualitative Results.} We report just some qualitative results to grasp the effective 
results obtained from their method. In Figure~\ref{fig:ndiff_qualitative} the output of the compared
methods is reported for three different scenes.It can be seen that NeuralDiff produce less ghosting artifacts,
captures most moving objects and shows more details, especially the \textit{upgraded} version.
In particular: NeRF struggles with the dynamic components and create blurry
reconstructions; NeRF-W obtains sharper static images but still struggle with
dynamic regions; NeuralDiff produces sharper results but the \textit{upgraded}
version capture more details, such as the arms in the efirst scene or the spoon in the
second.
\begin{figure}[t]
    \centering
    \includegraphics[width=1\linewidth]{images/relatedWorks/Neural/qualRes.png} 
    \caption{Three Scenes reconstruction from NeRF, NeRF-W and NeuralDiff, NeuralDiff+C+A.
     It can be seen that NeuralDiff produce less ghosting artifacts, captures most moving
     objects and shows more details, especially the \textit{upgraded} version.}\label{fig:ndiff_qualitative}
\end{figure}
Another qualitative result is shown in Figure~\ref{fig:qual2}.
\begin{figure}[t]
    \centering
    \includegraphics[width=1\linewidth]{images/relatedWorks/Neural/qual2.png} 
    \caption{\textbf{Segmentation Masks.}Here the masks for which NeuralDiff scored best(top 4 rows)
    and worst (last 2 rows) are reported.}\label{fig:qual2}
\end{figure}
In the best segmentation case for NeuralDiff+A, NeRF-WW create noisy results, where 
the plate, the pasta colander and the actor's body are barely captured. NeuralDiff 
instead manage to capture all moving objects and more body parts but misclassifies 
floor as foreground. NeuralDiff-A better estimates the shape of the actor's body(second and third row)
and correctly recognize the floor as static.

For the failure case instead, NeuralDiff still beats NeRF-W. In detail, NeRF-W fail 
in recognizing any foreground object. NeuralDiff performs a little better by discriminating
some foreground objects but still including some of the static background as foreground
(e.g. Last row, part of the sink is reported as foreground).