\chapter{Neural Rendering}\label{sec:Neural}
From CVPR 2020 tutorial~\cite{CVPRtutorial} Neural Rendering is defined as \textit{"Neural rendering
 is a new class of deep image and video generation approaches that 
 enable explicit or implicit control of scene properties such as 
 illumination, camera parameters, pose, geometry, appearance, and 
 semantic structure. It combines generative machine learning techniques 
 with physical knowledge from computer graphics to obtain controllable 
 and photo-realistic outputs."}.

\section{NeRF:Representing Scenes as Neural Radiance Fields for View Synthesis}
With NeRF~\cite{nerf} the authors introduced a new state-of-the-art model
for synthesizing novel views of complex scenes by just using a set of sparse
images and their relative positions.
\subsection{Intro?}
With this work they deal with the novel view synthesis problem of complex optimization
by modeling the scene representation with a fully-connected neural network(MLP) without any
convolutional layer, this is called \textit{neural radiance field}(NeRF). The network
take as input the position of a point $(x,y,z)$ and the direction $(\theta,\phi)$ from which we are looking it
and gives as results the color$(r,g,b)$ and density $(\sigma)$ of that point. 

To render a NeRF from a viewpoint one can:$1)$ march camera rays in the 
scene and sample some points from them $2)$ use those points as input to the neural
network to produce an output set of colors and densities, and $3)$ use volume 
rendering techniques to obtain a final 2D image. All the previous steps
are differentiable such tat we can use gradient descent to optimize the model
by minimizing the error between each image and the corresponding prediction.
Repeating this process from multiple viewpoints encourages the network to 
grasp the 3D scene representation. In Figure~\ref{fig:batteria} is reported an example
,presented in the paper,of the main steps.
\begin{figure}
    \centering
    \includegraphics[width=1\linewidth]{images/relatedWorks/batteria.png} 
    \caption{\textbf{NeRF.}Optimization of a continous 5D neural radiance field
    representation(volume density and view-dependent color at any continuous location)
    of a scene from a seet of input images. The 2D novel views are obtained thanks
    to classic volume rendering techniques. Here in this example, given 100
    images acquired from different viewpoints, they sample two novel views.}\label{fig:batteria}
\end{figure}

Anyway this approch is not enough for a complex scene, in fact the optimization 
does not converge to a sufficiently high-resolution representation. This problem
is solved by trasforming the input coordinates with a positional encoding
that enables the network to represent higher frequency functions.
Summing up, their contribution consists in:
\begin{itemize}
    \item An approach for representinc continous scenes as a 5D neural radiance fields.
    \item A differentiable rendering pipeline based on classical volume rendering
    techniques,used to optimize the scene representation via input images.
    \item A positional encoding to map each input 5D coordinate into a higher dimensional
    space which allows to represent high-frequency scene content.
\end{itemize}
\subsection{Related works}
At time of writing a promising direction in computer vision
consisted in encoding objects and scenes in the weights of an MLP. Hower this
method still struggled compared to other methods based on discrete representations
like traingle meshes or voxel grids. With their work the writers of the paper 
managed to obtain state-of-the-art results, denying past disbelief in continous 
volumetric representation.

\textbf{Neural 3D shape representations.} Implicit continous 3D shapes 
representation as level sets had been investigated by~\cite{nerf15,nerf32}, optimizing
signed distance functions or occupancy fields in~\cite{nerf11,nerf27}.
These methods were limited by the necessity of a 3D ground truth, which is 
rarely available. Subsequent works reformulated the problem using differentiable
rendering functions that allowed to use just 2D images as ground truth. Two were the main methods:
~\cite{nerf29} represents surfaces as 3D occupancy fields;~\cite{nerf42}use a 
representation that outputs a feature vector and RGB color at each continous 

coordinate, proposing a differentiable rendering function consisting of a RNN(Recurrent neural network).

Any way these methods have been limited to simple shapes with low complexity,
resulting in oversmoothed renderings. The authors strategy of optimizing 
the networks will reveal themself succesful.

\textbf{View synthesis and image-based rendering.} Photorealistic novel views 
can be reconstructed if a dense sampling of images have been provided by 
simple light field sample interpolation techniques~\cite{nerf21,nerf5}.
If sparser images are available, some methods have been introduced that 
rely on extracting traditional geometry and appearance representations
from observed images. A popular class of approached uses mesh-based representations
of scene with diffuse~\cite{nerf48} or view-dependent~\cite{nerf2} appearance. Also there are differentiable
rasterizers~\cite{nerf4,nerf10,nerf23} that can directly optimize mesh recontruction to reproduce
a set of input images using gradient descent. However due to local minima or 
poor conditioning of the loss landscape, the optimization is usually difficult.

High-quality photorealistic view synthesis is also performed by volumetric representations,
from a set of given RGB images. Volumetric methods can represent complex shapes and materials
,well suited for gradient-based optimization. First works~\cite{nerf19} used observed images to directly predict color voxel grids.
Later~\cite{nerf9,nerf13,nerf17} used deep networks to color a sampled volumetric region from some given images.
Other works tried using a mix of convolutional networks and sampled voxel grids.

Anyway these methods, being based on discrete sampling of the voxel grid, are restricted to low resolution due to time 
and computational power. With this paper instead a \textit{continous} representation was proposed, reducing 
the memory requirements and producing higher quality renderings.

\subsection{Implementation}
A continous 3D scene is represented as a 5D vector-valued function whose input is a 3D coordinate 
$\textbf{x}=(x,y,z)$ and 2D viewing direction $\textbf{d}=(\theta,\phi)$, and whose output is an emited color 
$\textbf{c} = (r,g,b)$ and volume density $\sigma$. In practice the scene is represented by an MLP network
$F_{\Theta}:(\textbf{x},\textbf{d}) \rightarrow (\textbf{c},\sigma)$, where $\Theta$ are the weights of the network.

The representation is encouraged to be multiview consistent by restricting the network to predict the 
volume density $\sigma$ as a function of just the location \textbf{x}, while the color \textbf{c} is a
function of both the input, location and direction. To do this the MLP $F_{\Theta}$ first processes the 
input \textbf{x} with 8 fully-connected layers (using ReLU activation functions and 256 channels per layer),
and outputs $\sigma$ and a 256-dimensional feature vector. The scheme is summed up in Figure~\ref{fig:mlp}.
\begin{figure}
    \centering
    \includegraphics[width=0.8\linewidth]{images/relatedWorks/Neural/NeRF_mlp.png} 
    \caption{\textbf{$F_{\Theta}$ Scheme.} The input position \textbf{x} pass through 8 Fully connected
    (FC) layers of 256-channels. Each FC layer is followed by a ReLU activation function. This intermediate
    result is then concatenated with the input direction (\textbf{d}) and fed to
    one last FC with 128 channels that feeds its output to a ReLU function. The output of the ReLU
    are the color \textbf{c} and the volume density ($\sigma$).}\label{fig:mlp}
\end{figure}

The effects of the direction in input can be seen in Figure~\ref{fig:lego}.
\begin{figure}
    \centering
    \includegraphics[width=1\linewidth]{images/relatedWorks/lego.png} 
    \caption{Here are reported the results obtained with different strategies, as written underneath each image.
     In particular removing view dependence prevents the model from recreating the specular 
     reflection on the bulldozer tread. Removing the Positional encoding instead we 
     obtain a blurred image, meaning that high frequencies are not captured nor represented.}\label{fig:lego}
\end{figure}

\subsubsection{Volume Rendering with Radiance Fields}
The implicit representation of the scene relies on the volume densities and on the color
of every point in that scene. The color of any ray passing through the scene is rendered using
principles from classical volume rendering~\cite{nerf16}. \textbf{The volume density $\sigma(\textbf{x})$
can be interpreted as the differential probability of a ray terminating at an infinitesimal
particle at location \textbf{x}.} While the expected color C(\textbf{r}) of camera ray 
\textbf{r}(t) = \textbf{o}+ t\textbf{d} with near and far bounds $t_n$ and $t_f$ is:
\begin{equation}
    C(\textbf{r}) = \int_{t_n}^{t_f} T(t)\sigma(\textbf{r}(t))\textbf{c}(\textbf{\textbf{r}(t),\textbf{d}}) \,dt 
\end{equation} 
where $T(t)$ denotes the accumulated transmittance along the ray from $t_n$ to $t$,i.e:
the probability that the ray travels from $t_n$ to $t$ without hitting any other particle.Namely:
\begin{equation}
    T(t) = exp(-\int_{t_n}^{t}\sigma(\textbf{r}(s))\,ds)
\end{equation}
To obtain a new view in our neural radiance field, we should estimate the $C(\textbf{r})$
function for each ray passing through each pixel of the focal plane(see Figure~\ref{fig:rt}).
\begin{figure}
    \centering
    \includegraphics[width=0.5\linewidth]{images/relatedWorks/Neural/ray_tracing.png} 
    \caption{Example of rays passing through an image plane of size 3x3 pixels.}\label{fig:rt}
\end{figure}
This integral is approximated using a numerical method known as \textit{quadrature}.Typically deterministic 
quadrature is used for rendering discretized voxel grids, but in our case it would limit our model's resolution
since the network would only be queried at a fixed discrete set of locations. To solve this problem a
\textit{stratified} sampling approach has been used, where each interval $[t_n,t_f]$ has been partitioned
into N evenly-spaced bins, from which a random sample is then uniformly extracted. Namely:
\begin{equation}
    t_i  \thicksim \mathcal{U} [t_n + \frac{i-1}{N}(t_f-t_n), t_n + \frac{i}{N}(t_f-t_n)]
\end{equation}
In this way, even if we are using a discrete set of samples, using stratified sampling allows us to represent a continuous scene 
representation because the network is evaluated at continuous positions during the training phase.
Following the quadrature rule discussed in~\cite{nerf26} they used the samples to estimate $C(\textbf{r})$:
\begin{equation}\label{eq:neural_C}
    \widehat{C}(\textbf{r}) = \sum_{i=1}^N T_i (1-exp(-\sigma_i \delta_i))\textbf{c}_i, \quad where \quad T_i = exp(-\sum_{j=1}^{i-1}\sigma_j\delta_j)
\end{equation}
where $\delta_i =t_(i+1)$ is the distance between adjacent samples. It can be noticed that the function that approximate $C(\textbf{r})$ is differentiable and can be 
reduced to traditional \textit{alpha compositing}\footnote{\textit{Alpha compositing} is a digital imaging technique used to combine multiple layers of images or graphics with transparency, known as alpha channels.}
with alpha values being $\alpha_i = 1-exp(-\sigma_i \delta_i)$.

\subsubsection{Optimizing a Neural Radiance Field}
In the previous section the core components of a Neural Radiance Field were presented, anyway these partss alone are not able to 
achieve state-of-the-art results. To improve the quality of the representation two improvements were found succesful:
\begin{itemize}
    \item \textbf{Positional Encoding} of the input coordinates, to encourage the representation of high-frequencies.
    \item \textbf{Hierarchical Sampling} procedure that allows to efficiently sample high-frequency representation.
\end{itemize} 

\textbf{Positional Encoding.} After having found that the model $F_\Theta$ performed poorly operating solely on $xyz\theta\phi$ input, in accordance
to the work by Rahaman \textit{et al.}~\cite{nerf35} that states that deep networks are biased towards
learning low-frequency functions; they encode the inputs to a higher dimensional space using high frequency functions. 

The new model $F_\Theta$ can thus be represented as a combinatino of functions $F_\Theta = F_\Theta'\circ \gamma$, where $\gamma$ is a function from $\mathbb{R}$ 
to a higher dimensional space $\mathbb{R}^{2L}$, and  $F_\Theta'$ is still the basic block introduced in the previous sections.
Formally the function used for the encoding part is:
\begin{equation}
    \gamma(p) = (sin(2^0 \pi p), cos(2^0 \pi p), ...,sin(2^L-1 \pi p), cos(2^L-1 \pi p))
\end{equation}
In particular $\gamma(\dot)$ is applied  separately to each component of \textbf{x} and \textbf{d}, after these values are normalized to lie in $[-1,1]$.
Reporting the paper results, they found out that good values of L were: 10 for $\gamma(\textbf{x})$ and 4 for $\gamma(\textbf{d})$.

\textbf{Hierarchical Sampling.} Stratified sampling allows to cover continuous
regions but anyway is still inefficient: free space and occluded regions that do not 
contribute to the rendered image are still sampled repeatedly. To solve this 
problem the authors proposed a \textit{hierarchical} representation which increased
rendering efficiency by distributing samples proportionally to their expected 
effect on the final rendering.

This solution consists in simultaneously optimize two networks: one "coarse" and
one "fine". The first step expect to sample a set of $N_c$ points using stratified 
sampling and feed those to the coarse model. Once this first partial result has 
been obtained a more informed sampling of points is prooduced. The coarse sampling 
infact allows to get a rough idea of which parts of the volume are the most relevant.
To do this they rewrite the alpha composited color from the coarse model $\hat{C}_c(\textbf{r})$
in Eq.~\ref{eq:neural_C} as a weighted sum of all sampled colors $c_i$ along the ray:
\begin{equation}
    \hat{C}_c(\textbf{r}) = \sum_{i=1}^{N_c} w_i c_i, \quad \quad w_i = T_i(1-exp(-\sigma_i \delta_i))
\end{equation}
By normalizing the weights as $\hat{w}_i = \frac{w_i}{\sum_{j=1}^{N_c}w_j}$ we can produce
a piecewise-constant probability density function along the ray as seen in Figure~\ref{fig:pdf_ray}.
\begin{figure}
    \centering
    \includegraphics[width=0.8\linewidth]{images/relatedWorks/Neural/pdf_rays.png} 
    \caption{PDF of normalized coarse weights $\hat{w}_i$ along a ray with $N_c$ samples.}\label{fig:pdf_ray}
\end{figure}
The next $N_f$ samples are extraced from this distribution and then, combined to the previous $N_c$ samples,
fed to the "fine" network. The final rendered color $\hat{C}_f(\textbf{r})$ is obtained using
Eq.~\ref{eq:neural_C} with $N_c+N_f$. This procedure revealed succesful in obtaining more
samples from regions we expect to contain visible content.

\subsubsection{Actual Implementation}
To optimize a scene RGB frames are required with the corresponding camera poses and intrinsic
parameters(for synthetic data these are easily retrievable from the scene model; while for
real images COLMAP was used to extract them).
At each iteration a batch of rays from the set of all pixels of all images of the dataset
is extracted and following the sampling procedure previosuly described the actual color 
is predicted for both the "coarse" and the "fine" model. The loss is computed as the 
total squared error betwwen the rendered and true pixel colors for the two models:
\begin{equation}
    \mathcal{L} = \sum_{r\in\mathcal{R} }[\left\lVert {\hat{C}_c(\textbf{r})-C(\textbf{r})} \right\rVert_2^2+\left\lVert {\hat{C}_f(\textbf{r})-C(\textbf{r})} \right\rVert_2^2]
\end{equation}

where $\mathcal{R}$ is the set of rays of each batch, $C(\textbf{r})$ is the RGB color
ground truth and $\hat{C}_c(\textbf{r})$,$\hat{C}_f(\textbf{r})$ are the predicted color
for the "coarse" and the "fine" model for ray \textbf{r}.

As for some more specific details, they used a batch size of 4086 rays, each
sampled at $N_c=64$ points for the "coarse" model and $N_f=128$ for the "fine"
model. They used the Adam optimizer with a learning rate beginning at $5x10^{-4}$
and decays exponentially to $5x10^-5$ over the course of optimization. Others parameters
of the optimizer where left at default values: $\beta_1 = 0.9,\beta_2=0.999$ and $\epsilon=10^{-7}$.
Each scene took them around 100-300l iterations to converge on NVIDIA V100 GPU(about 1-2 days).
Their tensorflow implementation is provided on \url{https://github.com/bmild/nerf}.

\subsection{Results}
To validate their results they generated and aggregated some datasets.
The main distinction is given by synthetic scenes, denoted as\textit{ Diffuse 
Synthetic 360$\degree$}~\cite{deepvoxels} and \textit{Realistic Synthetic 360$\degree$}, and 
realistic scenes,\textit{Real Forward Facing}. More in detail:
\begin{itemize}
    \item \textit{ Diffuse Synthetic 360$\degree$}: contains four Lambertian objects 
    with simple geometry. Each object is rendered at 512x512 pixels from viewpoints taken
    from the upper hemisphere.
    \item \textit{ Realistic Synthetic 360$\degree$}: new dataset that the authors introduce
    containing images of eight objects that exhibit complicated geometry and realistic non-Lambertian\footnote{Lambertian reflectance is the property that defines an ideal "matte" or diffusely reflecting surface. The apparent brightness of a Lambertian surface to an observer is the same regardless of the observer's angle of view~\cite{lambertian}.}
    materials. Six scenes were sampled from the upper hemisphere while the remaining two
    are rendered from a full sphere. For each scene 100 views are captured for training and 
    200 for testing, all at 800x800 pixels.
    \item \textit{Real Forward Facing}: consists of 8 scenes captured with a handheld cellphone
    (5 taken from the LLFF~\cite{LLFF} paper and 3 captured by them), captured with 20 to 62 images,
    holding out $1/8$ of these for the test set. All images have a resolution of 1008x756 pixels.
\end{itemize}

The algorithm which have been compared were the top-performing techniques for view synthesis and 
are the following:
\begin{itemize}
    \item \textbf{Neural Volumes (NV)}~\cite{neuralvol} synthesizes novel views of objects
    lying in a confined space with a distinct background(which must be captured alone).
    The model is based on a 3D convolutional network to predict a discretized RGB$\alpha$ voxel grid.
    \item \textbf{Scene Representation Networks (SRN)}~\cite{srn} represent a continuous scene
    as an opaque surface, implicitly defined by an MLP that maps spatial coordinates to a
    feature vector. The color is then obtained by training a RNN along the ray.
    \item \textbf{Local Light Field Fusion (LLFF)}~\cite{LLFF} is designed for producing photorealistic
    novel views for well-sampled forward facing scenes.
\end{itemize}
Quantitative results are reported from the paper in Table~\ref{tab:nerf_res}.
The metrics used were: PSNR,SSIM and LPIPS (see Section~\ref{sec:Metrics} for more details).
We can see that the method introduced outperformed past works in both realistic and 
synthetic scenarios.

\begin{table}

\resizebox{\textwidth}{!}{%
\begin{tabular}{l|ccc|ccc|ccc} 
    & \multicolumn{3}{|c|}{ Diffuse Synthetic $360^{\circ}$} & \multicolumn{3}{c|}{ Realistic Synthetic 360 } & \multicolumn{3}{c}{ Real Forward-Facing  } \\
    Method & PSNR $\uparrow$ & SSIM $\uparrow$ & LPIPS $\downarrow$ & PSNR $\uparrow$ & SSIM $\uparrow$ & LPIPS $\downarrow$ & PSNR $\uparrow$ & SSIM $\uparrow$ & LPIPS $\downarrow$ \\
    \hline SRN & 33.20 & 0.963 & 0.073 & 22.26 & 0.846 & 0.170 & 22.84 & 0.668 & 0.378 \\
    NV  & 29.62 & 0.929 & 0.099 & 26.05 & 0.893 & 0.160 & - & - & - \\
    LLFF  & 34.38 & 0.985 & 0.048 & 24.88 & 0.911 & 0.114 & 24.13 & 0.798 & $\mathbf{0 . 2 1 2}$ \\
    NeRF & $\mathbf{4 0 . 1 5}$ & $\mathbf{0 . 9 9 1}$ & $\mathbf{0 . 0 2 3}$ & $\mathbf{3 1 . 0 1}$ & $\mathbf{0 . 9 4 7}$ & $\mathbf{0 . 0 8 1}$ & $\mathbf{2 6 . 5 0}$ & $\mathbf{0 . 8 1 1}$ & 0.250
\end{tabular}}
\caption{\textbf{Quantitative results} In all datasets, for all metrics,except for the LPIPS,
the NeRF method outperforms old methods. }\label{tab:nerf_res}
\end{table}

A qualitative idea instead is given by Figure~\ref{fig:blender2} and ~\ref{fig:blender1}.
\begin{figure}
    \centering
    \includegraphics[width=1\linewidth]{images/relatedWorks/Neural/blender2.png} 
    \caption{Comparison on test images from the newly introduced
    synthetic dataset. NeRF method is able to recover fine details in both geometry
    and appearance. LLFF exhibits some artifacts on the microphone and some ghosting
    artifact in the other scenes. SRN produces distorted and blurry rendering for every
    scene. Neural Volumesstruggle capturing details we can see from
    the ship reeconstruction.}\label{fig:blender2}
\end{figure}

\begin{figure}
    \centering
    \includegraphics[width=1\linewidth]{images/relatedWorks/Neural/blender1.png} 
    \caption{Comparison on the test set of the real images.
    As expected LLFF is performing pretty well being projected
    for this specific use case(forward-facing captures of real scenes).
    Anyway NeRF is able to represent fine geometry more consistently across
    rendered views than LLFF as we can see in Fern's and in T-rex. NeRF
    is also able to reproduce partially occluded scene as in the second row.
    SRN instead completely fail to represent any high-frequency content.}\label{fig:blender1}
\end{figure}
An additional validation of their design choices is given by an ablation
study on the various parts that have been discussed in the implementation part.
In particular the result of the study is reported in Table~\ref{tab:nerf_ablation}.
\begin{table}

    \resizebox{\textwidth}{!}{%
\begin{tabular}{l|cccc|ccc} 
    & Input & \#Im. & $L$ & $\left(N_c, N_f\right)$ & PSNR $\uparrow$ & SSIM $\uparrow$ & LPIPS $\downarrow$ \\
    \hline 1) No PE, VD, H & $x y z$ & 100 & - & $(256,-)$ & 26.67 & 0.906 & 0.136 \\
    2) No Pos. Encoding & $x y z \theta \phi$ & 100 & - & $(64,128)$ & 28.77 & 0.924 & 0.108 \\
    3) No View Dependence & $x y z$ & 100 & 10 & $(64,128)$ & 27.66 & 0.925 & 0.117 \\
    4) No Hierarchical & $x y z \theta \phi$ & 100 & 10 & $(256,-)$ & 30.06 & 0.938 & 0.109 \\
    \hline 5) Far Fewer Images & $x y z \theta \phi$ & 25 & 10 & $(64,128)$ & 27.78 & 0.925 & 0.107 \\
    6) Fewer Images & $x y z \theta \phi$ & 50 & 10 & $(64,128)$ & 29.79 & 0.940 & 0.096 \\
    \hline 7) Fewer Frequencies & $x y z \theta \phi$ & 100 & 5 & $(64,128)$ & 30.59 & 0.944 & 0.088 \\
    8) More Frequencies & $x y z \theta \phi$ & 100 & 15 & $(64,128)$ & 30.81 & 0.946 & 0.096 \\
    \hline 9) Complete Model & $x y z \theta \phi$ & 100 & 10 & $(64,128)$ & $\mathbf{3 1 . 0 1}$ & $\mathbf{0 . 9 4 7}$ & $\mathbf{0 . 0 8 1}$
    \end{tabular}}
    \caption{\textbf{Quantitative results} In all datasets, for all metrics,except for the LPIPS,
the NeRF method outperforms old methods. }\label{tab:nerf_ablation}
\end{table}

\section{NeuralDiff: Segmenting 3D objects that move in egocentric videos}