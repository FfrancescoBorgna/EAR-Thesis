\section{Photogrammetry}
\textit{Photogrammetry is the science and technology of 
obtaining reliable information about physical objects and
the environment through the process of recording,
measuring and interpreting photographic images and patterns 
of electromagnetic radiant imagery and other 
phenomena}\cite{examplewebsite}.

It comprises all techniques concerned with making measurements of
real-world objects features from images.
Its utility range from the measuring of coordinates, quantification
of distances, heights, areas and volumes, preparation
of topographic maps, to generation of digital elevation 
models and orthophotographs. The functioning rely mostly on optics and projective geometry rules. 

\vspace{12pt}

As first assumption we have the modellization of the camera as a simplified
version of itself: the \textit{Pinhole Camera}. As in the first designed
cameras(\textit{camera obscura}), in the Pinhole Camera world's light is 
captured through a pinhole and then projected into the \textit{focal plane}.

\begin{figure}
    \centering
    \includegraphics[width=0.3\linewidth]{images/relatedWorks/Pinhole.png} % Replace "example-image" with your image file name
    \caption{Pinhole Camera}
    \label{fig:pinhole}
  \end{figure}
  
A Pinhole camera is characterized by two collection of parameters:
\begin{itemize}
    \item  \textbf{Extrinsic} parameters: gives us information on location
                                        and rotation in the world.
    \item  \textbf{Intrinscic} parameters: gives us internal property such as:
                                    focal length, field of view, resolution...
\end{itemize}  
These parameters can be rewritten in their corresponding matrices:
\[
  Intrinsic=K= \begin{bmatrix}
    f_{x} & s & c_{x} \\
    0 & f_{y} & c_{y} \\
    0 & 0     & 1     \\
  \end{bmatrix}
\]
where:
\begin{itemize}
    \item $f_{x},f_{y}$ are are the \textit{focal lengths} of the camera in the x and y directions, 
    they are needed to keep the image aspect ratio.
    \item $c_{x},c_{y}$ are the coordinates of the \textit{principal point}
    (the point where the optical axis intersects the image plane).
\end{itemize}

\[
  Extrinsic= \begin{bmatrix}
    \textbf{R}_{3x3} & \textbf{t}_{3x1}  \\
    0_{1x3} & \textbf{1}_{1x1}  \\
  \end{bmatrix}
\]
where:
\begin{itemize}
    \item $\textbf{R}_{3x3}$ is a rotation matrix
    \item $\textbf{t}_{3x1}$ is a translation vector
\end{itemize}
Extrinsic matrix is also known as the 4x4 transformation matrix 
that converts points from the world coordinate system to the camera
coordinate system.

Exploting homogeneous coordinates we can rewrite the image capturing process as the 
following combination of matrices:
\begin{align*}
    \begin{bmatrix}
        u \\
        v \\
        z
      \end{bmatrix}
    &= \begin{bmatrix}
        f_{x} & s & c_{x} & 0 \\
        0 & f_{y} & c_{y} & 0\\
        0 & 0     & 1     & 0\\
      \end{bmatrix} \begin{bmatrix}
        \textbf{R}_{3x3} & \textbf{t}_{3x1}  \\
        0_{1x3} & \textbf{1}_{1x1}  \\
      \end{bmatrix}
      \begin{bmatrix}
        X_w \\
        Y_w \\
        Z_w \\
        1    
      \end{bmatrix}
  \end{align*}




