\subsection{COLMAP}\label{sec:col}
\begin{figure}
    \centering
    \includegraphics[width=1\linewidth]{images/relatedWorks/sparse.png} % Replace "example-image" with your image file name
    \caption{Building Rome in one day}\label{fig:colmap_image}
\end{figure}
Abandoning theory and moving on to practice, let's see COLMAP.

\textit{"COLMAP is a general-purpose, end-to-end image-based 
3D reconstruction pipeline (i.e., Structure-from-Motion (SfM)
 and Multi-View Stereo (MVS)\footnote{Even if they seem similar, these
 two pipelines have different goals. In fact, \textit{SfM} The
  primary goal of SfM is to estimate the 3D structure of 
  a scene and camera poses (positions and orientations) 
  simultaneously from a set of 2D images taken from different
   viewpoints. \textit{MVS},on the other hand, is specifically focused on dense 3D reconstruction. It involves estimating the depth or 3D coordinates 
   for every pixel in the images to create a detailed 3D model with
   meshes and textures.
}) with a graphical and command-line 
 interface. It offers a wide range of features for 
 reconstruction of ordered and unordered image collections. 
 The software runs under Windows, Linux and Mac on regular
  desktop computers or compute servers/clusters. COLMAP is
   licensed under the BSD License"}\cite{colmap}.

The concepts we explained in the theoretical part are not always applicable In
real world, or their results are not quite satisfying. In literature a vastity of authors
tried and succeded in obtaining refined algorithms for specific scenarios. 
Anyway they still lacked a general-purpose method. With COLMAP they managed
to compensate for the lack of generalization. The actual implementation and characteristics
are explained from the authors in \cite{schoenberger2016sfm,schoenberger2016mvs}.

\subsubsection{Usage}
The usage of COLMAP is pretty straight forward. After the creation of a project
folder with a \textit{images} directory containing the images we want to process we 
can simply launch the script reported in Listing (\ref{lst:bash}).
\begin{lstlisting}[style=vscode, caption={Automatic COLMAP Reconstruction}, label={lst:bash}]
#!/bin/bash
#The project folder contains a folder "images" with all images.

DATASET_PATH=/path/to/project
colmap automatic_reconstructor \
    --workspace_path $DATASET_PATH \
    --image_path $DATASET_PATH/images \
    --single_camera 1
    \end{lstlisting}

Anyway the program offers ample freedom to the single usage of the various steps
need for \textit{SfM} or \textit{MVS}. Here  I report the actual pipeline that
I used for the extraction of the pointclouds for our experiments.
\begin{lstlisting}[style=vscode, caption={COLMAP Single Commands}, label={lst:col}]
    #!/bin/bash
    DATASET_PATH="/scratch/fborgna/EPIC_Diff/"

    colmap feature_extractor \
        --database_path $DATASET_PATH/database.db \
        --image_path $DATASET_PATH/images \
        --ImageReader.single_camera 1 \
    
    colmap exhaustive_matcher \
        --database_path $DATASET_PATH/database.db

    mkdir $DATASET_PATH/sparse

    colmap mapper \
        --database_path $DATASET_PATH/database.db \
        --image_path $DATASET_PATH/images \
        --output_path $DATASET_PATH/sparse \
        --camera_model SIMPLE_PINHOLE \

    mkdir $DATASET_PATH/dense

    colmap image_undistorter \
        --image_path $DATASET_PATH/images \
        --input_path $DATASET_PATH/sparse/0 \
        --output_path $DATASET_PATH/dense \
        --output_type COLMAP \
        --max_image_size 2000
    
    colmap patch_match_stereo \
        --workspace_path $DATASET_PATH/dense \
        --workspace_format COLMAP \
        --PatchMatchStereo.geom_consistency true
    
    colmap stereo_fusion \
        --workspace_path $DATASET_PATH/dense \
        --workspace_format COLMAP \
        --input_type geometric \
        --output_path $DATASET_PATH/dense/fused.ply
    
    colmap poisson_mesher \
        --input_path $DATASET_PATH/dense/fused.ply \
        --output_path $DATASET_PATH/dense/meshed-poisson.ply

        \end{lstlisting}
    