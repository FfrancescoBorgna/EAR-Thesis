\part{Our Contribution}
\chapter{Methodology}
Now that the reader has grasped the basics and some nuances of 
photogrammetry and neural rendering, let us see how we exploited 
the presented topics and used them to reach our goal, that is segmenting
Dynamic Objects in 3D from egocentric videos and intelligently filter
samples to remove redundant informations.  
\section{Goals}
\paragraph*{Main Goal}The main goal of this thesis is to Segment Dynamic Objects in 
3D from egocentric videos. Up to now SfM algorithms finds it 
challenging to reconstruct dynamic scenes, resulting in messy
pointclouds, where the same object can appears multiple times 
within the scene. An explicative example is reported in 
Figure~\ref{fig:pizza}. A scene where a pizza is prepared from dough
is reconstructed. It is clearly visible from the frames that on the 
induction cooker is slowly proceeding the making of the pizza but in the
reconstruction it is reported in its integrity. This is due to a lack
of temporal reasoning of structure from motion procedures. Also, above the
pizzas there seem to be multiple pans intersected one with each other.
Obviosuly this is not corresponding to the reality, but moving of few 
centimeters at some time intervals, the process register it as a new 
object each time it is in a new position. Other examples are visible like the 
chopping board and some object on the inductor stove. In this example 
our goal would be to reconstruct the kitchen cleaned of all the object that 
moved during the video recording. This would be of immense potential since any 
environment could be reconstructed from just a video of it and the presence 
of dynamic objects/person would not interfere with it.
\begin{figure}[t]
    \centering
    \includegraphics[width=1\linewidth]{images/method/pizza.png} 
    \caption{Reconstruction of a pizza preparation video. The top and bottom frames give
    us a glance of the action performed during the video while the central
    pointcloud highlighht the problems of SfM in dynamic environments like the 
    superimposition of the same object on itself or the reconstruction of objects
    that are not always present in the scene, e.g. the two pizzas.}\label{fig:pizza}
\end{figure}


\paragraph{Second Goal}The second goal of this thesis is to reduce the computational times of the 
overall pipeline. Being based on the heavy neural network of NeuralDiff, our pipeline
tries to speed it up by finding a filtering method that could reduce the 
number of frames while keeping the same important information of \textit{larger}
samplings. This will be presented in Section~\ref{sec:sampl}. Also, we took a 
simplified version of NeuralDiff, in which the actor,the person who is 
wearing the camera, and the foreground,the dynamic objects, are fused together,
since for our scopes the distinction was not needed and thus we could remove 
one of the three neural radiance fields by accelerating the computations.


\section{Pipelines}
In this section we will present the actual methods that we considered and implemented 
for actually achieve our goals. As we have seen from Related Works, 3D dynamic object 
segmentation is still an evolving field. We took inspiration from various works
to actually come up with some different ideas for actually segmenting 3D dynamic objects.
All the methods revolve around a COLMAP reconstruction and a NeuralDiff renderer. The 
basic block would in fact be NeuralDiff but a reconstruction of the scene with camera intrinsic and 
extrinsic parameters is required to make it work.

\paragraph{Colmap Pipeline}
The first and most trivial idea is to reconstruct the scene with the SfM algorithm,COLMAP,
and then separate the static and dynamics objects using NeuralDiff~\cite{neuraldiff}. Once 
the cleaning has been done we could reconstruct from the cleaned frames the static scene. 
In Figure~\ref{fig:dumb} is reported the pipeline of this first approach.
\begin{figure}[t]
    \centering
    \includegraphics[width=1\linewidth]{images/method/DumbPipeline.pdf} 
    \caption{\textbf{Basic Pipeline.} In the Basic Pipeline a video(represented by the image of a person cooking)
    is subsampled through EF-Sampling, reconstructed 
    via COLMAP, re-sampled on the reconstructed frames and then fed to NeuralDiff. At this stage the frames are
    decomposed in actor,foreground and background(as can be seen in the frames reported below NeuralDiff).
    The Clean reconstruction is obtained by running
    another COLMAP step on the extracted background frames.}\label{fig:dumb}
\end{figure}

\paragraph{Monocular Pipeline}
Anyway this way the previous method is not optimal because we have two COLMAP steps and one of NeuralDiff. We wanted 
to do better. So we tried with a second pipeline that we will call \textit{Monocular-Pipeline}.
This pipeline remove the last COLMAP step by using a pre-trained neural monocular depth estimator.
This last block allows to project any frame from 2D to 3D knowing the camera extrinsic and intrinsic
parameters. So we could train NeuralDiff(reconstruction of the scene included) and the 
project the processed frames into the space. Then to segment dynamic objects we can project
all dynamic frames,actor or foreground or both, in 3D and take all the points of the COLMAP pointcloud
that are at a distance less than a predefined value from the projected ones and classify them as dynamic.
The same could be done with static points. This second pipeline is reported in Figure~\ref{fig:monoPipe}.

\begin{figure}[t]
    \centering
    \includegraphics[width=1\linewidth]{images/method/MonocularPipeline.pdf} 
    \caption{\textbf{Monocular Pipeline.} The Monocular pipeline share the first part with the Basic Pipeline. A
    video is subsampled,reconstructed via COLMAP, subsampled again and fed to Neuraldiff. The difference is that here
    the dynamic layers are projected in 3D and points closer than a distance th are segmented as dynamic.}\label{fig:monoPipe}
\end{figure}

\paragraph{NeuralDiff Pipeline}
The last pipeline instead take advantage of the intrinsic knowledge of the neural renderer, removing the necessity of the 
last step, being that SfM or a Depth extractor. For the definition of a nural renderer we know that it is a neural 
network that take as input a point spatial coordinate with the direction from which it is observed and returns back the 
color and density of that point. This implies that when we are training we are already grasping the 3D structure of the scene,
and any additional step would be unnecessary. The overall pipeline can thus be simplified as in Figure~\ref{fig:ndiffPipe}, and it 
will be referred as \textit{NeuralDiff-Pipeline}. A point is segmented as moving if its corresponding point in the static
scene has a density$<$threshold.

\begin{figure}[t]
    \centering
    \includegraphics[width=1\linewidth]{images/method/NeuralDiffPipeline.pdf} 
    \caption{\textbf{NeuralDiff Pipeline.}In this pipeline we obtain the three motion layers as in the other methods
    but actually the segmentation is performed querying the static neural renderer with the positions
    of the points belonging to the COLMAP reconstruction. Each point is segmented as dynamic 
    if its density is less than a predefined value.}\label{fig:ndiffPipe}
\end{figure}

\paragraph{NeuralCleaner}The last modification that we made was to remove the distinction of the actor and the foreground, combining them into a single stream.
Since it was out of our scope to distinguish between these two layers, we hope to reduce the computational times by removing one of the
three neural streams of NeuralDiff. This pipeline has been named \textit{Neural Cleaner}.


\section{Filtering }\label{sec:sampl}
The filtering method consists 
in seeking temporal windows where frames are overlapped and keeping just
a frame per window. The overlap is computed by estimating homographies
\footnote{A homography is a transformation that maps points from one
 image to corresponding points in another image.
} on matched SIFT~\ref{sec:sfm} features. In Figure~\ref{fig:overlap}some examples 
of overlaps are visualized. 

This technique was proposed in EPIC-Fields~\cite{epic_fields} to obtain accurate 3D reconstructions
from egocentric videos, which present the challenging problems of: dynamic objects,long duration video(\~9min on avg) and
the skewed distributiton of viewpoints, namely the fact that in videos there are phases of slow motion around 
hot-spots (e.g. around the gas stove) alternating to high motion in transition actions(e.g. taking something from the pantry).
The main idea was to remove redundant frames while maintaining enough overlap and temporal coverage to allow an accurate reconstruction.
We will refer to this method as \textit{EF-Sampling}.

We took inspiration from this technique for our sampling method that we called \textit{Intelligent Sampling}. This sampling was 
used to keep only important frame, trying to remove any redundant information for the neural rendering step. The idea to maintain 
important frame is different from the SfM step. Here we thought that a better set of images would have less overlap between itself,
such that all the areas of the scene are covered, or at least the frames are equispaced in the scene if they are too few. In this way there should not
be any blind spot and the environment is captured in its entirety. And this is actually the contrary of the idea of EF-Sampling.
Its position in our pipeline is reported in Figure~\ref{fig:samplPipe}. The COLMAP reconstructed frames are \textit{Intelligently} sampled and fed to NeuralDiff.
\begin{figure}[t]
    \centering
    \includegraphics[scale=0.5]{images/method/samplingPipe.pdf} 
    \caption{Sampling Steps in our pipelines. The initial video is subsampled by
    EF-Sampling and the resulting frames are fed to COLMAP. Onve COLMAP reconstructed the scene 
    these frames are again subsampled via Intelligent Sampling and fed to NeuralDiff.}\label{fig:samplPipe}
\end{figure}

The actual implementation is strictly based on EF-Sampling and since its goal is the contrary of the latter, they share part of the method.
Intelligent sampling in fact given a set of N frames perform a EF-sampling with a overlap threshold such that it gives $ N-N_{desired}$ frames.
These frames are then discarded and the remaining $N_{desired}$ frames are kept.

We also tried a less rigid approach with the \textit{AU-Sampling}. This other method is an hybrid of Intelligent-Sampling and a Uniform sampling(which
is the baseline we tried to improve). In AU-Sampling we perform a Intelligent-Sampling relaxing the $N_{desired}$frames. The $N_{desired}$ frames are then 
uniformly extracted from the 'relaxed' samples. In this way we hoped to keep some overlapping frames by relaxing the threshold and then obtain
equitemporal spaced samples from all the duration of the video. 

\begin{figure}[t]
    \centering
    \includegraphics[width=1\linewidth]{images/method/overlap.png} 
    \caption{Example of overlapping frames(X and Y). The left and central
    examples present the same level of overlap even though the image frames 
    are closer together in the central example, because the 
    number of features shared are the same. The right examples instead
    present a higher level of overlapping due to the bigger number of 
    features.}\label{fig:overlap}
\end{figure}
