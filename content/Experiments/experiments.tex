\chapter{Experiments}
\section{Data selection}
The data selection was dictated by our problem. Indeed evaluating 3D scenes 
reconstructions is not an easy task due to the lack of 3D ground-truths. These
are usually very expensive due to costly hardware scanners but sometimes are 
also not really available, as in our scenario, where we would like a static-dynamic
segmentation. In our case in fact obtaining the static part would mean to 
actually clean the scene from all the possible moving objects which adds 
an extra cost in terms of time, but in other scenarios 'cleaning' the environment
could not be allowed.

For this fact we evaluated our scene reconstructions on a subsample\footnote{\textcolor{red}{Come giustifichiamo l'aver preso non tutte le scene? Per motivi di tempo è accettabile?}}
of the EPIC-KITCHENS extension proposed in NeurlDiff~\cite{neuraldiff}. In this extension
the authors of NeuralDiff added manually pixelwise segmented masks for ten scenes of which we just considered 
P01-01,P03-04,P04-01,P09-02,P16-01,P21-01. We looked also for 
the more recent VISOR~\cite{visor} dataset in which pixel annotations of hands 
and active objects are given but unfortunately their definition of \textit{active}
was not suuitable for our work. They labeled as active any object that is 
included in the current action, so it is common to see as active the sink or 
the gas stove, but in our case they should be considered as static(see Figure~\ref{fig:vis_exp}).
\begin{table}[]
    \centering
    \begin{tabular}{lr}
    \hline
    \textbf{Scene} & \multicolumn{1}{c}{\textbf{Frames}} \\ \hline
    P01-01         & 98935                               \\
    P03-04         & 100251                              \\
    P04-01         & 69292                               \\
    P09-02         & 22187                               \\
    P16-01         & 74592                               \\
    P21-01         & 41583                               \\ \hline
    \end{tabular}
    \caption{Total Frames for each scene}\label{tab:Frames}
    \end{table}
\begin{figure}[H]
    \centering
    \includegraphics[width=1\linewidth]{images/esperimenti/visor_exp.png} 
    \caption{Example of VISOR active annotations, on the left 'wash a knife' include the
    static sink as active; on the right 'pour spice' static gas stove is active}\label{fig:vis_exp}
\end{figure}


\section{COLMAP Reconstruction}
Once we have fixed the data we were working on, we proceeded to do some experiments
on COLMAP reconstructions. In particular we took the scene P01-01 and tried varying 
both the number of frames and their resolution for the reconstruction. In fact most
of the scene have too many frames to handle, which could results in out of memory issues
or at the least worst in a long computational time. Our aim was to find a good compromise
between \textit{quality of reconstruction} and \textit{computational time}.

The first thing we did was to subsample the frames using the same technique 
as reported in Epic FIELDS~\cite{epic_fields} and explained in Section~\ref{sec:sampl}.
We report the results of the COLMAP reeconstructions both quantitatively and qualitatively
in Table~\ref{tab:col_P01_01} and Figure~\ref{fig:colmap_P01}. It is worth noting
few things watching these two references. The first one is that the 
resolution plays an important role in the successfulness of the reconstruction as 
we can see from the first three coloumns of Table~\ref{tab:col_P01_01}. The same split
of frames is reported and the central one at a resolution of 114x64 failed. This 
is due to the feature extractor, that in a high resolution image can retrieve 
informations that instead are lost in low resolution frames. A lack of significant features 
means no matching between images so the reconstruction has very few frames matched.
The second thing is that the higher the frames the better. In fact chances of matching
increases and also we will have more areas of the environment covered, as shown in Figure~\ref{fig:colmap_P01}.
We can see that augmenting the number of frames more parts of the kitchen are revealed,
\textit{e.g.} the round table at the center of the room, the sideboard in front of the sink.
But also some important objects that are visible from the video,like dishes on top 
of the table. This is a keypoint to the development of our pipeline, because we need to 
be sure that actually the scene contains points deriving from the motion of objects.

By considering these results and always keeping in mind the time of computation at our disposal
we opted to feed the next pipeline with around five thousands frames at a resolution 
of 228x128. The pipeline of NeuralDiff in fact is really heavy and working at full resolution
was prohibitive in the number of experiments we could try.

% Please add the following required packages to your document preamble:
% \usepackage[table,xcdraw]{xcolor}
% Beamer presentation requires \usepackage{colortbl} instead of \usepackage[table,xcdraw]{xcolor}
\begin{table}[t]
    \centering
    \resizebox{\textwidth}{!}{
    \begin{tabular}{|l|rrrr|}
    \hline
    \rowcolor[HTML]{9B9B9B} 
    Scenes             & P01\_01\_04 & P01\_01\_06  & P01\_01\_08 & P01\_01\_09  \\ \hline
    Initial Frames    & 1231        & 1487         & 2598        & 5223         \\ \hline
    Reconstructed Frames  & 648         & 911          & 2045        & 4741         \\ \hline
    %Risoluzione        & 228x114     & 228x114      & 228x114     & 228x114      \\ \hline
    PCD points          & 152763      & 204024       & 460914      & 1079375      \\ \hline
    Duration              & 44min 14s   & 1h 12min 34s & 3h 6min 32s & 10h 41min 8s \\ \hline
    Feature Extraction & 7s          & 8s           & 13s         & 28s          \\ \hline
    Exhaustive Matcher & 33s         & 49s          & 2min 32s    & 10min 22s    \\ \hline
    Mapper             & 23min 57s   & 44min 5s     & 2h 1min 25s & 8h 18s       \\ \hline
    Image Undistorter  & 0s          & 1s           & 1s          & 2s           \\ \hline
    Patch Match Stereo & 19min 28s   & 27min 15s    & 1h 1min 18s & 2h 24min 16s \\ \hline
    Stereo Fusion      & 9s          & 16s          & 1min 3s     & 5min 42s     \\ \hline
    \end{tabular}}
    \caption{Comparison of Recostruction details for scene P01\_01 using different Initial frames at same resolution of 228x128.
        The higher the frames, the better the reconstruction but at a higher computational time. }\label{tab:col_P01_01_frames}
    \end{table}


    \begin{table}[t]
        \centering
        \resizebox{0.7\textwidth}{!}{
        \begin{tabular}{|l|rrr|}
        \hline
        \rowcolor[HTML]{9B9B9B} 
        Scene                 & P01\_01\_04 & P01\_01\_04& P01\_01\_04  \\ \hline
        Initial Frames        & 1231        & 1231       & 1231         \\ \hline
        Reconstructed Frames  & 765         & 648        & 6            \\ \hline
        Resolution            & 456x256     & 228x114    & 114x64       \\ \hline
        PCD points   & \textbf{629270}      & 152763     & -            \\ \hline
        Duration              & 1h 7min 5s  & 44min 14s  & -            \\ \hline
        Feature Extraction    & 16 s        & 7s         & -            \\ \hline
        Exhaustive Matcher    & 42s         & 33s        & -            \\ \hline
        Mapper                & 18min 35s   & 23min 57s  & -            \\ \hline
        Image Undistorter     & 1s          & 0s         & -            \\ \hline
        Patch Match Stereo    & 47min 1s    & 19min 28s  & -            \\ \hline
        Stereo Fusion         & 30s         & 9s         & -            \\ \hline
        \end{tabular}}
        \caption{Comparison of Recostruction of scene P01\_01 using different Resolutions. The higher the resolution, the better.
        Too low resolution, as 114x64 in this case can lead to a unsuccessful reconstruction.}\label{tab:col_P01_01_res}
        \end{table}
    \begin{comment}
    
    
    \begin{table}[]
    \centering
    \resizebox{\textwidth}{!}{
    \begin{tabular}{|l|rrrrrr|}
    \hline
    \rowcolor[HTML]{9B9B9B} 
    Scenes             & P01\_01\_04 & P01\_01\_04 & P01\_01\_04 & P01\_01\_06  & P01\_01\_08 & P01\_01\_09  \\ \hline
    Frames Iniziali    & 1231        & 1231        & 1231        & 1487         & 2598        & 5223         \\ \hline
    Frames Ricostruiti & 765         & 6           & 648         & 911          & 2045        & 4741         \\ \hline
    Risoluzione        & 456x256     & 114x64      & 228x114     & 228x114      & 228x114     & 228x114      \\ \hline
    Punti PCD          & 629270      & -           & 152763      & 204024       & 460914      & 1079375      \\ \hline
    Tempi              & 1h 7min 5s  & -           & 44min 14s   & 1h 12min 34s & 3h 6min 32s & 10h 41min 8s \\ \hline
    Feature Extraction & 16 s        & -           & 7s          & 8s           & 13s         & 28s          \\ \hline
    Exhaustive Matcher & 42s         & -           & 33s         & 49s          & 2min 32s    & 10min 22s    \\ \hline
    Mapper             & 18min 35s   & -           & 23min 57s   & 44min 5s     & 2h 1min 25s & 8h 18s       \\ \hline
    Image Undistorter  & 1s          & -           & 0s          & 1s           & 1s          & 2s           \\ \hline
    Patch Match Stereo & 47min 1s    & -           & 19min 28s   & 27min 15s    & 1h 1min 18s & 2h 24min 16s \\ \hline
    Stereo Fusion      & 30s         & -           & 9s          & 16s          & 1min 3s     & 5min 42s     \\ \hline
    \end{tabular}}
    \caption{Recostruction of scene P01\_01 with details}\label{tab:col_P01_01}
    \end{table}

    
    \end{comment}
\begin{figure}[H]
    \centering
    \includegraphics[width=1\linewidth]{images/esperimenti/Colmap_P01_kitchen.png} 
    \caption{Varying COLMAP pcd reconstruction changing number of samples and 
    resolution. Each row is the same reconstruction viewed from different viewpoints.From top to bottom the number of frames increase, while in the first
    two rows the same split is compared using different resolutions.}\label{fig:colmap_P01}
\end{figure}

\section{Monocular Pipeline}
Here I report the qualitative result obtained from the Monocular Pipeline. As expected the results are really poor.
The main reason of the failure of this technique is due to the inaccuracy of the depth estimator. Once the frames are 
projected in the space we also have to find a threshold for the distance at which a reconstruction point is labeled as 
dynamic or static. This make the pipeline highly scene-specific requiring each time a lot of fine tuning for a mediocre result as 
can be seen in Figure~\ref{fig:Monoc}.

Also using a distance principle for segmentation, we can see how the scene is deteriorated in this form of globular groups of points(see Figure~\ref{fig:glob}).
\begin{figure}
    \centering
    \includegraphics[width=1\linewidth]{images/esperimenti/monocular_bozza.png} 
    \caption{Frame and its projection in the 3D space.}\label{fig:Monoc}
\end{figure}
\begin{figure}
    \centering
    \includegraphics[width=1\linewidth]{images/esperimenti/glob.png} 
    \caption{Scene cleaned from dynamic points. (Dovrei rifarla...)}\label{fig:glob}
\end{figure}



\section{NeuralDiff Pipeline}
To evaluate our final pipeline we started by creating the splits upon which we would have trained the neural render.
In particular we focused on one scene, P01-01, to see how the number of frames selected and the method which 
selects them affect the pipeline performances.

In Figure~\ref{fig:samplFreq} it is given a visualization of the three different subsampling using the methods presented
in Section~\ref{sec:Sampl} for a total of 217 frames.

\begin{table}[t]
\centering
\begin{tabular}{|c|c|c|c|c|c|}
\hline \rowcolor[HTML]{A6A6A6}P01-01 & Sampling  & Durata [s] & PSNR & \begin{tabular}{l} 
PSNR \\
statico
\end{tabular} & mAP \\
\hline 2938 & Int. &                                  $11 \mathrm{~h} 36 \min 59 \mathrm{~s}$           & \textbf{24.82}         & 20.41                  & 72.21 \\
\hline 2938 & Unif &                                  $11 \mathrm{~h} 16 \mathrm{~min} 8 \mathrm{~s}$   & 24.42                  & 20.41                  & \textbf{72.79} \\
\hline \rowcolor[HTML]{D9D9D9} 2015 & Int. &          $8 \mathrm{~h} 3 \mathrm{~min} 50 \mathrm{~s}$    & \textbf{24.51}         & \textbf{20.46}         & \textbf{70.49 }\\
\hline \rowcolor[HTML]{D9D9D9} 2015 & Unif &          7h 52min 34s                                      & 23.93                  & 20.33                  & 69.87 \\
\hline 1000 & Int. &                                  3h 43min 41s                                      & \textbf{23.59}         & \textbf{20.37 }        & 67.55 \\
\hline 1000 & Unif &                                  $3 \mathrm{~h} 43 \mathrm{~min} 1 \mathrm{~s}$    & 22.8                   & 20.10                  & 66.51 \\
\hline 1000 & $\mathrm{AU}$ &                         $4 \mathrm{~h} 2 \mathrm{~min} 45 \mathrm{~s}$    & 23.43                  & 20.31                  & \textbf{67.99} \\
\hline \rowcolor[HTML]{D9D9D9} 722 & Int. &           2h $34 \mathrm{~min}$                             & \textbf{22.65}         & \textbf{20.20}         & \textbf{65.42} \\
\hline \rowcolor[HTML]{D9D9D9} 722 & Unif &           $2 \mathrm{~h} 30 \mathrm{~min} 7 \mathrm{~s}$    & 22.09                  & 19.65                  & 62.95 \\
\hline \rowcolor[HTML]{D9D9D9} 722 & $\mathrm{AU}$ &  2h $36 \mathrm{~min} 46 \mathrm{~s}$              & 22.48                  & 20.05                  & 64.10 \\
\hline 397 & Int. &                                   1h $19 \mathrm{~min} 53 \mathrm{~s}$              & \textbf{21.33}         & 19.64                  & 56.63 \\
\hline 397 & Unif &                                   1h $21 \min 42 \mathrm{~s}$                       & 20.98                  & 19.54                  & \textbf{61.6 }\\
\hline 397 & $\mathrm{AU}$ &                          1h 14min 36s                                      & 21.2                   & \textbf{19.95}         & 59.97 \\
\hline \rowcolor[HTML]{D9D9D9} 217 & Int. &           $40 \mathrm{~min} 55 \mathrm{~s}$                 & 20.32                  & \textbf{19.60}         & 51.69 \\
\hline \rowcolor[HTML]{D9D9D9} 217 & Unif &           $41 \mathrm{~min} 8 \mathrm{~s}$                  & \textbf{20.51}         & 19.42                  & \textbf{53.00} \\
\hline \rowcolor[HTML]{D9D9D9} 217 & $\mathrm{AU}$ &  $25 \mathrm{~min} 39 \mathrm{~s}$                 & 20.26                  & 19.39                  & 50.58 \\
\hline
\end{tabular}
\caption{\textbf{NeuralDiff Pipeline Results on P01-01 at 114x64. }For the same scene P01-01 results of NeuralDiff pipeline trained on different amount of frames are reported.The Frames
are selected using the three different sampling strategis:Intelligent,Uniform and AU(see Section~\ref{sec:sampl}). The frames are all at a 114x64 resolution.}\label{tab:EpicInt114}

\end{table}

We then proceeded in testing the various split for P01-01 obtaining the results reported in Table~\ref{tab:EpicInt114}.
As we can see the Intelligent sampling is actually working. The PSNR is always higher with respect to the other methods.
It can be seen that actually all methods suffer the scarcity of frames and as a matter of fact in the last split, 217, 
uniform sampling beats the intelligent one. For the static PSNR instead the Intelligent method is always better
than the uniform one, even at low frames. For the mean Average Precision instead we can see some oscillations,
but we have to be careful since our mask is actually combining the actor and the foreground layer, meaning that 
the average precision is not actually assessing the ability of the model to distinguish these two parts. 
For example in Figure~\ref{fig:comp} we can see that in the 400 frames splits is exactly present this deficit, where the 
uniform sampling is totally unable to detect the actor even though its mAP is higher than the Intelligent 
one($mAP_{Uniform}61.6\% > mAP_{Intelligent}56.63\%$).
\begin{figure}[H]
    \centering
    \includegraphics[scale=0.5,angle=90]{images/esperimenti/comparison53.pdf} 
    \caption{Visualization of the output of the different models trained on differen splits.}\label{fig:comp}
\end{figure}
\textcolor{red}{Also for our aim to segment dynamic objects, we are interested in the static PSNR (as we obtain dynamic objects as 
what is not static) and Figure~\ref{fig:comp} shows us that the static part is almost identical for each scene.}

On the other hand,going towards qualitative results, here we present the 3D static reconstruction for P01-01.
As shown in Figure~\ref{fig:statP01_01},the first row is the COLMAP pointcloud extracted from the sampled videosequence. Below are placed
instead the static reconstructions for the three different sampling strategies. The first thing that comes to our eyes is 
the overall colour which in the Intelligent sampling seem more faithful to the reality. The second thing is the segmentation of 
the plate on the table top, which can be seen in the COLMAP row. The plate is successfully removed in the Int. sampling while
it is still visible in the other splits, although the best model was the Unif. one according to the metrics. Another example 
is given by the pan highlighted with the green circle. Which is removed in the Intelligent sampling while not in the others.

Other comparison with other frames splits are provided in Figure~\ref{fig:statP01_02}.
\begin{figure}[H]
    \centering
    \includegraphics[width=1\linewidth]{images/esperimenti/nerf_pcd_comparison-06.png} 
    \caption{Qualitative results for the static reconstruction of P01-01 scene at 217 frames. In red is highlighted
            a dynamic plate, while in green a dynamic pan.}\label{fig:statP01_01}
\end{figure}
\begin{figure}
    
    \adjustbox{trim={1.5cm 0 0 0}} {\includegraphics[scale=0.7]{images/esperimenti/nerf_pcd_comparison-07.png} }
    \caption{Comparative of the qualitative results for different samplings of the P01-01 scene.}\label{fig:statP01_02}
\end{figure}
\textcolor{red}{Bastano questi esempi sul 3D?}

To further validate our results, we repeated the experiments using different scenes using a split of \~1000 frames.
 The results are reported in Table~\ref{tab:Epic_res} for resolution 114x64 and Figure~\ref{fig:Epic228} at 
 a resolution of 228x128. The results assess that our method is working.

\begin{table}[H]
\resizebox{\textwidth}{!}{\begin{tabular}{|c|c|c|c|c|c|c|}
    \hline  \rowcolor[HTML]{A6A6A6} Scene & Sampled Frames & Sampling & Durata [s] & PSNR & \begin{tabular}{l} 
    PSNR \\
    statico
    \end{tabular} & mAP \\
    \hline \multirow[t]{2}{*}{ P01-01 } & \multirow[t]{2}{*}{1089} &  \cellcolor[HTML]{C0C0C0}Intelligent & \cellcolor[HTML]{C0C0C0} $3 \mathrm{~h} 43 \mathrm{~min} 41 \mathrm{~s}$ & \cellcolor[HTML]{C0C0C0} 23.59 &  \cellcolor[HTML]{C0C0C0}20.37 & \cellcolor[HTML]{C0C0C0} 67.55 \\
     & & Uniform & 3h $43 \mathrm{~min} 1 \mathrm{~s}$ & 22.8 & 20.10 & 66.51 \\
    \hline \multirow[t]{2}{*}{ P03-04 } & \multirow[t]{2}{*}{868} &  \cellcolor[HTML]{C0C0C0}Intelligent &  \cellcolor[HTML]{C0C0C0}3h $14 \mathrm{~min} 44 \mathrm{~s}$ &  \cellcolor[HTML]{C0C0C0}19.90 &  \cellcolor[HTML]{C0C0C0} 16.73 &  \cellcolor[HTML]{C0C0C0} 61.92 \\
     & & Uniform & 3h $10 \mathrm{~min} 48 \mathrm{~s}$ & 19.13 & 16.73 & 64.00 \\
    \hline \multirow[t]{2}{*}{ P04-01 } & \multirow[t]{2}{*}{1098} & \cellcolor[HTML]{C0C0C0} Intelligent &  \cellcolor[HTML]{C0C0C0}4h $10 \mathrm{~min} 12 \mathrm{~s}$ &  \cellcolor[HTML]{C0C0C0}24.32 &  \cellcolor[HTML]{C0C0C0} 21.23 &  \cellcolor[HTML]{C0C0C0} 0.7137 \\
     & & Uniform & 4h $2 \min 9 s$ & 23.81 & 20.49 & 65.81 \\
    \hline \multirow[t]{2}{*}{ P09-02 } & \multirow[t]{2}{*}{903} & \cellcolor[HTML]{C0C0C0} Intelligent &  \cellcolor[HTML]{C0C0C0}3h $25 \mathrm{~min} 58 \mathrm{~s}$ &  \cellcolor[HTML]{C0C0C0}23.97 &  \cellcolor[HTML]{C0C0C0} 19.43 &  \cellcolor[HTML]{C0C0C0} 60.05 \\
     & & Uniform & 3h $17 \mathrm{~min} 5 \mathrm{~s}$ & 23.39 & 19.45 & 61.23 \\
    \hline \multirow[t]{2}{*}{ P16-01 } & \multirow[t]{2}{*}{1004} & \cellcolor[HTML]{C0C0C0} Intelligent & \cellcolor[HTML]{C0C0C0} 3h $46 \mathrm{~min} 57 \mathrm{~s}$ &  \cellcolor[HTML]{C0C0C0}22.89 &  \cellcolor[HTML]{C0C0C0} 20.17 &  \cellcolor[HTML]{C0C0C0} 66.89 \\
     & & Uniform & 3h 43min 11s & 22.75 & 19.94 & 63.61 \\
    \hline \multirow[t]{2}{*}{ P21-01 } & \multirow[t]{2}{*}{855} & \cellcolor[HTML]{C0C0C0} Intelligent & \cellcolor[HTML]{C0C0C0} 3h $15 \mathrm{~min} 59 \mathrm{~s}$ & \cellcolor[HTML]{C0C0C0} 20.02 &  \cellcolor[HTML]{C0C0C0} 15.73 &  \cellcolor[HTML]{C0C0C0} 72.94 \\
     & & Uniform & 3h $7 \mathrm{~min} 50 \mathrm{~s}$ & 19.11 & 15.07 & 68.88 \\
    \hline
    \end{tabular}}
    \caption{Results all scenes Epic 114}\label{tab:Epic_res}
\end{table}




\begin{figure}[t]
    \centering
    \includegraphics[width=1\linewidth]{images/esperimenti/Epic_228.png} 
    \caption{Experiments performed on 228x128 frames for each scene}\label{fig:Epic228}
\end{figure}
Other than the idea behind the Intelligent sampling which was expressed in Section~\ref{sec:sampl}, we found a
link between the positions of the sampled freqencies and the frequencies of the objects that were used in different 
equispaced time intervals. In Figure~\ref{fig:samplFreq} the three methods sampling 
frequencies are reported. In Figure~\ref{} and Figure~\ref{} instead are reported respectively
the comparison of Intelligent and Uniform sampling with the objects count for scene P01-01 with varying number of samples,
as can be read on each sub-figure; and the comparison of Intelligent and Uniform sampling with the objects count for each scene
with fixed sampling at \~1000 frames.

As we can see from the plots,at exception from few scenes, the profile of the Intelligent sampling looks closer
to the one of the object counts. This is giving a further explanation of the functioning of our proposed method.
In fact it means that our sampling is focusing on those areas where a lot of actions are performed. In this way
long redundant actions are filtered and only relevant frames are kept. 

\begin{figure}
    \centering
    \includegraphics[width=1\linewidth]{images/esperimenti/sampling_frequencies.pdf} 
    \caption{Visualization of the sampling for the three different methods: Intelligent, Uniform and AU using 217 frames in total.}\label{fig:samplFreq}
\end{figure}

\begin{figure}
    \centering
    \includegraphics[width=1\linewidth]{images/esperimenti/ObjectCountsScenes-02.png} 
    \caption{Comparison of frequencies for the Intelligent and Uniform sampling with the Object Count for the P01-01 scene changing the total
    number of sampled frames.}\label{fig:samplFreq}
\end{figure}

\begin{figure}
    \centering
    \includegraphics[width=1\linewidth]{images/esperimenti/ObjectCountsScenes-01.png} 
    \caption{Comparison of frequencies for the Intelligent and Uniform sampling with the Object Count for each scene at a fixed split \~1000 frames.}\label{fig:samplFreq}
\end{figure}

We also tried to give a quantitative measure of similarity and dissimilarity by comparing some different metrics: Cosine Similarity, Kullback-Leibler Divergence (KLD),
Jensen-Shannon Divergence (JSD), Correlation Coefficient. \textcolor{red}{Le devo spiegare? E' meglioo se le metto in Method?}
\begin{table}
\resizebox{\textwidth}{!}{\begin{tabular}{|c|
    >{\columncolor[HTML]{C0C0C0}}c | >{\columncolor[HTML]{C0C0C0}}c|c|c|
    >{\columncolor[HTML]{C0C0C0}}c|>{\columncolor[HTML]{C0C0C0}}c|c|c|c|}
\hline \rowcolor[HTML]{A6A6A6} & \multicolumn{2}{|c|}{ Cosine Similarity } & \multicolumn{2}{|c|}{ K-L Div. } & \multicolumn{2}{|c|}{ JS-Div } & \multicolumn{2}{|c|}{ Correlation } \\
\hline \rowcolor[HTML]{A6A6A6}& Intelligent & Uniform & Intelligent & Uniform & Intelligent & Uniform & Intelligent & Uniform \\
\hline P01-01 & 0.91319715 & 0.90718451 & 0.113071 & 0.13421 & 0.03931 & 0.0450167 & 0.5632 & 0.5391 \\
\hline P03-04 & 0.7638 & 0.6773 & 0.4224 & 0.5389 & 0.16092 & 0.2025 & 0.5483 & 0.2226 \\
\hline P04-01 & 0.6528 & 0.6316 & 0.5892 & 0.6631 & 0.2196 & 0.2438 & 0.1454 & 0.1353 \\
\hline P09-02 & 0.7438 & 0.7323 & 2.2962 & 1.2859 & 0.1377 & 0.1399 & -0.1736 & -0.1152 \\
\hline P16-01 & 0.8029 & 0.8130 & 0.2170 & 0.2431 & 0.0727 & 0.0815 & 0.3490 & 0.4416 \\
\hline P21-01 & 0.8804 & 0.8761 & 0.1694 & 0.1760 & 0.0574 & 0.0611 & 0.3633 & 0.1889 \\
\hline
\end{tabular}}
\caption{Metrics for comparing the profile of the histograms. In particular higher values of Cosine similarity and Correlation indicates similarity; while
the value of the two divergences represents the distance between the two distributions.}

\end{table}


\begin{comment}
\begin{table}
    \resizebox{\textwidth}{!}{\begin{tabular}{|c|c|c|c|c|c|c|c|c|}
    \hline \rowcolor[HTML]{A6A6A6} & \multicolumn{2}{|c|}{ Cosine Similarity } & \multicolumn{2}{|c|}{ K-L Div. } & \multicolumn{2}{|c|}{ JS-Div } & \multicolumn{2}{|c|}{ Correlation } \\
    \hline \rowcolor[HTML]{A6A6A6}& Intelligent & Uniform & Intelligent & Uniform & Intelligent & Uniform & Intelligent & Uniform \\
    \hline P01-01 & 0.91319715 & 0.90718451 & 0.113071 & 0.13421 & 0.03931 & 0.0450167 & 0.5632 & 0.5391 \\
    \hline P03-04 & 0.7638 & 0.6773 & 0.4224 & 0.5389 & 0.16092 & 0.2025 & 0.5483 & 0.2226 \\
    \hline P04-01 & 0.6528 & 0.6316 & 0.5892 & 0.6631 & 0.2196 & 0.2438 & 0.1454 & 0.1353 \\
    \hline P09-02 & 0.7438 & 0.7323 & 2.2962 & 1.2859 & 0.1377 & 0.1399 & -0.1736 & -0.1152 \\
    \hline P16-01 & 0.8029 & 0.8130 & 0.2170 & 0.2431 & 0.0727 & 0.0815 & 0.3490 & 0.4416 \\
    \hline P21-01 & 0.8804 & 0.8761 & 0.1694 & 0.1760 & 0.0574 & 0.0611 & 0.3633 & 0.1889 \\
    \hline
    \end{tabular}}
    \caption{Metrics for comparing the profile of the histograms. In particular higher values of Cosine similarity and Correlation indicates similarity; while
    the value of the two divergences represents the distance between the two distributions.}
    
    \end{table}
\end{comment}

\textcolor{red}{Avrei anche questa tabella ma il sampling anti/uniform sono sbagliati.
potrei rifarli? Altrimenti non abbiamo messo la parte di simplyfing NeuralDiff? Figure~\ref{fig:NeuralCleaner}}
\begin{figure}
    \centering
    \includegraphics[width=1\linewidth]{images/esperimenti/NeuralCleaner.png} 
    \caption{Results for architecture with foreground+actor= fused}\label{fig:NeuralCleaner}
\end{figure}

\textcolor{red}{Non ho messo da nessuna parte il risultato dei sampling. Aggiungere la tabella..!}