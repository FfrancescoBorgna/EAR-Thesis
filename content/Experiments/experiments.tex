\chapter{Experiments}
\section{Data selection}
The data selection was dictated by our problem. Indeed evaluating 3D scenes 
reconstructions is not an easy task due to the lack of 3D ground-truths. These
are usually very expensive due to costly hardware scanners but sometimes are 
also not really available, as in our scenario, where we would like a static-dynamic
segmentation. In our case in fact obtaining the static part would mean to 
actually clean the scene from all the possible moving objects which adds 
an extra cost in terms of time, but in other scenarios 'cleaning' the environment
could not be allowed.

\paragraph{EPIC-Diff.}For this fact we evaluated our scene reconstructions on a subsample\footnote{\textcolor{red}{Come giustifichiamo l'aver preso non tutte le scene? Per motivi di tempo è accettabile?}}
of the EPIC-KITCHENS extension proposed in NeuralDiff~\cite{neuraldiff},known as EPIC-Diff. In this extension
the authors of NeuralDiff added manually pixelwise segmented masks for ten scenes of which we just considered the 
P01-01,P03-04,P04-01,P09-02,P16-01,P21-01 splits.

\paragraph{VISOR.}We looked also for 
the more recent VISOR~\cite{visor} dataset in which pixel annotations of hands 
and active objects are given but unfortunately their definition of \textit{active}
was not suitable for our work. They labeled as active any object that is 
included in the current action, so it is common to see as active the sink or 
the gas stove, but in our case they should be considered as static(see Figure~\ref{fig:vis_exp}).
\begin{table}[]
    \centering
    \begin{tabular}{lr}
    \hline
    \textbf{Scene} & \multicolumn{1}{c}{\textbf{Frames}} \\ \hline
    P01-01         & 98935                               \\
    P03-04         & 100251                              \\
    P04-01         & 69292                               \\
    P09-02         & 22187                               \\
    P16-01         & 74592                               \\
    P21-01         & 41583                               \\ \hline
    \end{tabular}
    \caption{Total Frames for each scene}\label{tab:Frames}
    \end{table}
\begin{figure}[H]
    \centering
    \includegraphics[width=1\linewidth]{images/esperimenti/visor_exp.png} 
    \caption{Example of VISOR active annotations, on the left 'wash a knife' include the
    static sink as active; on the right 'pour spice' static gas stove is active}\label{fig:vis_exp}
\end{figure}

\section{Metrics}\label{sec:Metrics}
As regards metrics we looked in literature for a way to evaluate
our results but unfortunately each method involved a ground truth
which for our dataset is not available. Possible ways to obtain
a groundtruth could be manual annotations or simulating the environments.
Both these two methods would take a considerable large amount
of time and are also beyond the scope of this thesis.

For this reason we ended up by using the metrics proposed in~\cite{neuraldiff}.
Namely these are:
\begin{itemize}
    \item \textbf{PSNR}
    \item \textbf{mAP}
\end{itemize}
\subsection{PSNR:Peak signal-to-noise ratio}
The Peak Signal-to-Noise Ratio (PSNR) is a metric commonly used in image and
video processing to quantify the quality of a reconstructed or processed signal,
like an image or video. It gives a measures of the ratio between the 
maximum possible power of a signal (MAX) and the power of the distortion or noise 
that affects the signal (MSE).

The formula for PSNR is usually expressed in decibels (dB) and is given by:

\[ \text{PSNR} = 20 \cdot \log_{10}\left(\frac{{\text{MAX}}}{{\text{MSE}}}\right) \]

where:
\begin{itemize}
    \item MAX is the maximum possible pixel value of the image (1 in our case).
    \item MSE is the Mean Squared Error, which represents the average squared 
    difference between the original signal and the reconstructed or distorted signal.
\end{itemize}

It is worth noting that a high PSNR does not guarantee that the processed signal 
will be perceived as visually pleasing or high-quality by humans, especially in the case of perceptually sensitive applications like image and video compression.

\subsection{AP:Average Precision}
Average Precision (AP) is a metric commonly used in object detection and information retrieval
to evaluate the performance of machine learning models. It measures the \textit{precision-recall} trade-off of a model.

It can be useful to remind what \textit{Precision} and \textit{Recall} are. Namely:

\begin{equation}
    Precision=\frac{TP}{TP+FP}
\end{equation}
\begin{equation}
    Recall=\frac{TP}{TP+FN}
\end{equation}

where:
\begin{itemize}
    \item TP=True positive
    \item FP=False positive
    \item TN=True negative
\end{itemize}
Average precision is then computed as the area below the precision-recall curve, specifically
the curve obtained by varying the confidence threshold of the inference model as shown
in Figure \ref{fig:auc}. That is why 
it can also be found in literature as AUC(Area Under Curve).
Its scalar value summarize the precision-recall performance of the model.
A higher AP is desirable, indicating a model that effectively retrieves 
relevant instances while minimizing false positives.

\begin{figure}[t]
    \centering
    \includegraphics[width=0.5\linewidth]{images/metrics/auc.png} % Replace "example-image" with your image file name
    \caption{Example of Precision-recall curve.We can see how the bottom line model 
        represents the worst a model can perform, e.g. predict every sample as it is 
        coming from the same class,if the dataset is balanced. A better model would
        \textit{tend} to the upper-right corner, which instead represents the best
        possible model, a model that have maximum precision and recall.}\label{fig:auc}
\end{figure}

\section{COLMAP Reconstruction}
Once we have fixed the data we were working on, we proceeded to do some experiments
on COLMAP reconstructions. In particular we took the scene P01-01 and tried varying 
both the number of frames and their resolution for the reconstruction. In fact most
of the scene have too many frames to handle, which could results in out of memory issues
or at the least worst in a long computational time. Our aim was to find a good compromise
between \textit{quality of reconstruction} and \textit{computational time}.

\paragraph{Quantity vs Quality }The first thing we did was to subsample the frames using the same technique 
as reported in Epic FIELDS~\cite{epic_fields} and explained in Section~\ref{sec:sampl}.
We report the results of the COLMAP reeconstructions both quantitatively and qualitatively
in Tables~\ref{tab:col_P01_01_frames},~\ref{tab:col_P01_01_res} and Figures~\ref{fig:colmap_P01_frames},~\ref{fig:colmap_P01_res}. It is worth noting
few things watching these references. The first one is that the 
resolution plays an important role in the successfulness of the reconstruction as 
we can see from Table~\ref{tab:col_P01_01_res}. The same split
of frames is reported and the right one at a resolution of 114x64 failed. This 
is due to the feature extractor, that in a high resolution image can retrieve 
informations that instead are lost in low resolution frames. A lack of significant features 
means no matching between images so the reconstruction has very few frames matched.
The second thing is that the higher the frames the better. In fact chances of matching
increases and also we will have more areas of the environment covered, as shown in Figure~\ref{fig:colmap_P01_frames}.
We can see that augmenting the number of frames more parts of the kitchen are revealed,
\textit{e.g.} the round table at the center of the room, the sideboard in front of the sink.
But also some important objects that are visible from the video,like dishes on top 
of the table. This is a keypoint to the development of our pipeline, because we need to 
be sure that the scene actually contains points deriving from the motion of objects.

By considering these results and always keeping in mind the time of computation at our disposal
we opted to feed the next pipeline with around five thousands frames at a resolution 
of 228x128. The pipeline of NeuralDiff in fact is really heavy and working at full resolution
was prohibitive in the number of experiments we could try.

% Please add the following required packages to your document preamble:
% \usepackage[table,xcdraw]{xcolor}
% Beamer presentation requires \usepackage{colortbl} instead of \usepackage[table,xcdraw]{xcolor}
\begin{table}[t]
    \centering
    \resizebox{\textwidth}{!}{
    \begin{tabular}{|l|rrrr|}
    \hline
    \rowcolor[HTML]{9B9B9B} 
    Scenes             & P01\_01\_04 & P01\_01\_06  & P01\_01\_08 & P01\_01\_09  \\ \hline
    Initial Frames    & 1231        & 1487         & 2598        & 5223         \\ \hline
    Reconstructed Frames  & 648         & 911          & 2045        & 4741         \\ \hline
    %Risoluzione        & 228x114     & 228x114      & 228x114     & 228x114      \\ \hline
    PCD points          & 152763      & 204024       & 460914      & 1079375      \\ \hline
    Duration              & 44min 14s   & 1h 12min 34s & 3h 6min 32s & 10h 41min 8s \\ \hline
    Feature Extraction & 7s          & 8s           & 13s         & 28s          \\ \hline
    Exhaustive Matcher & 33s         & 49s          & 2min 32s    & 10min 22s    \\ \hline
    Mapper             & 23min 57s   & 44min 5s     & 2h 1min 25s & 8h 18s       \\ \hline
    Image Undistorter  & 0s          & 1s           & 1s          & 2s           \\ \hline
    Patch Match Stereo & 19min 28s   & 27min 15s    & 1h 1min 18s & 2h 24min 16s \\ \hline
    Stereo Fusion      & 9s          & 16s          & 1min 3s     & 5min 42s     \\ \hline
    \end{tabular}}
    \caption{Comparison of Recostruction details for scene P01\_01 using different Initial frames at same resolution of 228x128.
        The higher the frames, the better the reconstruction but at a higher computational time. }\label{tab:col_P01_01_frames}
    \end{table}


    \begin{table}[t]
        \centering
        \resizebox{0.7\textwidth}{!}{
        \begin{tabular}{|l|rrr|}
        \hline
        \rowcolor[HTML]{9B9B9B} 
        Scene                 & P01\_01\_04 & P01\_01\_04& P01\_01\_04  \\ \hline
        Initial Frames        & 1231        & 1231       & 1231         \\ \hline
        Reconstructed Frames  & 765         & 648        & 6            \\ \hline
        Resolution            & 456x256     & 228x114    & 114x64       \\ \hline
        PCD points   & \textbf{629270}      & 152763     & -            \\ \hline
        Duration              & 1h 7min 5s  & 44min 14s  & -            \\ \hline
        Feature Extraction    & 16 s        & 7s         & -            \\ \hline
        Exhaustive Matcher    & 42s         & 33s        & -            \\ \hline
        Mapper                & 18min 35s   & 23min 57s  & -            \\ \hline
        Image Undistorter     & 1s          & 0s         & -            \\ \hline
        Patch Match Stereo    & 47min 1s    & 19min 28s  & -            \\ \hline
        Stereo Fusion         & 30s         & 9s         & -            \\ \hline
        \end{tabular}}
        \caption{Comparison of Recostruction of scene P01\_01 using different Resolutions. The higher the resolution, the better.
        Too low resolution, as 114x64 in this case can lead to a unsuccessful reconstruction.}\label{tab:col_P01_01_res}
        \end{table}
    \begin{comment}
    
    
    \begin{table}[]
    \centering
    \resizebox{\textwidth}{!}{
    \begin{tabular}{|l|rrrrrr|}
    \hline
    \rowcolor[HTML]{9B9B9B} 
    Scenes             & P01\_01\_04 & P01\_01\_04 & P01\_01\_04 & P01\_01\_06  & P01\_01\_08 & P01\_01\_09  \\ \hline
    Frames Iniziali    & 1231        & 1231        & 1231        & 1487         & 2598        & 5223         \\ \hline
    Frames Ricostruiti & 765         & 6           & 648         & 911          & 2045        & 4741         \\ \hline
    Risoluzione        & 456x256     & 114x64      & 228x114     & 228x114      & 228x114     & 228x114      \\ \hline
    Punti PCD          & 629270      & -           & 152763      & 204024       & 460914      & 1079375      \\ \hline
    Tempi              & 1h 7min 5s  & -           & 44min 14s   & 1h 12min 34s & 3h 6min 32s & 10h 41min 8s \\ \hline
    Feature Extraction & 16 s        & -           & 7s          & 8s           & 13s         & 28s          \\ \hline
    Exhaustive Matcher & 42s         & -           & 33s         & 49s          & 2min 32s    & 10min 22s    \\ \hline
    Mapper             & 18min 35s   & -           & 23min 57s   & 44min 5s     & 2h 1min 25s & 8h 18s       \\ \hline
    Image Undistorter  & 1s          & -           & 0s          & 1s           & 1s          & 2s           \\ \hline
    Patch Match Stereo & 47min 1s    & -           & 19min 28s   & 27min 15s    & 1h 1min 18s & 2h 24min 16s \\ \hline
    Stereo Fusion      & 30s         & -           & 9s          & 16s          & 1min 3s     & 5min 42s     \\ \hline
    \end{tabular}}
    \caption{Recostruction of scene P01\_01 with details}\label{tab:col_P01_01}
    \end{table}

    
    \end{comment}
\begin{figure}[H]
    \centering
    \includegraphics[width=1\linewidth]{images/esperimenti/ColmapComparisonP01-01.png} 
    \caption{Different COLMAP pcd reconstructions changing number of samples. 
    Each row is the same reconstruction viewed from different viewpoints.From top to bottom the number of frames increase. We can see
    how the number of frames positively affect the reconstruction.}\label{fig:colmap_P01_frames}
\end{figure}
\begin{figure}[H]
    \centering
    \includegraphics[width=1\linewidth]{images/esperimenti/ColmapComparisonP01-02.png} 
    \caption{Different COLMAP pcd reconstructions changing
    resolution. Each row is the same reconstructions viewed from different viewpoints. The first report 
    the reconstruction for a resolution of 456x256 and the bottom one the half resolution. It is clear 
    how the resolution has a beneficial impact on the overall reconstruction.}\label{fig:colmap_P01_res}
\end{figure}
In Figure~\ref{fig:COL_all} I give some visualization of other kitchen's reconstructions.
\begin{figure}[H]
    \centering
    \includegraphics[width=1\linewidth]{images/esperimenti/Colmaps/Colmap-09.png} 
    \caption{COLMAP reconstructions for scenes P03-04, P04-01, P09-02, P16-01,P21-01. Each row represents a different kitchen.}\label{fig:COL_all}
\end{figure}
\section{Monocular Pipeline}
Here I report the qualitative result obtained from the Monocular Pipeline. As expected the results are really poor.
The main reason of the failure of this technique is due to the inaccuracy of the depth estimator. In Figure~\ref{fig:Monoc} we can see
that in some scene it seems to capture the main elements, like the hands in scene  P01-01 or the pot in scene P16-01. However others like P09-02 seem
to completely get it wrong.

Once the frames are 
projected in the environment space we also have to find a threshold for the distance at which a reconstruction point is labeled as 
dynamic or static. This make the pipeline highly scene-specific requiring each time a lot of fine tuning for a mediocre result.

Also using a distance principle for segmentation, we can see how the scene is deteriorated in this form of globular groups of points(see Figure~\ref{fig:glob}).
While in Figure~\ref{fig:glob2} we can see how the segmentation of what is dynamic and what is static change changing the distance threshold. In particular the 
scene take was P03-04 and the values reported are 0.5, 5 and 20.

\begin{figure}[H]
    \centering
    \includegraphics[width=1\linewidth]{images/esperimenti/Monocular/MonocularProj-06.png} 
    \caption{Different scenes where monocular depth estimation was performed. In particular on 
    the right we can find the frame that is instead projected(in red) on the left in the 3D reconstruction of that kitchen.
    The red frustum(pyramid) represents the camera position and orientation in the space.}\label{fig:Monoc}
\end{figure}
\begin{figure}[t]
    \centering
    \includegraphics[width=1\linewidth]{images/esperimenti/glob.png} 
    \caption{Dynamic points segmented in scene P01-01 using Monocular Pipeline.}\label{fig:glob}
\end{figure}
\begin{figure}[H]
    \centering
    \includegraphics[width=0.7\linewidth]{images/esperimenti/Monocular/Globular-02.png} 
    \caption{Different Segmentation changing the distance threshold. Each point is segmented as dynamic if its distance from a pixel 
    projected in 3D space is less than a threshold Th. The scene is P03-04 and in the 
    left side we can find the static part while in the right side we have the dynamic points. Here we can notice 
    how this method is pretty inaccurate.}\label{fig:glob2}
\end{figure}

\section{Sampling Frames}
In order to work with the NeuralDiff pipeline that follows this section, we have to further downsample the frames reconstructed by COLMAP
to have some reasonable computational times. We found that at a resolution of 114x64 $\sim1000$ a scene took  $\sim4h$ while at 228x128 $\sim700$ 
a scene took  $\sim11h$. In Table~\ref{tab:samplingInt700} and Table~\ref{tab:samplingInt1000} we can see the number of frames kept at each sampling 
step, obtained usiing Homography filter and the proposed Intelligent Sampling(see Section~\ref{sec:sampl}).
\begin{table}[t]
    \resizebox{\textwidth}{!}{
    \rowcolors{2}{gray!25}{white}
\begin{tabular}{|c|c|c|c|c|c|c|}
    \hline & \textbf{Original} & \textbf{Sampled} & \textbf{Reconstructed} & \textbf{Obtained} & \textbf{Int/Unif Samples} & \textbf{Threshold} \\
    \hline P01-01 & 98935 & 5223 & 4741 & 3652 & \textbf{1089} & 0.9 \\
    \hline P03-04 & 100251 & 5060 & 4522 & 3654 & \textbf{868} & 0.855 \\
    \hline P04-01 & 69292 & 4269 & 3242 & 2144 & \textbf{1098} & 0.89 \\
    \hline P09-02 & 22187 & 4953 & 4398 & 3495 & \textbf{903} & 0.975 \\
    \hline P16-01 & 74592 & 5531 & 5480 & 4476 & \textbf{1004} & 0.945 \\
    \hline P21-01 & 415853 & 4655 & 4588 & 3733 & \textbf{855} & 0.94 \\
    \hline
    \end{tabular}}
    \caption{\textbf{Split$\sim1000$. }Number of frames resulting from the different sampling steps. In particular from Original the frames are 
    reduced with the Homography filter to remove redundancy and keep overlap. The reconstructed frames are the ones
    which were successfully reconstructed by COLMAP. The obtained ones are the reconstructed frames filtered again
    with the homography filter. Int/Unif Samples are the reconstructed frames without the Obtaineds. The thresholds
    reported are referred to the last Homography filter step.
    }\label{tab:samplingInt1000}
\end{table}



\begin{table}[t]
    \resizebox{\textwidth}{!}{
    \rowcolors{2}{gray!25}{white}
\begin{tabular}{|l|r|r|r|r|r|r|}
    \hline & \textbf{Originali} & \textbf{Sampled} & \textbf{Reconstructed} & \textbf{Obtained} & \textbf{Int/Unif Samples} &\textbf{ Threshold} \\
    \hline P01-01 & 98935          & 5223          & 4741 & 4019 & \textbf{722} & 0.91 \\
    \hline P03-04 & 100251          & 5060         & 4522 & 3839 & \textbf{683} &0.86 \\
    \hline P04-01 & 69292          & 4269          & 3242 & 2463 & \textbf{779} & 0.896 \\
    \hline P09-02 & 22187          & 4953          & 4398 & 3776 & \textbf{622} & 0.977 \\
    \hline P16-01 & 74592          & 5531          & 5480 & 4837 & \textbf{643} &0.95 \\
    \hline P21-01 & 415853          & 4655         & 4588 &3942  & \textbf{646} & 0.945 \\
    \hline
    \end{tabular}}
    \caption{\textbf{Split$\sim700$. }Number of frames resulting from the different sampling steps. In particular from Original the frames are 
    reduced with the Homography filter to remove redundancy and keep overlap. The reconstructed frames are the ones
    which were successfully reconstructed by COLMAP. The obtained ones are the reconstructed frames filtered again
    with the homography filter. Int/Unif Samples are the reconstructed frames without the Obtaineds. The thresholds
    reported are referred to the last Homography filter step.
    }\label{tab:samplingInt700}
\end{table}

\section{NeuralDiff Pipeline}
\paragraph{Different number of samples for the same scene.}
To evaluate our final pipeline we started by creating the splits upon which we would have trained the neural render.
In particular we focused on one scene, P01-01, to see how the number of frames selected and the method which 
selects them affect the pipeline performances.
In Figure~\ref{fig:samplFreq} it is given a visualization of the three different subsampling obtained using the methods presented
in Section~\ref{sec:sampl} for a total of 217 frames.

We then proceeded in testing the various split for P01-01 obtaining the results reported in Table~\ref{tab:EpicInt114}.
As we can see the Intelligent sampling is actually working. The PSNR is always higher with respect to the other methods.
It can be seen that actually all methods suffer the scarcity of frames and as a matter of fact in the last split, 217, 
uniform sampling beats the intelligent one. For the static PSNR instead the Intelligent method is always better
than the uniform one, even at low frames. For the mean Average Precision instead we can see some oscillations,
but we have to be careful since our mask is actually combining the actor and the foreground layer, meaning that 
the average precision is not actually assessing the ability of the model to distinguish these two parts. 
For example in Figure~\ref{fig:comp} we can see that in the 400 frames splits is exactly present this deficit, where the 
uniform sampling is totally unable to detect the actor even though its mAP is higher than the Intelligent 
one($mAP_{Uniform}61.6\% > mAP_{Intelligent}56.63\%$).

Also for our aim to segment dynamic objects, we are interested in the static PSNR (as we obtain dynamic objects as 
what is NOT static) and Figure~\ref{fig:comp} shows us that the static part is almost identical for each scene.




\paragraph{Qualitative Results}On the other hand,going towards qualitative results, here we present the 3D static reconstruction for P01-01.
As shown in Figure~\ref{fig:statP01_01},the first row is the COLMAP pointcloud extracted from the sampled videosequence. Below are placed
instead the static reconstructions for the three different sampling strategies. The first thing that comes to our eyes is 
the overall colour which in the Intelligent sampling seem more faithful to the reality. The second thing is the segmentation of 
the plate on the table top, which can be seen in the COLMAP row. The plate is successfully removed in the Int. sampling while
it is still visible in the other splits, although the best model was the Unif. one according to the metrics. Another example 
is given by the pan highlighted with the green circle which is removed in the Intelligent sampling while not in the others.
We can also look at scene P03-04 in Figure~\ref{fig:col_p03} where Intelligent method manage to remove the can highlighted
in red while the Uniform methods can not.
\begin{figure}[H]
    \centering
    \includegraphics[width=1\linewidth]{images/esperimenti/samplingCorretto.pdf} 
    \caption{Visualization of the sampling of scene P01-01 for the three different methods: Intelligent, Uniform and AU using 217 frames in total.
        The left box is a proposal we gave to visualize how the frames actually spread along the temporal axis, where a line is drawn in correspondence
        of each sample. The right boxes represent instead histograms with the frequencies of the samples on the entire duration of the video.}\label{fig:samplFreq}
\end{figure}

\begin{table}[t]
\centering
\begin{tabular}{|c|c|c|c|c|c|}
\hline \rowcolor[HTML]{A6A6A6}P01-01 & Sampling  & Durata [s] & PSNR & \begin{tabular}{l} 
PSNR \\
statico
\end{tabular} & mAP \\
\hline 2938 & Int. &                                  $11 \mathrm{~h} 36 \min 59 \mathrm{~s}$           & \textbf{24.82}         & 20.41                  & 72.21 \\
\hline 2938 & Unif &                                  $11 \mathrm{~h} 16 \mathrm{~min} 8 \mathrm{~s}$   & 24.42                  & 20.41                  & \textbf{72.79} \\
\hline \rowcolor[HTML]{D9D9D9} 2015 & Int. &          $8 \mathrm{~h} 3 \mathrm{~min} 50 \mathrm{~s}$    & \textbf{24.51}         & \textbf{20.46}         & \textbf{70.49 }\\
\hline \rowcolor[HTML]{D9D9D9} 2015 & Unif &          7h 52min 34s                                      & 23.93                  & 20.33                  & 69.87 \\
\hline 1000 & Int. &                                  3h 43min 41s                                      & \textbf{23.59}         & \textbf{20.37 }        & 67.55 \\
\hline 1000 & Unif &                                  $3 \mathrm{~h} 43 \mathrm{~min} 1 \mathrm{~s}$    & 22.8                   & 20.10                  & 66.51 \\
\hline 1000 & $\mathrm{AU}$ &                         $4 \mathrm{~h} 2 \mathrm{~min} 45 \mathrm{~s}$    & 23.43                  & 20.31                  & \textbf{67.99} \\
\hline \rowcolor[HTML]{D9D9D9} 722 & Int. &           2h $34 \mathrm{~min}$                             & \textbf{22.65}         & \textbf{20.20}         & \textbf{65.42} \\
\hline \rowcolor[HTML]{D9D9D9} 722 & Unif &           $2 \mathrm{~h} 30 \mathrm{~min} 7 \mathrm{~s}$    & 22.09                  & 19.65                  & 62.95 \\
\hline \rowcolor[HTML]{D9D9D9} 722 & $\mathrm{AU}$ &  2h $36 \mathrm{~min} 46 \mathrm{~s}$              & 22.48                  & 20.05                  & 64.10 \\
\hline 397 & Int. &                                   1h $19 \mathrm{~min} 53 \mathrm{~s}$              & \textbf{21.33}         & 19.64                  & 56.63 \\
\hline 397 & Unif &                                   1h $21 \min 42 \mathrm{~s}$                       & 20.98                  & 19.54                  & \textbf{61.6 }\\
\hline 397 & $\mathrm{AU}$ &                          1h 14min 36s                                      & 21.2                   & \textbf{19.95}         & 59.97 \\
\hline \rowcolor[HTML]{D9D9D9} 217 & Int. &           $40 \mathrm{~min} 55 \mathrm{~s}$                 & 20.32                  & \textbf{19.60}         & 51.69 \\
\hline \rowcolor[HTML]{D9D9D9} 217 & Unif &           $41 \mathrm{~min} 8 \mathrm{~s}$                  & \textbf{20.51}         & 19.42                  & \textbf{53.00} \\
\hline \rowcolor[HTML]{D9D9D9} 217 & $\mathrm{AU}$ &  $25 \mathrm{~min} 39 \mathrm{~s}$                 & 20.26                  & 19.39                  & 50.58 \\
\hline
\end{tabular}
\caption{\textbf{NeuralDiff Pipeline Results on P01-01 at 114x64. }For the same scene P01-01 results of NeuralDiff pipeline trained on different amount of frames are reported.The Frames
are selected using the three different sampling strategis:Intelligent,Uniform and AU(see Section~\ref{sec:sampl}). The frames are all at a 114x64 resolution.}\label{tab:EpicInt114}

\end{table}


\begin{figure}[H]
    \hspace{-2cm}
    \centering
    \includegraphics[scale=0.5,angle=90]{images/esperimenti/comparison53-02.png} 
    \caption{\textbf{Qualitative results on P01-01 at 228x128}Visualization of the output of the different models trained on different
    sampling splits for scene P01-01. The first coloumn represent the real frame while the next ones
    are respectively: the predicted image,which is the combination of:the static part, the foreground and the actor part.}\label{fig:comp}
\end{figure}


Other comparison for the same scene P01-01 with different frames splits are provided in Figure~\ref{fig:statP01_02}. In Figure~\ref{fig:reco} we can 
find some other static reconstructions from othere scenes.
\begin{figure}[t]
    \centering
    \includegraphics[width=1\linewidth]{images/esperimenti/nerf_pcd_comparison-06.png} 
    \caption{Qualitative results for the static reconstruction of P01-01 scene at 217 frames. In red is highlighted
            a dynamic plate, while in green a dynamic pan. We can see that Intelligent sampling is correctly
            removing the objects while Uniform can not.}\label{fig:statP01_01}
\end{figure}
\begin{figure}[H]
    \centering
    \includegraphics[width=1\linewidth]{images/esperimenti/Colmaps/P03_04-10.png} 
    \caption{Qualitative results for the static reconstruction of P03-04 scene. In red is highlighted
            a dynamic can that is succesfully removed in Int. sampling while not in Uniform.}\label{fig:col_p03}
\end{figure}
\begin{figure}[H]
    
    \adjustbox{trim={1.5cm 0 0 0}} {\includegraphics[scale=0.7]{images/esperimenti/nerf_pcd_comparison-07.png} }
    \caption{Comparative of the qualitative results for different samplings of the P01-01 scene.}\label{fig:statP01_02}
\end{figure}

\begin{figure}[H]
    \centering
    {\includegraphics[width=1\linewidth]{images/esperimenti/Colmaps/P03_04-06.png} }
    \caption{Qualitative results of different kitchens. On the first row is reported the COLMAP reconstruction
    while below is the corresponding cleaned pointcloud.}\label{fig:reco}
\end{figure}


\paragraph{NeuralDiff Pipeline on all Scenes.}
To further validate our results, we repeated the experiments using different scenes using a split of $\sim$1000 frames and one of $\sim700$.
 The results are reported in Table~\ref{tab:Epic_res_114} for resolution 114x64 and Table~\ref{tab:Epic_res_228} at 
 a resolution of 228x128. The results assess that our method is working.
 \begin{table}[t]
    \resizebox{\textwidth}{!}{
    \rowcolors{2}{gray!25}{white}
    
\begin{tabular}{|c|c|c|c|c|c|c|c|c|c|}
    \hline \textbf{Scene} &\textbf{ Sampled Frames }&\textbf{ Sampling} &\textbf{ Durata [s]} & \textbf{PSNR }& \textbf{Improv. }& \begin{tabular}{l} 
    \textbf{PSNR} \\
    \textbf{statico}
    \end{tabular} &\textbf{ Improv. }& \textbf{mAP} &\textbf{ Improv.} \\
    \hline \multirow[t]{2}{*}{ P01-01 } & \multirow[t]{2}{*}{1089} & Intelligent & 3h $43 \mathrm{~min} 41 \mathrm{~s}$ & 23.59          &  \cellcolor{green!15}\multirow[t]{2}{*}{0.79} & 20.37 &  \cellcolor{green!15}\multirow[t]{2}{*}{0.27} & 67.55 & \cellcolor{green!15} \multirow[t]{2}{*}{1.04} \\
    \hline &                                                        & Uniform & $3 \mathrm{~h} 43 \mathrm{~min} 1 \mathrm{~s}$ & 22.8 & & 20.10 & & 66.51 & \\
    \hline \multirow[t]{2}{*}{ P03-04 } & \multirow[t]{2}{*}{868} & Intelligent & 3h $14 \mathrm{~min} 44 \mathrm{~s}$ & 19.90           & \cellcolor{green!15} \multirow[t]{2}{*}{0.77} & 16.73 & \multirow[t]{2}{*}{0} & 61.92                         & \cellcolor{red!15} \multirow[t]{2}{*}{-2.08} \\
    \hline &                                                        & Uniform & 3h $10 \mathrm{~min} 48 \mathrm{~s}$ & 19.13 & & 16.73 & & 64.00 & \\
    \hline \multirow[t]{2}{*}{ P04-01 } & \multirow[t]{2}{*}{1098} & Intelligent & 4h $10 \min 12 \mathrm{~s}$ & 24.32                   &  \cellcolor{green!15}\multirow[t]{2}{*}{0.51} & 21.23 &  \cellcolor{green!15}\multirow[t]{2}{*}{0.74} & 71.37 & \cellcolor{green!15} \multirow[t]{2}{*}{5.56} \\
    \hline &                                                        & Uniform & 4h $2 \min 9 s$ & 23.81 & & 20.49 & & 65.81 & \\
    \hline \multirow[t]{2}{*}{ P09-02 } & \multirow[t]{2}{*}{903} & Intelligent & $3 \mathrm{~h} 25 \mathrm{~min} 58 \mathrm{~s}$ & 23.97 &  \cellcolor{green!15}\multirow[t]{2}{*}{0.58} & 19.43 & \cellcolor{red!15} \multirow[t]{2}{*}{-0.02} & 60.05 &  \cellcolor{red!15}\multirow[t]{2}{*}{-1.18} \\
    \hline &                                                         & Uniform & $3 \mathrm{~h} 17 \mathrm{~min} 5 \mathrm{~s}$ & 23.39 & & 19.45 & & 61.23 & \\
    \hline \multirow[t]{2}{*}{ P16-01 } & \multirow[t]{2}{*}{1004} & Intelligent & 3h $46 \mathrm{~min} 57 \mathrm{~s}$ & 22.89         & \cellcolor{green!15} \multirow[t]{2}{*}{0.14} & 20.17 &  \cellcolor{green!15}\multirow[t]{2}{*}{0.23} & 66.89 &  \cellcolor{green!15}\multirow[t]{2}{*}{3.28} \\
    \hline &                                                        & Uniform & 3h 43min 11s & 22.75 & & 19.94 & & 63.61 & \\
    \hline \multirow[t]{2}{*}{ P21-01 } & \multirow[t]{2}{*}{855} & Intelligent & 3h $15 \min 59 \mathrm{~s}$ & 20.02                    & \cellcolor{green!15} \multirow[t]{2}{*}{0.91} & 15.73 &  \cellcolor{green!15}\multirow[t]{2}{*}{0.66} & 72.94 &  \cellcolor{green!15}\multirow[t]{2}{*}{4.06} \\
    \hline &                                                        & Uniform & $3 \mathrm{~h} 7 \mathrm{~min} 50 \mathrm{~s}$ & 19.11 & & 15.07 & & 68.88 & \\
    \hline
    \end{tabular}}
    \caption{NeuralDiff models trained on different scenes at $\sim$1000frames, resolution 114x64. The coloumn Improv represents
    the difference between the previous coloumn of the Intelligent split minus the Uniform one.}\label{tab:Epic_res_114}
\end{table}




\begin{table}[t]
    \resizebox{\textwidth}{!}{
    \rowcolors{2}{gray!25}{white}
    
    \begin{tabular}{|c|c|c|c|c|c|c|c|c|c|}
        \hline \textbf{Scene} & \textbf{Sampled Frames} &\textbf{ Sampling} & \textbf{Durata [s]} & \textbf{PSNR} & \textbf{I-U} & \begin{tabular}{l} 
        \textbf{PSNR} \\
        \textbf{statico}
        \end{tabular} & \textbf{I-U} & \textbf{mAP[\%]} & \textbf{I-U} \\
        \hline \multirow[t]{2}{*}{ P01-01 } & \multirow[t]{2}{*}{722} & Intelligent & $10 \mathrm{~h} 25 \mathrm{~min} 35 \mathrm{~s}$ & 21.64 & \cellcolor{green!15}\multirow[t]{2}{*}{0.18} & 19.58 &\cellcolor{green!15} \multirow[t]{2}{*}{0.08} & 66.26 & \cellcolor{green!15}\multirow[t]{2}{*}{1.93} \\
        \hline & &                                                            Uniform& 10h $5 \min 7 \mathrm{~s}$                       & 21.46 & & 19.50 & & 64.33 & \\
        \hline \multirow[t]{2}{*}{ P03-04 } & \multirow[t]{2}{*}{683} &  Intelligent & 9h $40 \mathrm{~min} 5 \mathrm{~s}$              & 18.89 & \cellcolor{green!15}\multirow[t]{2}{*}{0.39} & 16.17 & \cellcolor{red!15}\multirow[t]{2}{*}{-0.15} & 61.08 & \cellcolor{red!15}\multirow[t]{2}{*}{-1.09} \\
        \hline &                                                           & Uniform & 9h $16 \mathrm{~min} 59 \mathrm{~s}$             & 18.5 & & 16.32 & & 62.17 & \\
        \hline \multirow[t]{2}{*}{ P04-01 } & \multirow[t]{2}{*}{779} & Intelligent & 10h $45 \min 46 s$                               & 22.15 & \cellcolor{green!15}\multirow[t]{2}{*}{0.81} & 20.04 & \cellcolor{green!15}\multirow[t]{2}{*}{0.46} & 67.65 &\cellcolor{green!15} \multirow[t]{2}{*}{4.81} \\
        \hline & &                                                           Uniform & 11h $17 \mathrm{~min} 34 \mathrm{~s}$            & 21.34 & & 19.58 & & 62.84 & \\
        \hline \multirow[t]{2}{*}{ P09-02 } & \multirow[t]{2}{*}{622} &  Intelligent & $8 \mathrm{~h} 35 \mathrm{~min}$                 & 22.39 &\cellcolor{green!15}\multirow[t]{2}{*}{0.96} & 19.10 &\cellcolor{green!15} \multirow[t]{2}{*}{0.22} & 67.56 &\cellcolor{green!15} \multirow[t]{2}{*}{9.34} \\
        \hline & &                                                           Uniform & $8 \mathrm{~h} 53 \mathrm{~min} 13 \mathrm{~s}$  & 21.43 & & 18.88 & & 58.22 & \\
        \hline \multirow[t]{2}{*}{ P16-01 } & \multirow[t]{2}{*}{643} & Intelligent & $9 h 5 \min 38 s$                                & 21.25 & \cellcolor{green!15}\multirow[t]{2}{*}{0.38} & 19.38 &\cellcolor{green!15} \multirow[t]{2}{*}{0.39} & 63.34 & \cellcolor{red!15}\multirow[t]{2}{*}{-1.2} \\
        \hline & &                                                           Uniform & 9h 7 min 52s                                     & 20.87 & & 18.99 & & 64.54 & \\
        \hline \multirow[t]{2}{*}{ P21-01 } & \multirow[t]{2}{*}{646} &  Intelligent & $9 h 8 \min 54 s$                                & 18.09 & \cellcolor{red!15}\multirow[t]{2}{*}{-0.23} & 15.20 & \cellcolor{red!15}\multirow[t]{2}{*}{-0.28} & 65.49 &\cellcolor{red!15} \multirow[t]{2}{*}{-3.11} \\
        \hline & &                                                           Uniform & $9 \mathrm{~h} 8 \min 7 \mathrm{~s}$             & 18.32 & & 15.48 & & 68.60 & \\
        \hline
        \end{tabular}}
        \caption{NeuralDiff models trained on different scenes at $\sim$700frames, resolution 228x128. The coloumn I-U represents
        the difference between the previous coloumn of the Intelligent split minus the Uniform one.}\label{tab:Epic_res_228}
\end{table}




\paragraph{Sampling and Action Distributions. } Other than the idea behind the Intelligent sampling which was expressed in Section~\ref{sec:sampl}, we found a
link between the positions of the sampled frames and the frequencies of the objects/actions that were annotated during the videos.
In Figure~\ref{fig:samplFreq} the three sampling methods are reported. 
In Figure~\ref{fig:objectP01_01} and Figure~\ref{fig:objectsScenes} are reported respectively
the comparison of Intelligent and Uniform sampling with the objects count for scene P01-01 changing number of samples,
as can be read on each sub-figure; and the comparison of Intelligent and Uniform sampling with the objects count for each scene
with fixed sampling at $\sim1000$ frames.
 
As we can see from the plots,at exception from few scenes, the profile of the Intelligent sampling looks closer
to the one of the object counts. This is giving a further explanation of the functioning of our proposed method.
In fact it means that our sampling is focusing on those areas where a lot of actions are performed. In this way
long redundant actions are filtered and only relevant frames, where multiple actions happens, are kept. 

\begin{figure}[H]
    \centering
    \includegraphics[width=1\linewidth]{images/esperimenti/ObjectCountsScenes-02.png} 
    \caption{Comparison of frequencies for the Intelligent and Uniform sampling with the Object Count for the P01-01 scene changing the total
    number of sampled frames.}\label{fig:objectP01_01}
\end{figure}

\begin{figure}[H]
    \centering
    \includegraphics[width=1\linewidth]{images/esperimenti/ObjectCountsScenes-01.png} 
    \caption{Comparison of frequencies for the Intelligent and Uniform sampling with the Object Count for each scene at a fixed split \~1000 frames.}\label{fig:objectsScenes}
\end{figure}

\paragraph{Metrics for Profile similarity. }We also tried to give a quantitative measure of similarity and dissimilarity by comparing some 
different metrics: Cosine Similarity, Kullback-Leibler Divergence (KLD),
Jensen-Shannon Divergence (JSD).
More in details:
\begin{itemize}
    \item \textbf{Cosine Similarity.} It is the cosine of the angle between two vectors. It is derived from the dot product:  
    \begin{equation}
        \mathbf{v_{1}} \cdot \mathbf{v_{2}} =\left\lVert\mathbf{v_{1}} \right\rVert \left\lVert\mathbf{v_{2}} \right\rVert \cos(\theta) 
    \end{equation}
    \begin{equation}
        CosineSimilarity=\cos(\theta) = \frac{\mathbf{v_{1}} \cdot \mathbf{v_{2}} }{\left\lVert\mathbf{v_{1}} \right\rVert \left\lVert\mathbf{v_{2}} \right\rVert }
    \end{equation}
    In our case we took as vectors the bins of the frames histograms as vector one and the bins of the actions/objects as the second one.
    \item \textbf{Kullback-Leibler Divergence (KLD).} It is a non symmetric measure of the difference between two probability distributions
    P and Q. It represent the measure of the lost information when Q is used to approximate P.
    \begin{equation}
        D_{\mathrm{KL}}(P \| Q)=\sum_i P(i) \log _2\left(\frac{P(i)}{Q(i)}\right)
    \end{equation}
    \item \textbf{Jensen-Shannon Divergence (JSD).} It is based on the Kullback–Leibler divergence, but 
    it is modifief to be symmetric and always having a finite value.
    \begin{equation}
        \operatorname{JSD}(P \| Q)=\frac{1}{2} D(P \| M)+\frac{1}{2} D(Q \| M)
    \end{equation}
    where $M = \frac{1}{2}(P+Q)$ is a mixture distribution of P and Q.
\end{itemize}
Clearly we do not have distributions, so we obtained them by normalizing such that their elements summed to 1. The results we have obtained
are reported in Table~\ref{tab:samplMetrics}.
\begin{table}
\resizebox{\textwidth}{!}{\begin{tabular}{|c|
    >{\columncolor[HTML]{C0C0C0}}c | >{\columncolor[HTML]{C0C0C0}}c|c|c|
    >{\columncolor[HTML]{C0C0C0}}c|>{\columncolor[HTML]{C0C0C0}}c|c|c|c|}
\hline \rowcolor[HTML]{A6A6A6} & \multicolumn{2}{|c|}{ Cosine Similarity } & \multicolumn{2}{|c|}{ K-L Div. } & \multicolumn{2}{|c|}{ JS-Div } & \multicolumn{2}{|c|}{ Correlation } \\
\hline \rowcolor[HTML]{A6A6A6}& Intelligent & Uniform & Intelligent & Uniform & Intelligent & Uniform & Intelligent & Uniform \\
\hline P01-01 & 0.91319715 & 0.90718451 & 0.113071 & 0.13421 & 0.03931 & 0.0450167 & 0.5632 & 0.5391 \\
\hline P03-04 & 0.7638 & 0.6773 & 0.4224 & 0.5389 & 0.16092 & 0.2025 & 0.5483 & 0.2226 \\
\hline P04-01 & 0.6528 & 0.6316 & 0.5892 & 0.6631 & 0.2196 & 0.2438 & 0.1454 & 0.1353 \\
\hline P09-02 & 0.7438 & 0.7323 & 2.2962 & 1.2859 & 0.1377 & 0.1399 & -0.1736 & -0.1152 \\
\hline P16-01 & 0.8029 & 0.8130 & 0.2170 & 0.2431 & 0.0727 & 0.0815 & 0.3490 & 0.4416 \\
\hline P21-01 & 0.8804 & 0.8761 & 0.1694 & 0.1760 & 0.0574 & 0.0611 & 0.3633 & 0.1889 \\
\hline
\end{tabular}}
\caption{Metrics for comparing the profile of the histograms. In particular higher values of Cosine similarity and Correlation indicates similarity; while
the value of the two divergences represents the distance between the two distributions.}

\end{table}


\begin{comment}
\begin{table}
    \resizebox{\textwidth}{!}{\begin{tabular}{|c|c|c|c|c|c|c|c|c|}
    \hline \rowcolor[HTML]{A6A6A6} & \multicolumn{2}{|c|}{ Cosine Similarity } & \multicolumn{2}{|c|}{ K-L Div. } & \multicolumn{2}{|c|}{ JS-Div } & \multicolumn{2}{|c|}{ Correlation } \\
    \hline \rowcolor[HTML]{A6A6A6}& Intelligent & Uniform & Intelligent & Uniform & Intelligent & Uniform & Intelligent & Uniform \\
    \hline P01-01 & 0.91319715 & 0.90718451 & 0.113071 & 0.13421 & 0.03931 & 0.0450167 & 0.5632 & 0.5391 \\
    \hline P03-04 & 0.7638 & 0.6773 & 0.4224 & 0.5389 & 0.16092 & 0.2025 & 0.5483 & 0.2226 \\
    \hline P04-01 & 0.6528 & 0.6316 & 0.5892 & 0.6631 & 0.2196 & 0.2438 & 0.1454 & 0.1353 \\
    \hline P09-02 & 0.7438 & 0.7323 & 2.2962 & 1.2859 & 0.1377 & 0.1399 & -0.1736 & -0.1152 \\
    \hline P16-01 & 0.8029 & 0.8130 & 0.2170 & 0.2431 & 0.0727 & 0.0815 & 0.3490 & 0.4416 \\
    \hline P21-01 & 0.8804 & 0.8761 & 0.1694 & 0.1760 & 0.0574 & 0.0611 & 0.3633 & 0.1889 \\
    \hline
    \end{tabular}}
    \caption{Metrics for comparing the profile of the histograms. In particular higher values of Cosine similarity and Correlation indicates similarity; while
    the value of the two divergences represents the distance between the two distributions.}
    
    \end{table}
\end{comment}

\section{NeuralCleaner}
As a last step, after assessing that our method was functioning, we tried to speed up the overall pipeline.
We tried to eliminate the actor layer of the NeuralDiff pipeline since we are not interested in it, as we 
only want to know what is moving. The foreground and actor layer are thus merged together and we tried to see
how much we could improve by eliminating this distinction. In Table~\ref{tab:NClean} we can see the results in comparison with
the NeuralDiff pipeline. On average we obtained a descreasing factor of $\sim30\%$.

\begin{table}[t]
    \resizebox{\textwidth}{!}{
    
    \rowcolors{2}{gray!25}{white}    
    \begin{tabular}{|c|c|c|c|c|c|c|c|}
    \hline \textbf{P01-01} & \textbf{Sampling }& \begin{tabular}{l} 
    \textbf{Durata} \\
    \textbf{Ndiff[s]}
    \end{tabular} & \textbf{Durata [s]} & \textbf{Improvement} \% & \textbf{PSNR} & \begin{tabular}{l} 
    \textbf{PSNR} \\
    \textbf{statico}
    \end{tabular} & \textbf{mAP} \\
    \hline 1089 & Int & 3h 43min 41s & 2h $55 \mathrm{~min} 30 \mathrm{~s}$ & -21.54 & 23.49 & 19.92 & 61.00 \\
    \hline 1089 & Unif & 3h $43 \mathrm{~min} 1 \mathrm{~s}$ & 2h $58 \mathrm{~min} 56 \mathrm{~s}$ & -19.76 & 22.86 & 19.55 & 56.40 \\
    \hline 722 & Int & $2 \mathrm{~h} 34 \mathrm{~min}$ & 1h 46min 41s & -30.72 & 22.51 & 19.47 & 50.03 \\
    \hline 722 & Unif & $2 \mathrm{~h} 30 \mathrm{~min} 7 \mathrm{~s}$ & 1h 46min 8s & -29.29 & 22.29 & 19.46 & 51.83 \\
    \hline 397 & Int & 1h 19min 53s & $54 \mathrm{~min} 27 \mathrm{~s}$ & -31.83 & 21.46 & 19.72 & 53.69 \\
    \hline 397 & Unif & 1h 21min 42s & $57 \mathrm{~min} 30 \mathrm{~s}$ & -29.62 & 21.01 & 19.51 & 50.60 \\
    \hline 217 & Int & $40 \mathrm{~min} 55 \mathrm{~s}$ & $28 \mathrm{~min} 49 \mathrm{~s}$ & -29.57 & 20.55 & 18.20 & 40.87 \\
    \hline 217 & Unif & $41 \mathrm{~min} 8 \mathrm{~s}$ & 29 min 13s & -28.97 & 20.51 & 19.45 & 46.61 \\
    \hline
    \end{tabular}}
    \caption{Results for NeuralCleaner, compared with the durations of the NeuralDiff pipeline.
    We can see that the durations are on average shorter of $\sim$30\%.}\label{tab:NClean}
\end{table}

