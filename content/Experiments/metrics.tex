\section{Metrics}\label{sec:Metrics}
As regards metrics we looked in literature for a way to evaluate
our results but unfortunately each method involved a ground truth
which for our dataset is not available. Possible ways to obtain
a groundtruth could be manual annotations or simulating the environments.
Both these two methods would take a considerable large amount
of time and are also beyond the scope of this thesis.

For this reason we ended up by using the metrics proposed in~\cite{neuraldiff}.
Namely these are:
\begin{itemize}
    \item \textbf{PSNR}
    \item \textbf{mAP}
\end{itemize}
\subsection{PSNR:Peak signal-to-noise ratio}
The Peak Signal-to-Noise Ratio (PSNR) is a metric commonly used in image and
video processing to quantify the quality of a reconstructed or processed signal,
like an image or video. It gives a measures of the ratio between the 
maximum possible power of a signal (MAX) and the power of the distortion or noise 
that affects the signal (MSE).

The formula for PSNR is usually expressed in decibels (dB) and is given by:

\[ \text{PSNR} = 20 \cdot \log_{10}\left(\frac{{\text{MAX}}}{{\text{MSE}}}\right) \]

where:
\begin{itemize}
    \item MAX is the maximum possible pixel value of the image (1 in our case).
    \item MSE is the Mean Squared Error, which represents the average squared 
    difference between the original signal and the reconstructed or distorted signal.
\end{itemize}

It is worth noting that a high PSNR does not guarantee that the processed signal 
will be perceived as visually pleasing or high-quality by humans, especially in the case of perceptually sensitive applications like image and video compression.

\subsection{AP:Average Precision}
Average Precision (AP) is a metric commonly used in object detection and information retrieval
to evaluate the performance of machine learning models. It measures the \textit{precision-recall} trade-off of a model.

It can be useful to remind what \textit{Precision} and \textit{Recall} are. Namely:

\begin{equation}
    Precision=\frac{TP}{TP+FP}
\end{equation}
\begin{equation}
    Recall=\frac{TP}{TP+FN}
\end{equation}

where:
\begin{itemize}
    \item TP=True positive
    \item FP=False positive
    \item TN=True negative
\end{itemize}
Average precision is then computed as the area below the precision-recall curve, specifically
the curve obtained by varying the confidence threshold of the inference model as shown
in Figure \ref{fig:auc}. That is why 
it can also be found in literature as AUC(Area Under Curve).
Its scalar value summarize the precision-recall performance of the model.
A higher AP is desirable, indicating a model that effectively retrieves 
relevant instances while minimizing false positives.

\begin{figure}[t]
    \centering
    \includegraphics[width=0.5\linewidth]{images/metrics/auc.png} % Replace "example-image" with your image file name
    \caption{Example of Precision-recall curve.We can see how the bottom line model 
        represents the worst a model can perform, e.g. predict every sample as it is 
        coming from the same class,if the dataset is balanced. A better model would
        \textit{tend} to the upper-right corner, which instead represents the best
        possible model, a model that have maximum precision and recall.}\label{fig:auc}
\end{figure}